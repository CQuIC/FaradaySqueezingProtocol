\documentclass[preprint,aps,pra,onecolumn,superscriptaddress]{revtex4-1} %reprint
%\tightenlines

%\draft
\usepackage{etex}
\usepackage{amsmath}
\usepackage{bm}
\usepackage{bbm}
\usepackage{listings}
% % \textwidth 16cm \textheight 23.5cm
% \renewcommand{\baselinestretch}{1.2}
\usepackage{graphicx}
\usepackage{graphics}
\usepackage{epsfig}
\usepackage{color}
\usepackage[dvipsnames]{xcolor}
\usepackage{multirow}
\usepackage[colorlinks]{hyperref}
\usepackage{fancyhdr}
\usepackage{calc}
\usepackage{natbib} %[numbers]
\usepackage{bibentry}
\usepackage{bbm}

% todo list and commands
%\usepackage{todonotes}
%% to avoid the conflict with amths package % not working
%\makeatletter
%\providecommand\@dotsep{5}
%\makeatother
%\listoftodos\relax
%\usepackage{makeidx}
%\allowdisplaybreaks
%% for eps transfering to pdf.
%\usepackage[update,prepend]{epstopdf}
%\usepackage{ifpdf}
%
%\ifpdf
%   \usepackage{graphicx}
%   \usepackage{epstopdf}
%   \epstopdfsetup{suffix=}
%   \DeclareGraphicsRule{.eps}{pdf}{.pdf}{`epstopdf #1}
%   \pdfcompresslevel=9
%\else
%   \usepackage{graphicx}
%\fi
% subfig
%\usepackage{mwe}
%\usepackage{subfig}
% to fix a figure's position using [H] option of thec figure.
%\usepackage{float}
% to use \lesssim and other math symbols
%\usepackage{amssymb}


% self-defined short-cuts and commands
%\input{Mydef.tex}
\DeclareMathOperator{\tr}{tr}
\newcommand{\dt}[1]{\frac{{\mathrm d} {#1}}{{\mathrm d}t}}
\def\br{\mathbf{r}}
\def\bra#1{\langle{#1}\rvert}%{\mathinner{\langle{#1}\rvert}}
\def\ket#1{\lvert{#1}\rangle}%{\mathinner{\lvert{#1}\rangle}}
\def\Braket#1#2{\mathinner{\langle{#1}\! \mid\! {#2} \rangle}}
%========================================================================================
\newcommand{\erf}[1]{Eq.~(\ref{#1})}
\newcommand{\frf}[1]{Fig.~\ref{#1}}
\newcommand{\srf}[1]{Sec.~\ref{#1}}
\newcommand{\nn}{\nonumber}
\newcommand{\mbf}[1]{\mathbf{#1}}
%========================================================================================
% General quantum mechanics macros
%========================================================================================
\newcommand{\op}[2]{\ket{#1}\bra{#2}}
\newcommand{\expt}[1]{\langle{#1}\rangle}
\newcommand{\dg}{^\dagger}
\newcommand{\smallfrac}[2]{\mbox{$\frac{#1}{#2}$}}
\newcommand{\Tr}{\mbox{Tr}}
%========================================================================================
\newcommand{\expect}[1]{\big\langle #1 \big\rangle}
\newcommand{\eff}{\text{eff}}



% Redefine the tensor command.
%\renewcommand{\tensor}[1]{\boldsymbol{#1}}


%==== Ben's new macros ======
%\newcommand{\srf}[1]{Sec. \ref{#1}}
\newcommand{\half}{\smallfrac{1}{2}}

%==== subscripts ======
\newcommand{\oneD}{{\rm 1D}}
\newcommand{\vac}{{\rm vac}}
\newcommand{\cav}{{\rm cav}}
\newcommand{\inp}{{\rm in}}
\newcommand{\out}{{\rm out}}
\newcommand{\inter}{{\rm int}}
\newcommand{\scs}{{\rm SCS}}
\newcommand{\fwd}{+}
\newcommand{\bwd}{-}
\newcommand{\trans}{+}
\newcommand{\refl}{-}

 %==== operators/moments ======
\newcommand{\der}[1]{\frac{d {#1}}{dt}}
\newcommand{\unittens}{\tensor{\mathbf{I}}}
\newcommand{\poltens}{\hat{\tensor{\boldsymbol{\alpha}}}}
\newcommand{\varz}{\Delta J_3^2}
\newcommand{\jx}{\hat{J}_1}
\newcommand{\jz}{\hat{J}_3}
\newcommand{\shotnoise}{\Delta \mathcal{M}^2 |_{\rm SN}}
\newcommand{\projnoise}{\Delta \mathcal{M}^2_{\rm PN}}
\newcommand{\polcomp}{\hat{K}} % p,p' component of the tensor polarizability
\newcommand{\fo}{\hat{\mathbf{f}}}
\newcommand{\fx}{\hat{f}_x}
\newcommand{\fy}{\hat{f}_y}
\newcommand{\fz}{\hat{f}_z}
\newcommand{\Fx}{\hat{F}_x}
\newcommand{\Fy}{\hat{F}_y}
\newcommand{\Fz}{\hat{F}_z}
\newcommand{\rhoo}{\hat{\rho}}

%==== Microscopic moments for the qubit/qutrit subspace ====
\newcommand{\sigmauu}{\hat{\sigma}_{\uparrow\uparrow}}
\newcommand{\sigmaud}{\hat{\sigma}_{\uparrow\downarrow}}
\newcommand{\sigmaut}{\hat{\sigma}_{\uparrow \mathrm{T}}}
\newcommand{\sigmadu}{\hat{\sigma}_{\downarrow\uparrow}}
\newcommand{\sigmadd}{\hat{\sigma}_{\downarrow\downarrow}}
\newcommand{\sigmadt}{\hat{\sigma}_{\downarrow \mathrm{T}}}
\newcommand{\sigmatu}{\hat{\sigma}_{\mathrm{T}\uparrow}}
\newcommand{\sigmatd}{\hat{\sigma}_{\mathrm{T}\downarrow}}
\newcommand{\sigmatt}{\hat{\sigma}_{\mathrm{T}\mathrm{T}}}
\newcommand{\sigmaab}{\hat{\sigma}_{ab}}
\newcommand{\sigmaba}{\hat{\sigma}_{ba}}
\newcommand{\sigmadc}{\hat{\sigma}_{dc}}
\newcommand{\sigmacd}{\hat{\sigma}_{cd}}

\newcommand{\Dsigmauu}{\Delta\sigma_{\uparrow\uparrow}}
\newcommand{\Dsigmaud}{\Delta\sigma_{\uparrow\downarrow}}
\newcommand{\Dsigmaut}{\Delta\sigma_{\uparrow \mathrm{T}}}
\newcommand{\Dsigmadu}{\Delta\sigma_{\downarrow\uparrow}}
\newcommand{\Dsigmadd}{\Delta\sigma_{\downarrow\downarrow}}
\newcommand{\Dsigmadt}{\Delta\sigma_{\downarrow \mathrm{T}}}
\newcommand{\Dsigmatu}{\Delta\sigma_{\mathrm{T}\uparrow}}
\newcommand{\Dsigmatd}{\Delta\sigma_{\mathrm{T}\downarrow}}
\newcommand{\Dsigmatt}{\Delta\sigma_{\mathrm{T}\mathrm{T}}}
\newcommand{\Dsigmaab}{\Delta\sigma_{ab}}
\newcommand{\Dsigmaba}{\Delta\sigma_{ba}}
\newcommand{\Dsigmadc}{\Delta\sigma_{dc}}
\newcommand{\Dsigmacd}{\Delta\sigma_{cd}}

%==== physical parameters ======
\newcommand{\Eamp}{\mathcal{F}_0^{(+)}}
\newcommand{\charpol}{\alpha_0(\Delta_{f\!f'})}
\newcommand{\charpolq}{\alpha_0(\Delta_{f\!f'}^q)}
\newcommand{\qaxis}{\mathbf{e}_{\tilde{z}}}
\newcommand{\qangle}{\varphi}
\newcommand{\magic}[1]{\tilde{\omega}_{#1}}
\newcommand{\chiN}{\chi_{N}}
\newcommand{\NA}{N_C}
\newcommand{\chieff}{\chi_{\raisebox{-.1pt}{\tiny $J_3$}}}

%==== scattering and optical pumping rates ====%
\newcommand{\gammauu}{\gamma_{\uparrow \rightarrow \uparrow}}
\newcommand{\gammadd}{\gamma_{\downarrow \rightarrow \downarrow}}
\newcommand{\gammaud}{\gamma_{\uparrow \rightarrow \downarrow}}
\newcommand{\gammadu}{\gamma_{\downarrow \rightarrow \uparrow}}
\newcommand{\gammau}{\gamma_{\uparrow}}
\newcommand{\gammad}{\gamma_{\downarrow}}

%==== effective areas ======
\newcommand{\Ain}{A_{\rm in}}
\newcommand{\Abir}{A_N}
\newcommand{\AF}{A_F} % for the Faraday protocol.
\newcommand{\Ai}{A_0} % for the input light.
\newcommand{\Aint}{A_{\rm int}} % for the interaction area.

%==== eigenfunctions ======
\newcommand{\eigenf}{\mbf{f}_\eta}
\newcommand{\eigenfp}{\mbf{f}_{\eta'}}
\newcommand{\eigeng}{\mbf{g}_\eta}
\newcommand{\eigengp}{\mbf{g}_{\eta'}}

%==== field operators ======
\newcommand{\awg}{\hat{a}_{b,p}(\omega)}
\newcommand{\awr}{\hat{a}_{m,p}(\omega,\beta)}

%==== colors for editing ======
\newcommand{\change}[1]{{\color{RoyalBlue} #1}}
\newcommand{\comment}[1]{{\color{Maroon} #1}}
\newcommand{\error}[1]{{\color{red} #1}}

% =============================================================================


\begin{document}
\title{Enhanced cooperativity for QND measurement-induced spin squeezing of atoms coupled to a nanophotonic waveguide}
\author{Xiaodong Qi}
\affiliation{Center for Quantum Information and Control, University of New Mexico, Albuquerque, New Mexico 87131, USA}
\author{Yuan-Yu Jau}
\affiliation{Center for Quantum Information and Control, Sandia National Laboratories, Albuquerque, New Mexico 87185, USA}
\author{Ivan H. Deutsch}
\affiliation{Center for Quantum Information and Control, University of New Mexico, Albuquerque, New Mexico 87131, USA}
\date{\today}
\pacs{42.50.Lc, 03.67.Bg, 42.50.Dv, 42.81.Gs}

%================================================================%
\begin{abstract}
We study the enhancement of cooperativity in the atom-light interface near a nanophotonic waveguide for application to QND measurement of atomic spins.  Here the cooperativity per atom is determined by the ratio between the  measurement strength and the decoherence rate.  Counterintuitively, by placing the atoms at an azimuthal position where the guided probe mode has the lowest intensity, we increase the cooperativity.  This arises because the QND measurement strength depends on the interference between the probe and scattered light guided into an orthogonal polarization mode, while the decoherence rate depends on the local intensity of the probe.  Thus, by proper choice of geometry, the ratio of good to bad scattering can be strongly enhanced for highly anisotropic modes. We apply this to study spin squeezing for two nanophotonic geometries, a cylindrical nanofiber and square waveguide, and for two QND interactions, linear and circular birefringence (Faraday interaction), based on a full stochastic master equation model.  We find�
\end{abstract}

\maketitle

%===================INTRODUCTION=====================%
\section{Introduction}

Cooperativity is the essential measure of the entangling strength of the atom-light interface in quantum optics.  Originally introduced in cavity QED, the cooperativity per atom can be expressed in terms of the ratio of the coherent coupling to decoherence rate, $C_1 = g^2/(\Gamma_c \Gamma_A)$ where $g$ is the vacuum Rabi frequency,  $\Gamma_c$ is the cavity decay rate, and and $\Gamma_A$ is atomic spontaneous emission rate out of the cavity.  Alternatively, we can write $C_1 = (\sigma_0/A) \mathcal{F}$, where $\sigma_0$ is the resonant photon scattering cross section of the atom, $A$ is the cavity mode area, and $\mathcal{F}$ is the cavity finesse.  Expressed this way, cooperativity is seen to arise due to scattering of photons preferentially into the cavity mode, compared to emission into free space, here enhanced by the finesses due to the Purcell effect. Strong coupling dynamics seen in pioneering in atomic cavity QED~\cite{?} is now a mainstay in quantum information processing in systems ranging from quantum dots~\cite{?} to circuit QED~\cite{}.  The $N_A$ atom cooperativity, $C_N = (N_A \sigma_0/A) \mathcal{F} =( OD) \mathcal{F}$, where $OD$ is the resonant optical depth.  In this configuration, collective degrees of the atom can be manipulated by its common coupling to the cavity mode.

Cooperativity also characterizes the atom-light interface in the absence of a cavity.  In free space, an atom at the waist of a laser beam will scatter into the forward direction at a rate $\kappa = (\sigma_0/A) \gamma_s$, where $\gamma_s$ is the photon scattering rate into $4 \pi$ steradians.  Here the single atom cooperativity can be expressed of the ratio of these rates, $C_1 = \kappa/\gamma_s = \sigma_0/A$.  The $N_A$ atom cooperativity, in a plane wave approximation, ignoring effects of diffraction and cloud geometry, $C_N = N_A \sigma_0/A = OD$.  To be self-consistent, here the beam area must be very large, so $C_1$ is very small, e.g. $C_1 \sim 10^{-6}$, but for a sufficiently large ensemble, the $OD$ can be large enough to lead to entanglement between the collective atomic degrees of freedom and the light~\cite{?}.  In that situation, measurement of the light leads to back action on the ensemble, and for an appropriate QND interaction, results in squeezing of the collective spin~\cite{?}.

In recent years, nanophotonic waveguides have emerged as new geometry that complements cavity QED, and can lead to strong cooperativity~\cite{?}.  Notably, the effective area of a tightly guided mode can be much smaller than free space and propagate for long distances without diffraction.  As such,  $\sigma_0/A$ can be orders of magnitude larger than in free space, e.g., $\sigma_0/A \sim 0.1$, and contribute collectively for a modest ensemble of a few thousand atoms trapped near the surface of the waveguide.  Moreover, in some cases the Purcell effect can further enhance forward scattering into the guided mode when compared with scattering into free space.  Taken together, these features make  nanophotonic waveguides a promising platform for a quantum atom-light interface~\cite{?}.
 
In this paper we show that one can achieve an additional enhancement to the cooperativity in a nanophotonic geometry that is not possible in free space. In particular, we consider QND measurement of the atomic spin via polarization spectroscopy.  Counterintuitively, we will see that the strongest cooperativity arises when the atom is placed at the position of {\em minimum} intensity of the azimuthally anisotropic probe mode.  This arises because decoherence due to photon scattering into unguided $4 \pi$ steradians occurs at a rate $\gamma_s$ dominated by the probe {\em intensity}.  In contrast, the measurement strength in polarization polarization spectroscopy arises from a homodyne measurement, where the probe is ``local oscillator" and the signal is the light scattered into the orthogonally polarized guided mode.  The the measurement strength depends only on the {\em amplitude} of the probe. Nonetheless, the spatial overlap of these probe and signal modes can be large, even when the local oscillator small.  The net result can be substantial enhancement of  $C_1 = \kappa/\gamma_s$, especially for highly anisotropic guided modes.  

We study this effect for two nanophotonic geometries: a cylindrical nanofiber formed by tapering a standard optical fiber, as currently employed in a variety of experimental studies~\cite{?}, and a square waveguide, currently nanofabricated at Sandia National Laboratories.  For each geometry we study QND measurement-induced squeezing of collective spins of cesium atoms in two protocols: squeezing the pseudo-spin clock transition by measuring the induced birefringence~\cite{Qi2016}, and squeezing the physical spin in the stretched state by measuring the induced Faraday rotation~\cite{?}.  Through the enhanced cooperativity, QND measurement can lead to substantial squeezing, greater than 10 dB in some geometries, for 2500 atoms.

The remainder of the paper is organized as follows.  In Sec. II we lay out the theoretical description of the QND measurement and the relevant measurement strength.  In addition we describe how decoherence is included in the model through a first-principles stochastic master equation description.  From this we will see the cooperativity emerge as the key parameter that characterizes the squeezing.  We calculate in Sec. III the squeezing dynamics for the different nanophotonic waveguides, atomic preparations, and measurement protocols.  We conclude with a summary and outlook for future work.





%========================== Theory ===================================%
\section{Theory} \label{Sec::Theory}

\bibliography{refs/Archive}

\end{document}
