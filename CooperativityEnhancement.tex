\documentclass[preprint,aps,pra,onecolumn,superscriptaddress]{revtex4-1} %reprint
%\tightenlines

%\draft
\usepackage{etex}
\usepackage{amsmath}
\usepackage{bm}
\usepackage{bbm}
\usepackage{listings}
% % \textwidth 16cm \textheight 23.5cm
% \renewcommand{\baselinestretch}{1.2}
\usepackage{graphicx}
\usepackage{graphics}
\usepackage{epsfig}
\usepackage{color}
\usepackage[dvipsnames]{xcolor}
\usepackage{multirow}
\usepackage[colorlinks]{hyperref}
\usepackage{fancyhdr}
\usepackage{calc}
\usepackage{natbib} %[numbers]
\usepackage{bibentry}
\usepackage{bbm}

% todo list and commands
%\usepackage{todonotes}
%% to avoid the conflict with amths package % not working
%\makeatletter
%\providecommand\@dotsep{5}
%\makeatother
%\listoftodos\relax
%\usepackage{makeidx}
%\allowdisplaybreaks
%% for eps transfering to pdf.
%\usepackage[update,prepend]{epstopdf}
%\usepackage{ifpdf}
%
%\ifpdf
%   \usepackage{graphicx}
%   \usepackage{epstopdf}
%   \epstopdfsetup{suffix=}
%   \DeclareGraphicsRule{.eps}{pdf}{.pdf}{`epstopdf #1}
%   \pdfcompresslevel=9
%\else
%   \usepackage{graphicx}
%\fi
% subfig
%\usepackage{mwe}
%\usepackage{subfig}
% to fix a figure's position using [H] option of thec figure.
%\usepackage{float}
% to use \lesssim and other math symbols
%\usepackage{amssymb}


% self-defined short-cuts and commands
%\input{Mydef.tex}
\DeclareMathOperator{\tr}{tr}
\newcommand{\dt}[1]{\frac{{\mathrm d} {#1}}{{\mathrm d}t}}
\def\br{\mathbf{r}}
\def\bra#1{\langle{#1}\rvert}%{\mathinner{\langle{#1}\rvert}}
\def\ket#1{\lvert{#1}\rangle}%{\mathinner{\lvert{#1}\rangle}}
\def\Braket#1#2{\mathinner{\langle{#1}\! \mid\! {#2} \rangle}}
%========================================================================================
\newcommand{\erf}[1]{Eq.~(\ref{#1})}
\newcommand{\frf}[1]{Fig.~\ref{#1}}
\newcommand{\srf}[1]{Sec.~\ref{#1}}
\newcommand{\nn}{\nonumber}
\newcommand{\mbf}[1]{\mathbf{#1}}
%========================================================================================
% General quantum mechanics macros
%========================================================================================
\newcommand{\op}[2]{\ket{#1}\bra{#2}}
\newcommand{\expt}[1]{\langle{#1}\rangle}
\newcommand{\dg}{^\dagger}
\newcommand{\smallfrac}[2]{\mbox{$\frac{#1}{#2}$}}
\newcommand{\Tr}{\mbox{Tr}}
%========================================================================================
\newcommand{\expect}[1]{\big\langle #1 \big\rangle}
\newcommand{\eff}{\text{eff}}



% Redefine the tensor command.
%\renewcommand{\tensor}[1]{\boldsymbol{#1}}


%==== Ben's new macros ======
%\newcommand{\srf}[1]{Sec. \ref{#1}}
\newcommand{\half}{\smallfrac{1}{2}}

%==== subscripts ======
\newcommand{\oneD}{{\rm 1D}}
\newcommand{\vac}{{\rm vac}}
\newcommand{\cav}{{\rm cav}}
\newcommand{\inp}{{\rm in}}
\newcommand{\out}{{\rm out}}
\newcommand{\inter}{{\rm int}}
\newcommand{\scs}{{\rm SCS}}
\newcommand{\fwd}{+}
\newcommand{\bwd}{-}
\newcommand{\trans}{+}
\newcommand{\refl}{-}

 %==== operators/moments ======
\newcommand{\der}[1]{\frac{d {#1}}{dt}}
\newcommand{\unittens}{\tensor{\mathbf{I}}}
\newcommand{\poltens}{\hat{\tensor{\boldsymbol{\alpha}}}}
\newcommand{\varz}{\Delta J_3^2}
\newcommand{\jx}{\hat{J}_1}
\newcommand{\jz}{\hat{J}_3}
\newcommand{\shotnoise}{\Delta \mathcal{M}^2 |_{\rm SN}}
\newcommand{\projnoise}{\Delta \mathcal{M}^2_{\rm PN}}
\newcommand{\polcomp}{\hat{K}} % p,p' component of the tensor polarizability
\newcommand{\fo}{\hat{\mathbf{f}}}
\newcommand{\fx}{\hat{f}_x}
\newcommand{\fy}{\hat{f}_y}
\newcommand{\fz}{\hat{f}_z}
\newcommand{\Fx}{\hat{F}_x}
\newcommand{\Fy}{\hat{F}_y}
\newcommand{\Fz}{\hat{F}_z}
\newcommand{\rhoo}{\hat{\rho}}

%==== Microscopic moments for the qubit/qutrit subspace ====
\newcommand{\sigmauu}{\hat{\sigma}_{\uparrow\uparrow}}
\newcommand{\sigmaud}{\hat{\sigma}_{\uparrow\downarrow}}
\newcommand{\sigmaut}{\hat{\sigma}_{\uparrow \mathrm{T}}}
\newcommand{\sigmadu}{\hat{\sigma}_{\downarrow\uparrow}}
\newcommand{\sigmadd}{\hat{\sigma}_{\downarrow\downarrow}}
\newcommand{\sigmadt}{\hat{\sigma}_{\downarrow \mathrm{T}}}
\newcommand{\sigmatu}{\hat{\sigma}_{\mathrm{T}\uparrow}}
\newcommand{\sigmatd}{\hat{\sigma}_{\mathrm{T}\downarrow}}
\newcommand{\sigmatt}{\hat{\sigma}_{\mathrm{T}\mathrm{T}}}
\newcommand{\sigmaab}{\hat{\sigma}_{ab}}
\newcommand{\sigmaba}{\hat{\sigma}_{ba}}
\newcommand{\sigmadc}{\hat{\sigma}_{dc}}
\newcommand{\sigmacd}{\hat{\sigma}_{cd}}

\newcommand{\Dsigmauu}{\Delta\sigma_{\uparrow\uparrow}}
\newcommand{\Dsigmaud}{\Delta\sigma_{\uparrow\downarrow}}
\newcommand{\Dsigmaut}{\Delta\sigma_{\uparrow \mathrm{T}}}
\newcommand{\Dsigmadu}{\Delta\sigma_{\downarrow\uparrow}}
\newcommand{\Dsigmadd}{\Delta\sigma_{\downarrow\downarrow}}
\newcommand{\Dsigmadt}{\Delta\sigma_{\downarrow \mathrm{T}}}
\newcommand{\Dsigmatu}{\Delta\sigma_{\mathrm{T}\uparrow}}
\newcommand{\Dsigmatd}{\Delta\sigma_{\mathrm{T}\downarrow}}
\newcommand{\Dsigmatt}{\Delta\sigma_{\mathrm{T}\mathrm{T}}}
\newcommand{\Dsigmaab}{\Delta\sigma_{ab}}
\newcommand{\Dsigmaba}{\Delta\sigma_{ba}}
\newcommand{\Dsigmadc}{\Delta\sigma_{dc}}
\newcommand{\Dsigmacd}{\Delta\sigma_{cd}}

%==== physical parameters ======
\newcommand{\Eamp}{\mathcal{F}_0^{(+)}}
\newcommand{\charpol}{\alpha_0(\Delta_{f\!f'})}
\newcommand{\charpolq}{\alpha_0(\Delta_{f\!f'}^q)}
\newcommand{\qaxis}{\mathbf{e}_{\tilde{z}}}
\newcommand{\qangle}{\varphi}
\newcommand{\magic}[1]{\tilde{\omega}_{#1}}
\newcommand{\chiN}{\chi_{N}}
\newcommand{\NA}{N_C}
\newcommand{\chieff}{\chi_{\raisebox{-.1pt}{\tiny $J_3$}}}

%==== scattering and optical pumping rates ====%
\newcommand{\gammauu}{\gamma_{\uparrow \rightarrow \uparrow}}
\newcommand{\gammadd}{\gamma_{\downarrow \rightarrow \downarrow}}
\newcommand{\gammaud}{\gamma_{\uparrow \rightarrow \downarrow}}
\newcommand{\gammadu}{\gamma_{\downarrow \rightarrow \uparrow}}
\newcommand{\gammau}{\gamma_{\uparrow}}
\newcommand{\gammad}{\gamma_{\downarrow}}

%==== effective areas ======
\newcommand{\Ain}{A_{\rm in}}
\newcommand{\Abir}{A_N}
\newcommand{\AF}{A_F} % for the Faraday protocol.
\newcommand{\Ai}{A_0} % for the input light.
\newcommand{\Aint}{A_{\rm int}} % for the interaction area.

%==== eigenfunctions ======
\newcommand{\eigenf}{\mbf{f}_\eta}
\newcommand{\eigenfp}{\mbf{f}_{\eta'}}
\newcommand{\eigeng}{\mbf{g}_\eta}
\newcommand{\eigengp}{\mbf{g}_{\eta'}}

%==== field operators ======
\newcommand{\awg}{\hat{a}_{b,p}(\omega)}
\newcommand{\awr}{\hat{a}_{m,p}(\omega,\beta)}

%==== Famous names ======
\usepackage{xspace}
\newcommand{\Poincare}{Poincar\'e\xspace}

%==== colors for editing ======
\newcommand{\change}[1]{{\color{RoyalBlue} #1}}
\newcommand{\comment}[1]{{\color{Maroon} #1}}
\newcommand{\error}[1]{{\color{red} #1}}

% =============================================================================


\begin{document}
\title{Enhanced cooperativity for QND measurement-induced spin squeezing of atoms coupled to a nanophotonic waveguide}
\author{Xiaodong Qi}
\affiliation{Center for Quantum Information and Control, University of New Mexico, Albuquerque, New Mexico 87131, USA}
\author{Yuan-Yu Jau}
\affiliation{Center for Quantum Information and Control, Sandia National Laboratories, Albuquerque, New Mexico 87185, USA}
\author{Ivan H. Deutsch}
\affiliation{Center for Quantum Information and Control, University of New Mexico, Albuquerque, New Mexico 87131, USA}
\date{\today}
\pacs{42.50.Lc, 03.67.Bg, 42.50.Dv, 42.81.Gs}

%================================================================%
\begin{abstract}
We study the enhancement of cooperativity in the atom-light interface near a nanophotonic waveguide for application to QND measurement of atomic spins.  Here the cooperativity per atom is determined by the ratio between the  measurement strength and the decoherence rate.  Counterintuitively, we find that by placing the atoms at an azimuthal position where the guided probe mode has the lowest intensity, we increase the cooperativity.  This arises because the QND measurement strength depends on the interference between the probe and scattered light guided into an orthogonal polarization mode, while the decoherence rate depends on the local intensity of the probe.  Thus, by proper choice of geometry, the ratio of good to bad scattering can be strongly enhanced for highly anisotropic modes. We apply this to study spin squeezing resulting from QND measurement of spin projection noise via the Faraday effect in two nanophotonic geometries, a cylindrical nanofiber and square waveguide.  We find�
\end{abstract}

\maketitle

%===================INTRODUCTION=====================%
\section{Introduction}

Cooperativity is the essential measure of the entangling strength of the atom-light interface in quantum optics.  Originally introduced in cavity QED, the cooperativity per atom can be expressed in terms of the ratio of the coherent coupling to decoherence rate, $C_1 = g^2/(\Gamma_c \Gamma_A)$ where $g$ is the vacuum Rabi frequency,  $\Gamma_c$ is the cavity decay rate, and and $\Gamma_A$ is atomic spontaneous emission rate out of the cavity.  Alternatively, we can write $C_1 = (\sigma_0/A) \mathcal{F}$, where $\sigma_0$ is the resonant photon scattering cross section of the atom, $A$ is the cavity mode area, and $\mathcal{F}$ is the cavity finesse.  Expressed this way, cooperativity is seen to arise due to scattering of photons preferentially into the cavity mode, compared to emission into free space, here enhanced by the finesses due to the Purcell effect. Strong coupling dynamics seen in pioneering in atomic cavity QED~\cite{?} is now a mainstay in quantum information processing in systems ranging from quantum dots~\cite{?} to circuit QED~\cite{}.  The $N_A$ atom cooperativity, $C_N = (N_A \sigma_0/A) \mathcal{F} =( OD) \mathcal{F}$, where $OD$ is the resonant optical depth.  In this configuration, collective degrees of the atom can be manipulated by its common coupling to the cavity mode.

Cooperativity also characterizes the atom-light interface in the absence of a cavity.  In free space, an atom at the waist of a laser beam will scatter into the forward direction at a rate $\kappa = (\sigma_0/A) \gamma_s$, where $\gamma_s$ is the photon scattering rate into $4 \pi$ steradians.  Here the single atom cooperativity can be expressed of the ratio of these rates, $C_1 = \kappa/\gamma_s = \sigma_0/A$.  The $N_A$ atom cooperativity, in a plane wave approximation, ignoring effects of diffraction and cloud geometry, $C_N = N_A \sigma_0/A = OD$.  To be self-consistent, here the beam area must be very large, so $C_1$ is very small, e.g. $C_1 \sim 10^{-6}$, but for a sufficiently large ensemble, the $OD$ can be large enough to lead to entanglement between the collective atomic degrees of freedom and the light~\cite{?}.  In that situation, measurement of the light leads to back action on the ensemble, and for an appropriate QND interaction, results in squeezing of the collective spin~\cite{?}.

In recent years, nanophotonic waveguides have emerged as a new geometry that complements cavity QED, and can lead to strong cooperativity~\cite{?}.  Notably, the effective area of a tightly guided mode can be much smaller than free space and propagate for long distances without diffraction.  As such,  $\sigma_0/A$ can be orders of magnitude larger than in free space, e.g., $\sigma_0/A \sim 0.1$, and contribute collectively for a modest ensemble of a few thousand atoms trapped near the surface of the waveguide.  Moreover, in some cases the Purcell effect can further enhance forward scattering into the guided mode when compared with scattering into free space.  Taken together, these features make  nanophotonic waveguides a promising platform for a quantum atom-light interface~\cite{?}.
 
In this paper we show that one can achieve an additional enhancement to the cooperativity in a nanophotonic geometry that is not possible in free space. In particular, we consider QND measurement of the atomic spin via a Faraday interaction and polarization spectroscopy.  In this configuration the polarimeter effectively performs a homodyne measurement, where the probe is the ``local oscillator" that interferes with the light scattered into the orthogonally polarized guided mode.   This signal thus depends on the spatial overlap of the two orthogonal modes at the position of the atom.  In contrast, decoherence due to photon scattering into unguided $4 \pi$ steradians occurs at a rate $\gamma_s$ is determined only by the intensity of the probe.   The net result is that cooperativity per atom, $C_1 = \kappa/\gamma_s$, depends only on the strength of the orthogonal mode, and this factor can be enhanced, especially for highly anisotropic guided modes.  Counterintuitively, we will see that the strongest cooperativity arises when the atom is placed at the position of  minimum intensity of the azimuthally anisotropic probe mode where the  intensity of the initially unoccupied orthogonal mode is maximum.    

We study this effect for two nanophotonic geometries: a cylindrical nanofiber formed by tapering a standard optical fiber, as recently employed in a variety of experimental studies~\cite{?}, and a square waveguide, currently nanofabricated at Sandia National Laboratories~\cite{?|}.  For each geometry we study the use of the Faraday effect to perform a QND measurement of the magnetic spins, and thereby induce squeezing of collective spins of cesium atoms~\cite{?}.  Through the enhanced cooperativity, QND measurement can lead to substantial squeezing, greater than 10 dB in some geometries, for 2500 atoms.

The remainder of the paper is organized as follows.  In Sec. II we lay out the theoretical description of the QND measurement and the relevant measurement strength.  In addition we describe how decoherence is included in the model through a first-principles stochastic master equation description.  From this we will see the cooperativity emerge as the key parameter that characterizes the squeezing.  We calculate in Sec. III the squeezing dynamics for the different nanophotonic waveguides, atomic preparations, and measurement protocols.  We conclude with a summary and outlook for future work.


%========================== Theory ===================================%
\section{QND measurement and cooperativity} \label{Sec::Theory}
The theoretical framework describing the propagation of light guided in a nanofiber and interacting with trapped atoms in the dispersive regime is detailed in our previous work~\cite{Qi2016}.  We review the salient features here and include the generalization to a square waveguide.

For waveguides that are symmetric under a $\pi/2$ rotation around the $z$ (propagation) axis, there are two degenerate polarizations for each guided mode and for each propagation direction.  Assuming a nanophotonic waveguide that supports only the lowest order guided mode, and restricting our attention to modes propagating in the positive $z$-direction, we denote $\mbf{u}_H(\mbf{r}_\perp)$ and  $\mbf{u}_V(\mbf{r}_\perp)$ as the horizontally and vertically polarized modes that adiabatically connect to $x$ and $y$ linearly polarized modes as the cross section of the waveguide become large compared to optical wavelength.  Note, in typical nanophotonic geometries, these guided modes also have a nonnegligible $z$ component.  For a cylindrically symmetric nanofiber, these are the well-studied HE$_{11}$ modes; for a square waveguide, these are the TE$_{01}$ and TM$_{01}$ modes, both shown in Fig. 1 and and described in Appendix A.

The quasimonochromatic positive frequency component of the quantized field associated with these guided modes $(g)$ at frequency $\omega_0$ takes the form
\begin{align}\label{eq:Ebp}
\hat{\mathbf{E}}^{(+)}_g(\mbf{r}, t) &= \sqrt{ \frac{2 \pi \hbar \omega_0}{ v_g} } \left[\mathbf{u}_H(\mbf{r}\!_\perp)  \hat{a}_H(t) + \mathbf{u}_V(\mbf{r}\!_\perp) \hat{a}_V(t)\right]  e^{i (\beta_0 z- \omega_0 t)},
\end{align}
where $v_g$ is the group velocity.  In the first Born approximation the dispersive interaction of the guided field with $N_A$ atoms trapped near the surface of the waveguide at positions $\{\mbf{r}'_\perp, z_n\}$, detuned far from resonance,  is defined by the scattering equation,
\begin{equation}
\hat{\mathbf{E}}^{(+)}_{g,out}(\mbf{r}, t)=\hat{\mathbf{E}}^{(+)}_{g,in}(\mbf{r}, t)+\sum_{n=1}^{N_A} \tensor{\mbf{G}}_{g} (\mbf{r}, \mbf{r}'_n,\omega_0) \cdot \hat{\tensor{\alpha}}_n \cdot \hat{\mathbf{E}}^{(+)}_{g,in}(\mbf{r}'_n, t),
\end{equation}
where $\hat{\tensor{\alpha}}_n$ is the atomic polarizability operator of the $n^{th}$ atom, and 
\begin{equation}
		\tensor{\mathbf{G}}^{(+)}_g(\br,\br'_n; \omega_0) =  2\pi i \frac{\omega_0}{v_g } \sum_{p} \mathbf{u}_{p} (\br_\perp)\mathbf{u}^*_{p} 
(\br_{\perp}^\prime) e^{i \beta_0(z-z'_n)}  \label{Eq::GreensGuided}
\end{equation}
is the dyadic Green's function for a dipole to radiate into the forward propagating guided mode.  In principle the Green's function for a $N_A$-atom chain decomposes into a collective sub- and superradiant normal modes~\cite{Asenjo-Garcia2017Atom,Asenjo-Garcia2017Exponential}, but in the far-detuning limit, these all are equally excited.  The result is equivalent to the symmetric mode of independently radiating dipoles.  The input-output relation for the mode operators then reads
\begin{equation}
\hat{a}^{out}_p(t) = \hat{a}^{in}_p(t)  +i \sum_{p'} \hat{\phi}_{p,p'} \hat{a}^{in}_{p'}(t) ,
\end{equation}
where 
\begin{equation}
\hat{\phi}_{p,p'} = 2\pi \frac{\omega_0}{v_g} \mbf{u}^*_p (\mbf{r}'_\perp) \cdot \sum_{n=1}^{N_A} \hat{\tensor{\alpha}}_n \cdot \mbf{u}_{p'} (\mbf{r}'_\perp)
\end{equation}
is the phase operator associated with scattering polarization $p \rightarrow p'$ by a collective atomic operator.  When $p=p'$ is this a phase shift; for $p \neq p'$ this leads to a transformation of the polarization of the guided mode.

The Faraday effect arises from the irreducible rank-1 (vector) component of the polarizability tensor.  Given an atom with hyperfine spin $f$, this contribution is $\hat{\alpha}^{vec}_{ij} = i \alpha_1 \epsilon_{ijk} \hat{f}_k$, where $\alpha_1 = -\frac{\sigma_0}{4\pi k_0 }\frac{\Gamma_A}{\Delta_{\eff}} $ is the characteristic polarizability, expressed here in terms of an effective on the detuning and resonant scattering cross section~\footnote{The exact expression is a sum of excited-state hyperfine levels $f'$, each weighted by the appropriate oscillator strengths of the transition.  We have defined $1/\Delta_{\eff} = \sum_{f'} C^{(1)}_{ff'}/\Delta_{ff'}$. For details see~\cite{?}}.   The polarization transformation associated with scattering from $H$ to $V$ mode is set by the phase operator
\begin{equation}
\hat{\phi}_{VH} = i 2\pi \frac{\omega_0}{v_g}\alpha_1 \left[ \mbf{u}^*_V (\mbf{r}'_\perp) \times  \mbf{u}_{H} (\mbf{r}'_\perp) \right] \cdot \hat{\mbf{F}},
\end{equation}
where $\hat{\mbf{F}}=\sum_n \hat{\mbf{f}}^{(n)}$ is the collective spin of the atomic ensemble.  Thus,
\begin{equation}
\hat{a}^{out}_V(t) = \hat{a}^{in}_V(t)  +i  \hat{\phi}_{V,H} \hat{a}^{in}_{H}(t)= \hat{a}^{in}_V(t)  - 2\pi \frac{\omega_0}{v_g}\alpha_1 \left[ \mbf{u}^*_V (\mbf{r}'_\perp) \times  \mbf{u}_{H}\right(\mbf{r}'_\perp)]  \cdot \hat{\mbf{F}}\, \hat{a}^{in}_{H}(t),
\end{equation}
 and similarly for scattering from $V$ to $H$.

The polarization transformation can be expressed as a rotation of the Stokes vector of the light on the \Poincare sphere with operator components
\begin{subequations}\label{Eq::StokesComponents}
	\begin{align}
		\hat{S}_1(t) & = \smallfrac{1}{2}\big[ \hat{a}^\dag_H(t) \hat{a}_H(t)-\hat{a}^\dag_V(t) \hat{a}_V(t) \big], \\
	 	\hat{S}_2(t) & = \smallfrac{1}{2}\big[ \hat{a}^\dag_H(t) \hat{a}_V(t)+\hat{a}^\dag_V(t) \hat{a}_H(t) \big], \\ 
		\hat{S}_3(t) & = \smallfrac{1}{2i}\big[ \hat{a}^\dag_H(t) \hat{a}_V(t) -\hat{a}^\dag_V(t) \hat{a}_H(t) \big].
	\end{align}
\end{subequations}
By measuring the output Stokes vector in a polarimeter, we perform a QND measurement of a collective atomic operator to which it was entangled.  In a proper configuration, this leads to squeezing of a collective spin.  Launching $H$-polarized light corresponds to the initial Stokes vector along $S_1$, and Faraday rotation leads to an $S_2$ component, which is measured in a polarimeter (Fig. 2a).  Taking the $H$-mode as a coherent state with amplitude $\beta_H$, the signal of the polarimeter measures $\hat{S}_2^{out} = (\beta_H \hat{a}_V^{\dag out} +\beta^*_H \hat{a}_V^{out})/2$.  Expressed in this way we see that the polarimeter acts as a homodyne detector, with the input $H$-mode acting as the local oscillator and the photons scattered into the $V$-mode as the signal.  Formally, the input-out relation follows from the scattering equations, Eq. (?), and reads
\begin{equation}
\hat{S}^{out}_2 = \hat{S}^{in}_2 +i \big( \hat{\phi}_{VH}- \hat{\phi}_{HV} \big) \hat{S}^{in}_1 =  \hat{S}^{in}_2 + \chi_3(\mbf{r}'_\perp) \hat{F}_z \hat{S}^{in}_1.
\end{equation}
The first term $\hat{S}^{in}_2$ represents the shot-noise that fundamentally limits the resolution of spin squeezing that can be obtained in a given time interval.  The second term is the homodyne signal, where we have expressed the rotation angle around the 3-axis as
\begin{equation}
\chi_3(\mbf{r}'_\perp) = -\frac{4 \pi \omega_0}{v_g} \alpha_1 \left\vert \text{Re} \left[ \mbf{u}^*_V (\mbf{r}'_\perp) \times \mbf{u}_H (\mbf{r}'_\perp) \right] \right\vert = \frac{\sigma_0}{A_{Far}(\mbf{r}'_\perp)} \frac{\Gamma_A}{2 \Delta_{\eff}}.
\end{equation}
We emphasize here the dependence of the rotation angle on the position of the atom in the transverse plane, $\mbf{r}'_\perp$, assumed equal for all atoms in the chain.  In particular $\chi_3(\mbf{r}'_\perp)$ depends on the {\em overlap} of the $\mbf{u}_H (\mbf{r}'_\perp)$ and $\mbf{u}_V (\mbf{r}'_\perp)$ , indicative of atomic scattering of photons from the $H$ to $V$ modes associated with the Faraday interaction.  We have characterized this overlap by an effective area that defines the Faraday interaction at the position of the atom,
\begin{equation}
A_{Far}(\mbf{r}'_\perp) = \frac{1}{n_g \left\vert \text{Re} \left[ \mbf{u}^*_V (\mbf{r}'_\perp) \times \mbf{u}_H (\mbf{r}'_\perp) \right]\right\vert},
\end{equation}
where $n_g = v_g/c$ is the group index.  A more tightly confined (smaller) area corresponds to a stronger interaction.
By ignoring the transition rate difference among the excited states in the same hyperfine manifold, the optimal choice of quantization axis is determined by the real cross production of the two orthogonal fundamental modes, $ \text{Re} \left[ \mbf{u}^*_V (\mbf{r}'_\perp) \times \mbf{u}_H (\mbf{r}'_\perp) \right] $.
For both waveguide geometries we are interested in, or cylindrical waveguides in general, the real cross product of the modes and hence the optimal quantization $ z $-axis is along the waveguide axis or the $ z $-direction.

By monitoring the Faraday rotation, we can perform a continuous measurement on the collective spin projection $\hat{F}_z$.  The ``measurement strength'' which characterizes the rate at which we gain information and thereby squeeze the spin is given by
\begin{equation}
\kappa = \left\vert \chi_3(\mbf{r}'_\perp) \right\vert^2 \frac{P_{in}}{\hbar \omega_0},
\end{equation}
where $P_{in}$ is the input power transported into the guided mode.  The measurement strength is the rate at which photons are scattered from the guided $H$ to $V$ mode.  Decoherence arises due to diffuse scattering into unguided modes and the accompanied optical pumping of the spin.  In principle the diffuse photon scattering rate is modified over free space due to the Purcell effect, but we neglect this correction here.  In the case of the nanofiber, this is a small effect at typical distances at which the atom is trapped~\cite{?}.  For the square waveguide, this may be nonnegligible; we will return to correction in future work.  The free-space diffuse photon scattering rate is given by
\begin{equation}
\gamma_s = \sigma(\Delta_{\eff}) \frac{I_{in}(\mbf{r}'_\perp)}{\hbar \omega_0}
\end{equation}
where $\sigma(\Delta_{\eff}) = \sigma_0 \frac{\Gamma_A^2}{4 \Delta^2_{\eff}}$ is the photon scattering cross-section at this detuning and  $I_{in}(\mbf{r}'_\perp) = n_g P_{in}\vert \mbf{u}_H (\mbf{r}'_\perp)  \vert^2 \equiv P_{in}/A_{in}(\mbf{r}'_\perp) $ is the input intensity into the guided $H$-mode at the position of the atom, where we have defined
\begin{equation}
A_{in}(\mbf{r}'_\perp) =  \frac{1}{n_g \vert \mbf{u}_H (\mbf{r}'_\perp) \vert ^2}
\end{equation}
to be the effective area associated with the input mode.  We thus define the cooperativity/atom
\begin{equation}
C_1 (\mbf{r}'_\perp)  = \frac{\kappa}{\gamma_s} = \sigma_0 \frac{  A_{in}(\mbf{r}'_\perp) }{[A_{Far}(\mbf{r}'_\perp)]^2}.
\end{equation}
This is the central result.  $A_{Far}(\mbf{r}'_\perp)$  appears the square because of the relation of the measurement strength to the homodyne measurement.  Roughly, $1/[A_{Far}(\mbf{r}'_\perp)]^2 \sim \vert \mbf{u}_V (\mbf{r}'_\perp) \vert ^2 \vert \mbf{u}_H (\mbf{r}'_\perp) \vert ^2$, thus $ C_1(\mbf{r}'_\perp) \sim \sigma_0 \vert \mbf{u}_V (\mbf{r}'_\perp) \vert ^2$. In the context of homodyne measurement, the signal to be measured is proportional to the real overlap between the $ H $- and $ V $-modes, while the local oscillator or $ H $-mode generates decoherent noise. How large the local vacant $ V $-mode is at the atom positions--not the local input mode--basically determines the signal to noise ratio and eventually the atom-light coupling in the context of QND measurement. 

In fact, 
\begin{align}\label{eq:c1_bound}
C_1\le \sigma_0\frac{n_g \vert \mbf{u}_V (\mbf{r}'_\perp) \vert ^2 \vert \mbf{u}_H (\mbf{r}'_\perp) \vert ^2}{\vert \mbf{u}_H (\mbf{r}'_\perp) \vert ^2}=\sigma_0n_g \vert \mbf{u}_V (\mbf{r}'_\perp) \vert ^2,
\end{align}
where the equal sign is hardly reachable for a dielectric waveguide which usually has an out-of-phase $ z $-component in either or both of the orthogonal base modes at arbitrary positions, while the upper bound naturally occurs in free-space modes basis and $ C_1=\frac{\sigma_0}{A_{in}}=\frac{\sigma_0}{\vert \mbf{u}_H (\mbf{r}'_\perp) \vert ^2} $ well-known for a Gaussian or plane wave input modes in free-space when $ \vert \mbf{u}_H (\mbf{r}'_\perp) \vert ^2=\vert \mbf{u}_V (\mbf{r}'_\perp) \vert ^2 $ independent of azimuthal position and $ n_g=1 $. However, as we will show next that due to the tight and anisotropic confinement of the modes in a nanophotonic structure, the cooperativity can be enhanced by choosing the position of the atom so that the {\em orthogonal}, unoccupied mode is large, while the intensity that causes decoherence is small.

Figure ? shows a plot of $1/A_{Far}$, $1/A_{in}$, and $C_1$ as a function of  $\mbf{r}'_\perp$ for the two nanophotonic geometries.  We see that $A_{Far}$ is cylindrically symmetric for both nanofiber and SWG geometries which have a $ C_4 $ rotation symmetry, and thus the measurement strength is independent of the azimuthal position, depending only on $\vert \mbf{r}'_\perp\vert$.  In contrast $A_{in}$ is azimuthally anisotropic.  $1/A_{in}$ is smallest along the $y$-axis at a given radial distance, which corresponds to smallest intensity of the $H$-mode, and thus smallest scattering rate $\gamma_s$.  This angle corresponds to the position at which $\vert \mbf{u}_V (\mbf{r}'_\perp) \vert$ is largest and thus yields the largest enhancement of $C_1$.  Thus, counterintuitively, we enhance the cooperativity by placing the atom at the angle of minimum input intensity.   This enhancement is even greater for the square waveguide, as the modes are even more anisotropic in this case.  For typical geometries, given a nanofiber with radius $a = 250$nm and atoms trapped and a radial distance $?$ from the surface, the single atom cooperativity is $C_1 =?$ at the optimal trapping angle; for the square waveguide of width $w =?$nm, and atoms trapped $?$nm from the surface, $C_1 =?$ at optimum.  Thus with order 1000 trapped atoms the $N_A$-atom cooperativity is of order ?, sufficient to generate substantial spin squeezing.

\section{Spin squeezing dynamics}

Given an ensemble of $N_A$ atoms initially prepared in a spin coherent state for the hyperfine spin $f$, polarized in the transverse plane, e.g., along the $x$-axis, a QND measurement of the collective spin  $F_z$ will squeeze the uncertainty of that component.  The metrologically relevant squeezing parameter defined by Wineland {\emph{et al.}}~\cite{Wineland1992} is,
\begin{align}\label{eq:xi2Faraday}
\xi^2 &\equiv  \frac{2 f N_A\expect{\Delta F_z ^2}}{\expect{\hat{F}_x}^2}.
\end{align}
Under the assumption that the state is symmetric with respect to exchange of any two atoms, the collective expectation value can be decomposed into \begin{align}
\expect{\Delta F_\perp^2} &= N_A \expect{\Delta f_z^2}+N_A(N_A-1)\left. \expect{\Delta f_z^{(i)}\Delta f_z^{(j)}}\right|_{i\neq j}\label{eq:DeltaFz2}\\
\expect{\hat{F}_x } & =N_A \expect{\hat{f}_x},\label{eq:expectFx}
\end{align}
where the first term of Eq.~\eqref{eq:DeltaFz2} and Eq.~\eqref{eq:expectFx} is the projection noise associated with the  $N_A$ identical spin-$f$  atoms, and  the second term of Eq.~\eqref{eq:DeltaFz2} is determined by two-body covariances, $ \left.\expect{\Delta f_\perp^{(i)}\Delta f_\perp^{(j)}}\right|_{i\neq j}=\expect{\Delta f_z^{(1)}\Delta f_z^{(2)}} = \expect{\hat{f}_z^{(1)}\hat{f}_z^{(2)}}-\expect{\hat{f}_z^{(1)}} \expect{\hat{f}_z^{(1)}} $.  Negative values in these two-body correlations correspond to the pairwise entanglement among atoms that yields spin squeezing~\cite{Wang2003Spin}.  Note that the collective state of atoms can be treated as pairwise-symmetric if the detuning is far-off resonance so that all collective sub- and super-radiant modes~\cite{Asenjo-Garcia2017Atom,Asenjo-Garcia2017Exponential} are equally (and thus symmetrically) excited.  In this paper, we work in the dispersive regime with a few thousands of atoms and can safely ignore the atom-atom interaction caused by multiple scattering, and hence the collective atomic system satisfy the exchange symmetry. 

To study the spin squeezing dynamics, we employ a first-principles stochastic master equation for the collective state of $N_A$ atoms,
\begin{align}\label{eq:totaldrhodt}
\mathrm{d}\hat{\rho}= \left.\mathrm{d}\hat{\rho}\right|_{QND}+\left.\mathrm{d}\hat{\rho}\right|_{op}.
\end{align}
The first term on the right-hand side of Eq.\eqref{eq:totaldrhodt} governs the spin dynamics arising from QND measurement,
\begin{align}
\left.\mathrm{d}\hat{\rho}\right|_{QND} &= \sqrt{\frac{\kappa}{4}}\mathcal{H}\left[\hat{\rho} \right]\mathrm{d}W + \frac{\kappa}{4}\mathcal{L}\left[ \hat{\rho}\right]\mathrm{d}t, 
\end{align}
where  $\kappa$ is the measurement strength defined in Eq. (?), and $\mathrm{d}W$ is a stochastic Weiner interval. The conditional dynamics are generated by superoperators that depend on the {\em collective} spin
\begin{subequations}
\begin{align}
\mathcal{H}\left[ \hat{\rho}\right] &= \hat{F}_z \hat{\rho} + \hat{\rho}\hat{F}_z -2\expect{\hat{F}_z}\hat{\rho}, \\
\mathcal{L}\left[ \hat{\rho} \right] &= \hat{F}_z \hat{\rho}\hat{F}_z -\frac{1}{2}\left(\hat{\rho}\hat{F}_z^2+\hat{F}_z^2\hat{\rho} \right)=\frac{1}{2}\left[\hat{F}_z,\left[\hat{\rho},\hat{F}_z \right] \right].
\end{align}
\end{subequations}
The second term governs decoherence arising from optical pumping, which acts {\em locally} on each atom$,\mathrm{d}\hat{\rho}|_{op}=\sum_i^{N_A} \mathcal{D}^{(i)}\left[ \hat{\rho}\right] \mathrm{d}t$, where 
\begin{equation}
\mathcal{D}^{(i)}\left[ \hat{\rho}\right] = -\frac{i}{\hbar}\left(\hat{H}^{(i)}_{\rm eff}\hat{\rho} - \hat{\rho} \hat{H}^{(i)\dag}_{\rm eff}\right) + \gamma_s\sum_q \hat{W}^{(i)}_q \hat{\rho}\hat{W}^{(i)\dag}_q.
\end{equation}
Here $\hat{H}^{(i)}_{\rm eff}$ is the effective nonHermitian Hamiltonian describing the local light shift and absorption by the $i^{th}$ atom and $\hat{W}^{(i)}_q$ is the jump operator corresponding to optical pumping through spontaneous emission of a photon of polarization $q$~\cite{Deutsch2010a} (see Appendix ?).   
The rate of decoherence is characterized by the free space scattering rate, Eq. (?).  Note, optical pumping superoperator, Eq. (?),  is not trace preserving when restricted to a given hyperfine manifold $f$.  In this case, optical pumping of atoms to the other hyperfine manifold in the ground-electronic state is treated as loss.  If the atoms are placed at the optimal angle, the local field is linearly polarized.  In that case the vector light shift vanishes, and for detunings large compared to the excited-state hyperfine splitting, the rank-2 tensor light shift is negligible.  In that case the light shift is dominated by the scalar component, which has no effect on the spin dynamics.  In that case $\hat{H}_{\rm eff} \propto -i\hbar \gamma_s 1$.

The solution to the master equation is made possible by three approximations. Firstly, we restrict the subspace of internal magnetic sublevels that participate in the dynamics.  The system is initialized in a spin coherent state, with all atoms spin-polarized along the $x$-axis.  We denote this as the ``fiducial state" $\ket{\uparrow} = \ket{f, m_x =f}$.   Through QND measurement, spin squeezing is induced by entanglement with the  ``coupled state"  $\ket{\downarrow} = \ket{f, m_x=f-1}$.  Optical pumping is dominated by ``spin flips" $\ket{\uparrow}\rightarrow \ket{\downarrow}$ and ``loss" due to pumping to the other hyperfine level.  We finally include a third internal magnetic sublevel $\ket{T} = \ket{f, m_x=f-2}$ to account for  ``transfer of coherences" that can occur in spontaneous emission~\cite{Norris2014}.  Restricted to this qutrit basis with dimension $ d=3 $, the internal hyperfine spin operators are
\begin{subequations}
\begin{align}
\hat{f}_x &= -\left[f \hat{\sigma}_{\uparrow \uparrow} +(f-1) \hat{\sigma}_{\downarrow \downarrow} + (f-2)  \hat{\sigma}_{T T}\right], \\
\hat{f}_z &= \sqrt{\frac{f}{2}} \left(\hat{\sigma}_{\uparrow \downarrow} + \hat{\sigma}_{\downarrow \uparrow}\right) + \sqrt{\frac{2f-1}{2}}  \left(\hat{\sigma}_{\downarrow T} + \hat{\sigma}_{T \downarrow }\right),
\end{align}
\end{subequations}
where we have defined the atomic population and coherence operators $\hat{\sigma}_{ba}=\ket{b}\bra{a}$.

Secondly, we assume the collective spin state conserves its form under exchange symmetry. This approximation is valid in the far-detuning regime where the atom-atom interactions ($ \sim 1/\Delta^2 $) are negligible~\cite{Asenjo-Garcia2017Atom,Asenjo-Garcia2017Exponential}. With this, we can limit our attention to the symmetric subspace and define, for example, the symmetric two-body covariances by
\begin{align}
\expect{\Delta\sigma_{ba}^{(1)}\Delta\sigma_{dc}^{(2)}}_s \equiv \frac{1}{2}\left[\expect{\Delta\sigma_{ba}^{(1)}\Delta\sigma_{dc}^{(2)}}+\expect{\Delta\sigma_{ba}^{(2)}\Delta\sigma_{dc}^{(1)}} \right] ,
\end{align}
where the superscripts, $ ^{(1)} $ and $ ^{(2)} $, label arbitrary two atoms in the ensemble. Due to the exchange symmetry, $ \expect{\Delta\sigma_{ba}^{(1)}\Delta\sigma_{dc}^{(2)}}_s=\expect{\Delta\sigma_{ba}^{(1)}\Delta\sigma_{dc}^{(2)}}=\expect{\Delta\sigma_{ba}^{(2)}\Delta\sigma_{dc}^{(1)}} $ and reduces the number of $ n $-body moments required to compute the spin dynamics of the ensemble.

Thirdly, we make the Gaussian approximation, valid for large atomic ensembles, so that the many-body state is fully characterized by one- and two-body correlations. Equivalently, the state is defined by the one and two-body density operators, with matrix elements $\rho^{(1)}_{a, b} =\expect{\hat{\sigma}_{ba}}$, $\rho^{(1,2)}_{ac,bd}=\expect{\Delta \sigma_{ba}^{(1)}\Delta\sigma_{dc}^{(2)} }_s$ in the symmetric subspace.   Optical pumping, acting locally, couple only $n$-body correlations to themselves, e.g.,
\begin{equation}
\left.d\expect{\Delta \sigma_{ba}^{(1)}\Delta\sigma_{dc}^{(2)} }_s\right|_{op} = \expect{\mathcal{D}[\Delta \sigma_{ba}^{(1)}]\Delta\sigma_{dc}^{(2)} }_sdt + \expect{\Delta \sigma_{ba}^{(1)} \mathcal{D}[\Delta\sigma_{dc}^{(2)}] }_sdt .
\end{equation}
QND measurement generates higher order correlations according to
\begin{equation}
\left.d\expect{\hat{\sigma}_{ba}}\right|_{QND} =\frac{\kappa}{4}\expect{\mathcal{L}^\dagger\left[\hat{\sigma}_{ba} \right]}dt + \sqrt{\frac{\kappa}{4}}\expect{\mathcal{H}^\dagger\left[\hat{\sigma}_{ba} \right]}dW .
\end{equation}
We can truncate this hierarchy in the Gaussian approximation, setting third order cumulants to zero.  Thus, for example,
\begin{align}
\left.d\expect{\Delta \sigma_{ba}^{(1)} \Delta \sigma_{dc}^{(2)}}_s \right|_{QND} &= \left.d\expect{\hat{\sigma}_{ba}^{(1)} \hat{\sigma}_{dc}^{(2)}}_s \right|_{QND} - \left. \expect{\hat{\sigma}_{ba}} \right|_{QND} \left( \left.d\expect{\hat{\sigma}_{dc}} \right|_{QND}\right) \nonumber\\
&\quad - \left. \expect{\hat{\sigma}_{dc}} \right|_{QND} \left( \left.d\expect{\hat{\sigma}_{ba}} \right|_{QND}\right)
- \left.d\expect{\sigma_{ba}} \right|_{QND}\left.d\expect{\sigma_{dc}} \right|_{QND} \nonumber \\
&= -\kappa\expect{\Delta \sigma^{(1)}_{ba}  \Delta F_z }_s \expect{\Delta F_z \Delta \sigma_{dc}^{(2)} }_sdt,
\end{align}
where we have employed the Ito calculus $dW^2 = dt$. Using the approximations above, the collective spin dynamics can be efficiently calculated with $ d^2=9 $ equations for the one-body quantity, $ \expect{\hat{\sigma}_{ba}} $, and $ \frac{d^2(d^2+1)}{2}=45 $ equations for the two-body covariances, $ \expect{\Delta \sigma_{ba}^{(1)}\Delta\sigma_{dc}^{(2)} }_s $, in the symmetric subspace independent of the number of atoms.

With this formalism in hand, we can calculate the squeezing parameter, Eq.\eqref{eq:xi2Faraday}, as a function of time by finding time-dependent solutions for the one-body averages $\expect{\hat{f}_x}$ and  $\expect{\Delta f_z^2}$, and the two-body correlations $\expect{\Delta f_z^{(1)} \Delta f_z^{(1)}}$ and hence the collective quantities.  This is done in the qutrit basis according to Eq. (?) and presented in Appendix (?).

Apparently, the cooperativity is increasing on the order of $ 1/r\!_\perp^2 $ if the atoms are placed closer to the waveguide surface. Fig (?) shows the inversed peak squeezing parameter as a function of $ r\!_\perp $ for both nanofiber and SWG geometries. We cut off the calculation at $ ~20 $dB when the spin squeezed state is about to become a non-Gaussian state with $2500$ atoms.

To illustrate how the anisotropy of the orthogonal linear modes of the waveguides affect the spin squeezing effect, we plot the ratio of local $ V $- verse $ H $-mode intensity as a function of bulk index of refraction when the atoms are placed $ ~150 $nm away from the surface of a SWG with a side of $ 300 $nm, and compare it with the inversed peak squeezing parameter in Fig (?). 

\section{Conclusion and outlook}


\bibliography{refs/Archive}
\end{document}
