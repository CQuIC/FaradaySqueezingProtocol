\documentclass[preprint,aps,pra,onecolumn,superscriptaddress]{revtex4-1} %reprint
%\tightenlines

%\draft
\usepackage{etex}
\usepackage{amsmath}
\usepackage{bm}
\usepackage{bbm}
\usepackage{listings}
% % \textwidth 16cm \textheight 23.5cm
% \renewcommand{\baselinestretch}{1.2}
\usepackage{graphicx}
\usepackage{graphics}
\usepackage{epsfig}
\usepackage{color}
\usepackage[dvipsnames]{xcolor}
\usepackage{multirow}
\usepackage[colorlinks]{hyperref}
\usepackage{fancyhdr}
\usepackage{calc}
\usepackage{natbib} %[numbers]
\usepackage{bibentry}
\usepackage{bbm}

% todo list and commands
%\usepackage{todonotes}
%% to avoid the conflict with amths package % not working
%\makeatletter
%\providecommand\@dotsep{5}
%\makeatother
%\listoftodos\relax
%\usepackage{makeidx}
%\allowdisplaybreaks
%% for eps transfering to pdf.
%\usepackage[update,prepend]{epstopdf}
%\usepackage{ifpdf}
%
%\ifpdf
%   \usepackage{graphicx}
%   \usepackage{epstopdf}
%   \epstopdfsetup{suffix=}
%   \DeclareGraphicsRule{.eps}{pdf}{.pdf}{`epstopdf #1}
%   \pdfcompresslevel=9
%\else
%   \usepackage{graphicx}
%\fi
% subfig
%\usepackage{mwe}
%\usepackage{subfig}
% to fix a figure's position using [H] option of thec figure.
%\usepackage{float}
% to use \lesssim and other math symbols
%\usepackage{amssymb}


% self-defined short-cuts and commands
%\input{Mydef.tex}
\DeclareMathOperator{\tr}{tr}
\newcommand{\dt}[1]{\frac{{\mathrm d} {#1}}{{\mathrm d}t}}
\def\br{\mathbf{r}}
\def\bra#1{\langle{#1}\rvert}%{\mathinner{\langle{#1}\rvert}}
\def\ket#1{\lvert{#1}\rangle}%{\mathinner{\lvert{#1}\rangle}}
\def\Braket#1#2{\mathinner{\langle{#1}\! \mid\! {#2} \rangle}}
%========================================================================================
\newcommand{\erf}[1]{Eq.~(\ref{#1})}
\newcommand{\frf}[1]{Fig.~\ref{#1}}
\newcommand{\srf}[1]{Sec.~\ref{#1}}
\newcommand{\nn}{\nonumber}
\newcommand{\mbf}[1]{\mathbf{#1}}
%========================================================================================
% General quantum mechanics macros
%========================================================================================
\newcommand{\op}[2]{\ket{#1}\bra{#2}}
\newcommand{\expt}[1]{\langle{#1}\rangle}
\newcommand{\dg}{^\dagger}
\newcommand{\smallfrac}[2]{\mbox{$\frac{#1}{#2}$}}
\newcommand{\Tr}{\mbox{Tr}}
%========================================================================================
\newcommand{\expect}[1]{\big\langle #1 \big\rangle}
\newcommand{\eff}{\text{eff}}



% Redefine the tensor command.
%\renewcommand{\tensor}[1]{\boldsymbol{#1}}


%==== Ben's new macros ======
%\newcommand{\srf}[1]{Sec. \ref{#1}}
\newcommand{\half}{\smallfrac{1}{2}}

%==== subscripts ======
\newcommand{\oneD}{{\rm 1D}}
\newcommand{\vac}{{\rm vac}}
\newcommand{\cav}{{\rm cav}}
\newcommand{\inp}{{\rm in}}
\newcommand{\out}{{\rm out}}
\newcommand{\inter}{{\rm int}}
\newcommand{\scs}{{\rm SCS}}
\newcommand{\fwd}{+}
\newcommand{\bwd}{-}
\newcommand{\trans}{+}
\newcommand{\refl}{-}

 %==== operators/moments ======
\newcommand{\der}[1]{\frac{d {#1}}{dt}}
\newcommand{\unittens}{\tensor{\mathbf{I}}}
\newcommand{\poltens}{\hat{\tensor{\boldsymbol{\alpha}}}}
\newcommand{\varz}{\Delta J_3^2}
\newcommand{\jx}{\hat{J}_1}
\newcommand{\jz}{\hat{J}_3}
\newcommand{\shotnoise}{\Delta \mathcal{M}^2 |_{\rm SN}}
\newcommand{\projnoise}{\Delta \mathcal{M}^2_{\rm PN}}
\newcommand{\polcomp}{\hat{K}} % p,p' component of the tensor polarizability
\newcommand{\fo}{\hat{\mathbf{f}}}
\newcommand{\fx}{\hat{f}_x}
\newcommand{\fy}{\hat{f}_y}
\newcommand{\fz}{\hat{f}_z}
\newcommand{\Fx}{\hat{F}_x}
\newcommand{\Fy}{\hat{F}_y}
\newcommand{\Fz}{\hat{F}_z}
\newcommand{\rhoo}{\hat{\rho}}

%==== Microscopic moments for the qubit/qutrit subspace ====
\newcommand{\sigmauu}{\hat{\sigma}_{\uparrow\uparrow}}
\newcommand{\sigmaud}{\hat{\sigma}_{\uparrow\downarrow}}
\newcommand{\sigmaut}{\hat{\sigma}_{\uparrow \mathrm{T}}}
\newcommand{\sigmadu}{\hat{\sigma}_{\downarrow\uparrow}}
\newcommand{\sigmadd}{\hat{\sigma}_{\downarrow\downarrow}}
\newcommand{\sigmadt}{\hat{\sigma}_{\downarrow \mathrm{T}}}
\newcommand{\sigmatu}{\hat{\sigma}_{\mathrm{T}\uparrow}}
\newcommand{\sigmatd}{\hat{\sigma}_{\mathrm{T}\downarrow}}
\newcommand{\sigmatt}{\hat{\sigma}_{\mathrm{T}\mathrm{T}}}
\newcommand{\sigmaab}{\hat{\sigma}_{ab}}
\newcommand{\sigmaba}{\hat{\sigma}_{ba}}
\newcommand{\sigmadc}{\hat{\sigma}_{dc}}
\newcommand{\sigmacd}{\hat{\sigma}_{cd}}

\newcommand{\Dsigmauu}{\Delta\sigma_{\uparrow\uparrow}}
\newcommand{\Dsigmaud}{\Delta\sigma_{\uparrow\downarrow}}
\newcommand{\Dsigmaut}{\Delta\sigma_{\uparrow \mathrm{T}}}
\newcommand{\Dsigmadu}{\Delta\sigma_{\downarrow\uparrow}}
\newcommand{\Dsigmadd}{\Delta\sigma_{\downarrow\downarrow}}
\newcommand{\Dsigmadt}{\Delta\sigma_{\downarrow \mathrm{T}}}
\newcommand{\Dsigmatu}{\Delta\sigma_{\mathrm{T}\uparrow}}
\newcommand{\Dsigmatd}{\Delta\sigma_{\mathrm{T}\downarrow}}
\newcommand{\Dsigmatt}{\Delta\sigma_{\mathrm{T}\mathrm{T}}}
\newcommand{\Dsigmaab}{\Delta\sigma_{ab}}
\newcommand{\Dsigmaba}{\Delta\sigma_{ba}}
\newcommand{\Dsigmadc}{\Delta\sigma_{dc}}
\newcommand{\Dsigmacd}{\Delta\sigma_{cd}}

%==== physical parameters ======
\newcommand{\Eamp}{\mathcal{F}_0^{(+)}}
\newcommand{\charpol}{\alpha_0(\Delta_{f\!f'})}
\newcommand{\charpolq}{\alpha_0(\Delta_{f\!f'}^q)}
\newcommand{\qaxis}{\mathbf{e}_{\tilde{z}}}
\newcommand{\qangle}{\varphi}
\newcommand{\magic}[1]{\tilde{\omega}_{#1}}
\newcommand{\chiN}{\chi_{N}}
\newcommand{\NA}{N_C}
\newcommand{\chieff}{\chi_{\raisebox{-.1pt}{\tiny $J_3$}}}

%==== scattering and optical pumping rates ====%
\newcommand{\gammauu}{\gamma_{\uparrow \rightarrow \uparrow}}
\newcommand{\gammadd}{\gamma_{\downarrow \rightarrow \downarrow}}
\newcommand{\gammaud}{\gamma_{\uparrow \rightarrow \downarrow}}
\newcommand{\gammadu}{\gamma_{\downarrow \rightarrow \uparrow}}
\newcommand{\gammau}{\gamma_{\uparrow}}
\newcommand{\gammad}{\gamma_{\downarrow}}

%==== effective areas ======
\newcommand{\Ain}{A_{\rm in}}
\newcommand{\Abir}{A_N}
\newcommand{\AF}{A_F} % for the Faraday protocol.
\newcommand{\Ai}{A_0} % for the input light.
\newcommand{\Aint}{A_{\rm int}} % for the interaction area.

%==== eigenfunctions ======
\newcommand{\eigenf}{\mbf{f}_\eta}
\newcommand{\eigenfp}{\mbf{f}_{\eta'}}
\newcommand{\eigeng}{\mbf{g}_\eta}
\newcommand{\eigengp}{\mbf{g}_{\eta'}}

%==== field operators ======
\newcommand{\awg}{\hat{a}_{b,p}(\omega)}
\newcommand{\awr}{\hat{a}_{m,p}(\omega,\beta)}

%==== Famous names =====
\usepackage{xspace}
\newcommand{\Poincare}{Poincar\'e\xspace}

%==== colors for editing ======
\newcommand{\change}[1]{{\color{RoyalBlue} #1}}
\newcommand{\comment}[1]{{\color{Maroon} #1}}
\newcommand{\error}[1]{{\color{red} #1}}

% =============================================================================


\begin{document}
\title{Effective optical depth per atom of spin squeezing dynamics using nanophotonic waveguides}
\author{Xiaodong Qi}
\affiliation{Center for Quantum Information and Control, University of New Mexico, Albuquerque, New Mexico 87131, USA}
\author{Ezad Shojaee}
\affiliation{Center for Quantum Information and Control, University of New Mexico, Albuquerque, New Mexico 87131, USA}
\author{Poul S. Jessen}
\affiliation{Center for Quantum Information and Control, University of Arizona, Tucson, Arizona 87521, USA}
\author{Ivan H. Deutsch}
\affiliation{Center for Quantum Information and Control, University of New Mexico, Albuquerque, New Mexico 87131, USA}
\author{Yuan-Yu Jau}
\affiliation{Sandia National Laboratories, Albuquerque, New Mexico 87185, USA}
\date{\today}
\pacs{42.50.Lc, 03.67.Bg, 42.50.Dv, 42.81.Gs}

%================================================================%
\begin{abstract}
We study the strong coupling between photons and atoms that can be achieved in nanophotonic geometries in the dispersive regime to implement efficient quantum interface with neutral atoms.
First, we extend our previous work on spin squeezing and quantum nondemolition (QND) measurement with the birefringence protocol using a nanofiber to a more commonly considered protocol using the Faraday interaction.
We established a general theory to calculate the spin squeezing dynamics and show that $7dB$ of spin squeezing may be achievable with $2500$ atoms trapped $1.8$ times of the fiber radius from the fiber axis, in comparison with $~5dB$ of spin squeezing using the birefringence protocol.
The Faraday interaction protocol does not require a sophisticated search for the magic frequencies and can avoid the difficulty of preparing the atoms on the harder-to-prepare clock states as we did for the birefringence protocol,
and hence is more robust and easier to implement.
Meanwhile, compared to the spin squeezing protocol using Faraday interaction in free space, the nanofiber platform enables us to increase
the photon-atom coupling strength dramatically while to reduce the decoherence due to photon scattering process simultaneously,
and hence can achieve a high peak spin squeezing efficiently.
To achieve an even stronger spin squeezing effect towards a non-Gaussian collective spin state, for example, we propose a generalized optical depth per atom concept applicable to general nanophotonic waveguides and explicitly separate the photon-atom coupling strength from the decoherence characteristic parameter, which can be used to guide the design of novel nanophotonic quantum interfaces and protocols towards efficient quantum control and measurement of atomic states as well as preparing non-Gaussian atomic ensemble states.
We also study the decoherence mechanism for atoms with arbitrary total angular momentum quantum number $f$ with the nanofiber interface, and confirm $f=1$ is the optimal case for the Faraday protocol we study.
Finally, we give an example of analysing a nanophotonic waveguide interface with square cross-section to implement efficiently strong photon-atom coupling and discuss the challenges of using waveguides without cylindrical symmetry beyond optical nanofibers towards efficient photon-atom coupling.
\end{abstract}

\maketitle

%===================INTRODUCTION=====================%
\section{Introduction}

Strong coupling between atoms and photons via nanophotonic structures and such.

The difficulty and challenges.

Our protocol and the structure of this paper.


%========================== Theory ===================================%
\section{Theory} \label{Sec::Theory}

\subsection{Method to calculate spin squeezing dynamics in a QND measurement process}
To study the spin dynamics of the QND measurement process, it is usually valid to calculate the density if the dimension is small. 
For a large ensemble of atoms, however, it requires a big dimension to store the density matrix and it becomes extremely difficult to calculate the evolution of the density operator.
Below, we outline our general method to study the spin dynamics and the spin squeezing evolution without calculating the density operator.

A \textit{squeezed coherent spin state}\index{squeezed spin state!squeezed coherent spin state} is a spin state that the uncertainty principle of two quadratures of collective spin operators saturates and one quadrature is smaller than the other. 
For example, a squeezed coherent spin state may satisfy $ \Delta F_1\Delta F_2=\frac{1}{2} $ (we set $ \hbar=1 $) and $ \Delta F_1<\Delta F_2 $, where $ \hat{F}_1 $ and $ \hat{F}_2 $ are two collective spin operators. 
In general, a \textit{squeezed spin state}\index{squeezed spin state} always satisfies the uncertainty principle relationship $ \Delta F_1\Delta F_2\ge\frac{1}{2} $.
We call the operator that yields the smaller quadrature as the atomic angular momentum operator $ \hat{F}_\perp $ and 
recall the spin squeezing parameter defined by Wineland {\emph{et al.}}~\cite{Wineland1992},
\begin{align}
\zeta^2 &\equiv \frac{\expect{\hat{F}_\parallel(t=0)}^2}{\Delta F_\perp^2(t=0)} \frac{\Delta F_\perp^2}{\expect{\hat{F}_\parallel}^2},
\end{align}
which only requires to calculate the expectation value of the collective atomic angular momentum operator in parallel with the total atomic angular momentum vector in the generalized Bloch sphere, $ \expect{\hat{F}_\parallel} $, as well as the variance of the collective atomic angular momentum operator perpendicular to the total atomic angular momentum operator, $ \Delta F_\perp^2 $. 
Assume the atom number is $ N_A $, these two collective quantities can be decomposed into microscopic quantities by 
\begin{align}
\expect{\Delta F_\perp^2} &= N_A \expect{\Delta f_\perp^2}+\frac{N_A(N_A-1)}{2}\left. \expect{\Delta f_\perp^{(i)}\Delta f_\perp^{(j)}}_s\right|_{i\neq j}\label{eq:DeltaFz2}\\
\expect{\hat{F}_\parallel } &= \sum_i^{N_A} \expect{\hat{f}_\parallel ^{(i)}}=N_A \expect{\hat{f}_\parallel},\label{eq:expectFx}
\end{align}
where the first term of Eq.~\eqref{eq:DeltaFz2} and Eq.~\eqref{eq:expectFx} are solely determined by a symmetric sum over $N_A$ identical spin-$f$ single-body operators, $ \hat{f}_\perp=\hat{f}_\perp^{(i)} $ and $ \hat{f}_\parallel=\hat{f}_\parallel^{(i)} $ with atom labels $ i=1,\cdots,N_A $; the second term of Eq.~\eqref{eq:DeltaFz2} is determined by symmetric two-body covariance terms, $ \left.\expect{\Delta f_\perp^{(i)}\Delta f_\perp^{(j)}}_s\right|_{i\neq j}=\expect{\Delta f_\perp^{(1)}\Delta f_\perp^{(2)}}_s\equiv \expect{\hat{f}_\perp^{(1)}\hat{f}_\perp^{(2)}}_s-\left( \expect{\hat{f}_\perp^{(1)}} \expect{\hat{f}_\perp^{(1)}}\right)_s $, which correspond to the pairwise entanglement among atoms and eventually yield spin squeezing~\cite{Wang2003Spin}.
Above, we have assumed there is a pairwise exchange symmetry among atoms so that we only care about the symmetrized quantities like $ \expect{\Delta f_\perp^{(1)}\Delta f_\perp^{(2)}}_s=\left(\expect{\Delta f_\perp^{(1)}\Delta f_\perp^{(2)}} + \expect{\Delta f_\perp^{(2)}\Delta f_\perp^{(1)}} \right)/2 $. 
Note that the collective state of atoms can be treated as pairwise-symmetric if the detuning is far off resonance or the numbers of atoms is not that large so that the photon scattering among atoms~\cite{Asenjo-Garcia2017Atom,Asenjo-Garcia2017Exponential} can be ignored compared to the measurement backaction which generates spin squeezing as we will discuss later.
In this paper, we will work in the dispersive regime with a few thousands of atoms and ignore the atom-atom interaction caused by photon scattering, and hence the collective atomic system satisfy the exchange symmetry. 

In the following discussions, we will define the spin squeezing operator $ \hat{F}_\perp=\hat{F}_z=\sum_i\hat{f}_z^{(i)} $. 
We define the fiducial state of an atom as the \textit{up state}, or $ \ket{\uparrow} $. 
Applying $ \hat{f}_z $ on the fiducial state will yield the coupled state or $ \ket{\downarrow} $.
That is, $ \hat{f}_z^{(i)}\ket{\uparrow}^{(i)}\rightarrow \ket{\downarrow}^{(i)} $ for each atom.

To study the spin dynamics, we formally define a stochastic master equation of the atomic ensemble by
\begin{align}\label{eq:totaldrhodt}
\mathrm{d}\hat{\rho}=\left.\mathrm{d}\hat{\rho}\right|_{op} + \left.\mathrm{d}\hat{\rho}\right|_{QND}.
\end{align}
It includes two collective spin dynamic processes. 
The first process is the optical pumping dynamics on each individual atom $i$ positioned at $\br'$ which yields the $\mathrm{d}\hat{\rho}|_{op}=\sum_i^{N_A} \left.\mathrm{d}\hat{\rho}^{(i)}\right|_{op} $ term given by
\begin{align}
&\quad\left.\mathrm{d}\hat{\rho}^{(i)}\right|_{op} =\gamma_s\mathcal{D}^{(i)}\mathrm{d}t\\
&= -\frac{i\gamma_s}{\hbar} \left\{\hat{h}_{\rm loss},\hat{\rho}^{i} \right\}\mathrm{d}t + \gamma_s\sum_q \hat{W}_q(\br')\hat{\rho}^{i}\hat{W}_q(\br')\mathrm{d}t,
\end{align}
where the characteristic photon scattering rate $ \gamma_s\equiv \frac{\Gamma_0\Omega^2}{4\Delta_F}=\frac{\sigma_0}{A_{in}}\frac{\Gamma_0^2}{4\Delta_F^2}\dot{N}_L $ with the effective input mode area $ A_{in}=1/n_g|u_{\mathrm{in}}(\br'\!_\perp)|^2 $ and the effective detuning $ \Delta_F $ defined by $ \frac{1}{\Delta_F}=\sum_{f'}\frac{C_{f'ff'}^{(1)}}{\Delta_{ff'}} $ and $ \Delta_{ff'}=\omega-\omega_{ff'} $, where $ \omega_{ff'} $ is the resonance angular frequency between the ground hyperfine structure level $ f $ and the excited hyperfine structure level $ f' $, and $ C_{j'ff'}^{(K)} $ are the coefficients for irreducible rank-$K$ components defined in \cite{Deutsch2010a}.
$\gamma_s$ characterizes the rate of decoherence dynamics and is proportional to the local photon flux of the probe light, $ \dot{N}_L $.

The second term on the right-hand side of Eq.\eqref{eq:totaldrhodt} gives rise to the collective spin dynamics due to QND measurement,
\begin{align}
\left.\mathrm{d}\hat{\rho}\right|_{QND} &= \sqrt{\frac{\kappa}{4}}\mathcal{H}\left[\hat{\rho} \right]\mathrm{d}W + \frac{\kappa}{4}\mathcal{L}\left[ \hat{\rho}\right]\mathrm{d}t.
\end{align}
Above, we have defined the measurement strength $\kappa \equiv |\chi|^2\dot{N}_L\equiv \frac{\sigma_0A_{in}}{A_{int}^2}\gamma_s $ determining the rate of the spin squeezing in absence of decoherent processes, where $\dot{N}_L$ is the photon number flux, $\chi$ is the light-atom coupling strength and $A_{int}$ is the effective interaction mode area which can be specified for a particular QND measurement protocol. We have also assumed the measurement backation is a stochastic Weiner process where $\mathrm{d}W$ is the increment satisfying $\mathrm{d}W^2 = \mathrm{d}t$. The conditional dynamics responding to the measurement evolve under the superoperator
\begin{align}
\mathcal{H}\left[ \hat{\rho}\right] &= \hat{F}_\perp\hat{\rho} + \hat{\rho}\hat{F}_\perp -2\expect{\hat{F}_\perp}\hat{\rho}
\end{align}
and the collective Lindblad map due to the direct photon scattering of the guided modes from the atoms
\begin{align}
\mathcal{L}\left[ \hat{\rho} \right] &= \hat{F}_\perp\hat{\rho}\hat{F}_\perp-\frac{1}{2}\left(\hat{\rho}\hat{F}_\perp^2+\hat{F}_\perp^2\hat{\rho} \right)=\frac{1}{2}\left[\hat{F}_\perp,\left[\hat{\rho},\hat{F}_\perp \right] \right].
\end{align}

As shown in the equations above, the spin squeezing dynamics is a competition between the coherent squeezing process and all decoherent processes which are characterized by $\kappa$ and $\gamma_s$, respectively. 
If we define an effective cooperativity or optical depth (OD) per atom quantity for the spin squeezing dynamics by
\begin{align}
\frac{\mathrm{OD}_{\rm eff}}{N_A} \equiv \frac{\kappa}{\gamma_s}=\frac{\sigma_0A_{in}}{A_F^2},
\end{align}
the peaking spin squeezing dynamics can then be characterized by $\frac{\mathrm{OD}_{\rm eff}}{N_A}$, and the geometry of the spin squeezing protocol can then be roughly designed with the goal to maximize $\frac{\mathrm{OD}_{\rm eff}}{N_A}$ by minimizing $A_{in}$ and maximizing $A_{int}$.  

We can bring in the spin coherence operator $\hat{\sigma}_{ba}=\ket{b}\bra{a}$ to represent the matrix element of any atomic angular momentum operator $ \hat{f}_m $ ($ m=x,y,z $) of a single atom, and hence the spin squeezing dynamics can be characterized by the expectation value of single-body spin coherence operators $\expect{\hat{\sigma}_{ba}}$ and the symmetric two-body covariances $\expect{\Delta \sigma_{ba}^{(1)}\Delta\sigma_{dc}^{(2)} }_s$. 
%the symmetric three-body correlations $\expect{\Delta \sigma^{(1)}_{b_1a_1}\Delta \sigma^{(2)}_{b_2a_2}\Delta \sigma^{(3)}_{b_3a_3} }_s$ and so on. 
%As higher-order correlations becomes negligible, one can cut off the correlation terms at a certain order.
If one can truncate the spin dynamics up to the two-body correlations, we only need the following two sets of stochastic differential equations:
\begin{subequations}
\begin{align}
d\expect{\hat{\sigma}_{ba}} &=\left. d{\expect{\hat{\sigma}_{ba}}}\right|_{op} + \left. d{\expect{\hat{\sigma}_{ba}}}\right|_{\mathcal{H}}+\left. d{\expect{\hat{\sigma}_{ba}}}\right|_{\mathcal{L}} \\
d\expect{\Delta \sigma_{ba}^{(1)}\Delta \sigma_{dc}^{(2)}}_s &= \left. d{\expect{\Delta \sigma_{ba}^{(1)}\Delta \sigma_{dc}^{(2)}}_s}\right|_{op} + \left. d{\expect{\Delta \sigma_{ba}^{(1)}\Delta \sigma_{dc}^{(2)}}_s}\right|_{\mathcal{H}} + \left. d{\expect{\Delta \sigma_{ba}^{(1)}\Delta \sigma_{dc}^{(2)}}_s}\right|_{\mathcal{L}}.
\end{align}
\end{subequations}

In details, the optical dynamics part can be given by
\begin{align}
\left. \dt{\expect{\hat{\sigma}_{ba}}}\right|_{op} &= \gamma_s\expect{\mathcal{D}^\dagger \left[ \hat{\sigma}_{ba}\right]}\\
&= \gamma_s\sum_{d,c}\tr\left(\mathcal{D}^\dagger \left[ \hat{\sigma}_{ba}\right]\hat{\sigma}_{dc} \right)\expect{\hat{\sigma}_{dc} }\\
\left. \dt{\expect{\Delta \sigma_{ba}^{(1)}\Delta \sigma_{dc}^{(2)}}_s}\right|_{op} &=\gamma_s\expect{\Delta\mathcal{D}^\dagger[\hat{\sigma}_{ba}^{(1)}]\Delta\sigma_{dc}^{(2)} }_s + \gamma_s\expect{\Delta\sigma_{ba}^{(1)}\Delta\mathcal{D}^\dagger[\hat{\sigma}_{dc}^{(2)}] }_s\\
&= \gamma_s\sum_{m,n}\tr\left(\mathcal{D}^\dagger[\hat{\sigma}_{ba}]\hat{\sigma}_{mn} \right)\expect{\Delta \sigma_{mn}^{(1)}\Delta \sigma_{dc}^{(2)} }_s + \gamma_s\sum_{m,n}\tr\left(\mathcal{D}^\dagger[\hat{\sigma}_{dc}]\hat{\sigma}_{mn} \right) \expect{\Delta \sigma_{ba}^{(1)}\Delta \sigma_{mn}^{(2)} }_s.
\end{align} 
Similarly, we will need the one- and two-body correlations due to the $ \mathcal{H} $ and $ \mathcal{L} $ superoperators given by the following.
\begin{subequations}
\begin{align}
\left.d\expect{\hat{\sigma}_{ba}}\right|_\mathcal{H} &=\sqrt{\frac{\kappa}{4}}\expect{\mathcal{H}^\dagger\left[\hat{\sigma}_{ba} \right]}dW \\
\left.d\expect{\hat{\sigma}_{ba}}\right|_\mathcal{L} &= \frac{\kappa}{4}\expect{\mathcal{L}^\dagger\left[\hat{\sigma}_{ba} \right]}dt
\end{align}
\end{subequations}
In principle, the two-body covariance terms can be coupled to high-order many-body terms. 
In our case, we assume the state of the ensemble can be well captured in the symmetric Gaussian state limit, and hence the two-body covariance equations due to the collective measurement can be given by
\begin{subequations}
\begin{align}
\left.d\expect{\Delta \sigma_{ba}^{(1)} \Delta \sigma_{dc}^{(2)}} \right|_\mathcal{H} &= -\kappa\expect{\Delta\sigma_{ba}^{(1)}\Delta F_\perp }_s \expect{\Delta F_\perp \Delta \sigma_{dc}^{(2)} }dt \\
\left.d\expect{\Delta \sigma_{ba}^{(1)} \Delta \sigma_{dc}^{(2)}}\right|_\mathcal{L} &= 0.
\end{align}
\end{subequations}

\subsection{A spin squeezing protocol using Faraday interaction with spin coherent states}
\comment{To be done: Double check all equations in this section.}

Following the process demonstrated in our previous work~\cite{Qi2016}, the light-atom interaction Hamiltonian with one atom can be written as
\begin{align}
\hat{h}_\eff &= -\hat{\mathbf{E}}^{(-)}(\br')\cdot\hat{\tensor{\mathbf{\alpha}}}\cdot\hat{\mathbf{E}}^{(+)}(\br')\nn\\
&= -\frac{2\pi\hbar\omega}{v_g}\left[\mathbf{u}_H^*\cdot\hat{\tensor{\mathbf{\alpha}}}\cdot \mathbf{u}_H\hat{a}_H^\dagger\hat{a}_H\right.
+ \mathbf{u}_H^*\cdot\hat{\tensor{\mathbf{\alpha}}}\cdot \mathbf{u}_V\hat{a}_H^\dagger\hat{a}_V\nn\\
&\quad\quad + \mathbf{u}_V^*\cdot\hat{\tensor{\mathbf{\alpha}}}\cdot \mathbf{u}_H\hat{a}_V^\dagger\hat{a}_H 
\left. + \mathbf{u}_V^*\cdot\hat{\tensor{\mathbf{\alpha}}}\cdot \mathbf{u}_V\hat{a}_V^\dagger\hat{a}_V\right]\\
&= \hbar\left[(\hat{\chi}_{HH}+\hat{\chi}_{VV})\hat{S}_0 + (\hat{\chi}_{HH}-\hat{\chi}_{VV})\hat{S}_1 + (\hat{\chi}_{HV}+\hat{\chi}_{VH})\hat{S}_2 + i(\hat{\chi}_{HV}-\hat{\chi}_{VH})\hat{S}_3 \right]\\
%\hbar \left[\left(\chi_{RR\uparrow} + \chi_{RR\downarrow} +\chi_{LL\uparrow}+\chi_{LL\downarrow} \right)\hat{F}_0\hat{S}_0 \right.\nonumber\\
%&\quad+\left(\chi_{RR\uparrow} + \chi_{RR\downarrow} -\chi_{LL\uparrow}-\chi_{LL\downarrow} \right)\hat{F}_0\hat{S}_3\nonumber\\
%&\quad+\left(\chi_{RR\uparrow} + \chi_{LL\uparrow} -\chi_{RR\downarrow}-\chi_{LL\downarrow} \right)\hat{F}_3\hat{S}_0\nonumber\\
%&\quad+\left(\chi_{RR\uparrow} - \chi_{RR\downarrow} +\chi_{LL\downarrow}-\chi_{LL\uparrow} \right)\hat{F}_3\hat{S}_3\nonumber\\
%&\quad+i\left(\chi_{LR\uparrow} - \chi_{RL\uparrow} +\chi_{RL\downarrow}-\chi_{RL\downarrow} \right)\hat{F}_0\hat{S}_1\nonumber\\
%&\quad+\left(\chi_{RL\uparrow} + \chi_{LR\uparrow} +\chi_{RL\downarrow}+\chi_{LR\downarrow} \right)\hat{F}_0\hat{S}_2\nonumber\\
%&\quad+i\left(\chi_{LR\uparrow} - \chi_{RL\uparrow} +\chi_{RL\downarrow}-\chi_{LR\downarrow} \right)\hat{F}_3\hat{S}_1\nonumber\\
%&\quad+\left.\left(\chi_{LR\uparrow} + \chi_{RL\uparrow} -\chi_{LR\downarrow}-\chi_{RL\downarrow} \right)\hat{F}_3\hat{S}_2 \right]\\
&=\hbar\sum_{i=0}^3 \hat{\chi}_{i}\hat{S}_i\\
&=\hbar\sum_{i,j=0} \chi_{ij}\hat{f}_i\hat{S}_j,
\end{align}
where $ \hat{S}_i $ are the Stokes vector operators of the light indicating its polarization, and the mode-atom coupling operator
\begin{align}
\hat{\chi}_{pp'} 
&=-\frac{2\pi \omega}{v_g}\mathbf{u}_{p}^*(r'\!_\perp,\phi')\cdot \hat{\tensor{\alpha}}\cdot \mathbf{u}_{p'}(r'\!_\perp,\phi')\\
&= \sum_{f'} \frac{n_g\sigma_0}{4}\frac{\Gamma_{f'}}{\Delta_{ff'}+i\Gamma_{f'}/2}\cdot \left\{ C_{j'ff'}^{(0)}\mathbf{u}_p^*(r'\!_\perp)\cdot \mathbf{u}_{p'}(r'\!_\perp)\hat{\mathbbm{1}}\right.\nn\\
&\quad\quad +iC_{j'ff'}^{(1)}\left(\mathbf{u}_p^*(r'\!_\perp)\times\mathbf{u}_{p'}(r'\!_\perp) \right)\cdot \hat{\mathbf{f}} \nonumber\\
&\quad\quad\left. + C_{j'ff'}^{(2)}\sum_{i,j}\left[u^*_{p,i}u_{p',j}(\frac{\hat{f}_i\hat{f}_j+\hat{f}_j\hat{f}_i}{2}-\frac{\delta_{ij}}{3}\hat{\mathbf{f}}\cdot\hat{\mathbf{f}}) \right]\right\}
%&\left.+C_{jj'ff'}^{(2)}\left[\mathbf{u}_p^*(r'\!_\perp)\cdot \mathbf{u}_{p'}(r'\!_\perp)\left(\frac{f(f+1)}{6}-\frac{m^2}{2} \right)+\mathbf{u}_p^*(r'\!_\perp)\cdot (\hat{e}^*_{\tilde{z}}\hat{e}_{\tilde{z}})\cdot \mathbf{u}_{p'}(r'\!_\perp)\left(\frac{3m^2}{2}-\frac{f(f+1)}{2} \right) \right] \right\}
\label{eq:chippp}
\end{align}
with the horizontally(H)- and vertically(V)-linearly polarized guided modes, $ \mathbf{u}_p(r'\!_\perp) $, at the atom position $ \br'=(r'\!_\perp,\phi',z') $. 
$ \chi_{ij}=\tr[\hat{f}_i\hat{\chi}_j]/(2f+1) $ is the coupling strength between spin operator $ \hat{f}_i $ and Stokes operator $ \hat{S}_j $. 
For example, $ \chi_{33} $ is the coupling strength between $ \hat{f}_z $ and $ \hat{S}_3 $.
The fundamental guided modes of an optical nanofiber has been defined in the appendix of our previous paper~\cite{Qi2016}. 
In general, for a cylindrical waveguide, the H- and V-modes are the guided modes adiabatically transferred from a corresponding linearly polarized input light from one end of the waveguide, where H- and V-directions are orthogonal to each other in the transverse plane.
The coupling operator or Eq.\eqref{eq:chippp} includes three terms corresponding to scalar, vector and tensor interactions between atoms and the probe light which are proportional to $ C_{j'ff'}^{(K)} $ with $ K=0,\,1,\,2 $, respectively.

Faraday interaction is the interaction when the helicity of the input and output of the light signal is preserved. 
In the context of QND measurement, a Faraday interaction based protocol is the protocol when the phase change of the output light on the equator of the \Poincare sphere with a linear polarization input is used to calibrate the spin state after the atom-light interaction.
This phase change of light polarization corresponds to a rotation about $ \mathbf{S}_3 $ axis on the \Poincare sphere. 
From Eq.\eqref{eq:chippp}, the coupling atomic term is proportional to 
\begin{align}
\hat{\chi}_{i3} &= \hat{\chi}_{HV}-\hat{\chi}_{VH}=\hat{\chi}_{RR}-\hat{\chi}_{LL},
\end{align}
where the last step is derived in appendix~\ref{Appendix:LRbases}. These relationships indicate the Faraday effect is generated by the intensity or photon number difference between the right- and left-polarized modes or by the phase difference between two linearly polarized modes due to the polarizability of the atoms.
One can prove that the strength of the Faraday interaction is dominated by the vector interaction term in Eq.\eqref{eq:chippp}.
This implies that, to maximize the Faraday interaction, it is optimal to choose a quantization axis along the direction of $ \mathbf{v}_F=\mathbf{u}_H^*(\br'\!_\perp)\!\times\!\mathbf{u}_{V}(\br'\!_\perp)\!-\!\mathbf{u}_V^*(\br'\!_\perp)\!\times\!\mathbf{u}_{H}(\br'\!_\perp)$ given the atoms are placed at $ \br'\!_\perp $ position in the transverse plane of the waveguide. 
In reality, this product of modes may be elliptical with at least one direction component is imaginary while others are real.
If this happens, since the quantization axis is the direction in which a magnetic field is pointing to in 3D real space, the optimal choice of quantization axis should be the direction corresponding to the largest component of the $ \mathbf{v}_F $ vector. 
For a cylindrical waveguide, this doesn't seem to happen.
Take the example of a nanofiber, at an arbitrary position $ \br'\!_\perp=(r'\!_\perp,\phi') $ of atoms in the transverse plane,
\begin{align}
\mathbf{u}_H^*(r'\!_\perp,\phi')\times \mathbf{u}_V(r'\!_\perp,\phi') &= 2u_{r\!_\perp} u_\phi\mathbf{e}_z - 2iu_zu_{r\!_\perp}\sin2\phi \mathbf{e}_\phi + 2iu_\phi u_z\cos2\phi \mathbf{e}_{r\!_\perp} \\
\mathbf{u}_V^*(r'\!_\perp,\phi')\times \mathbf{u}_H(r'\!_\perp,\phi') &= -2u_{r\!_\perp} u_\phi\mathbf{e}_z - 2iu_zu_{r\!_\perp}\sin2\phi \mathbf{e}_\phi + 2iu_\phi u_z\cos2\phi \mathbf{e}_{r\!_\perp},
\end{align}
and therefore,
\begin{align}\label{eq:Faradayaxis}
\mathbf{v}_F=\mathbf{u}_H^*(\br'\!_\perp)\!\times\!\mathbf{u}_{V}(\br'\!_\perp)\!-\!\mathbf{u}_V^*(\br'\!_\perp)\!\times\!\mathbf{u}_{H}(\br'\!_\perp) = 4u_{r\!_\perp} u_\phi\mathbf{e}_z,
\end{align}
where $ u_{r\!_\perp}=u_{r\!_\perp}(r'\!_\perp) $, $ u_\phi=u_\phi(r'\!_\perp) $ and $ u_z=u_z(r'\!_\perp) $ are the right-circularly polarized mode components independent of longitudinal and azimuthal positions defined in the appendix A of Ref~\cite{Qi2016} or the corresponding appendix in this dissertation.
Based on this result, choosing $ z $-direction as the quantization axis is optimal for the QND measurement and spin squeezing protocol for atoms trapped near an optical nanofiber.

This conclusion can be generalized to cylindrical waveguides at large, which have a smooth or slow change of index of refraction along the light propagation direction in the wavelength scale while the cross-section of the waveguides can be arbitrary.
From the perspective of transformation optics~\cite{Leonhardt2006Optical,Kundtz2011Electromagnetic}, we can consider a set of $H$- and $V$-modes for the new waveguide are generated by adiabatically transforming the orthogonal set of modes from a cylindrical nanofiber to the target cross-section shape of the waveguide.
Since the transformation is approximately limited to the $xy$-plane of the coordinate system transformation determined by a Jacobian matrix, Eq.\eqref{eq:Faradayaxis} will preserve the form in the new waveguide coordinate system where only $ z $-component is non-zero and should be chosen as the optimal choice of the quantization axis.

With the quantization axis chosen along $ \mathbf{e}_{\tilde{z}}=\mathbf{e}_z $, one can define the effective QND measurement Hamiltonian for an atomic ensemble by
\begin{align}
\hat{H}_F &= \hbar \chi_{33}\hat{F}_z \hat{S}_3,
\end{align}
where $ \chi_{33} $ is the measurement strength characterizing the entanglement between the collective spin state of $ \hat{F}_z $ and the probe's polarization state of $ \hat{S}_3 $.
Now that $ \hat{F}_\perp=\hat{F}_z $ is the squeezing operator. 
Let us consider an ensemble of $ ^{133} $Cs atoms initially prepared as a spin coherent state (SCS) with every atom in the stretch state of the $ 6S_{1/2}$ $f=4 $ ground manifold in the $ x $-basis, where the quantization $ x $-direction is along the diagonal or $ \phi=\pi/4 $ direction in the $ H $-$ V $ Cartesian coordinate system sitting on the fiber axis.
One can show that a stretch state can generate the maximum coupling between atoms and light when all atoms are prepared in a SCS.
We call this initial state of one atom as the fiducial state or $ \ket{\phi_0}=\ket{\uparrow}_x = \ket{6S_{1/2},f=4,m_x=4} $ and the collective SCS can be written as $ \ket{\Psi_0}=\ket{\uparrow}_x^{\otimes N_A} $.
We define the coupled state by applying the individual squeezing operator $ \hat{f}_z$ on the fiducial state $\ket{\uparrow}_x $.
In our case, the coupled state can be defined as $ \ket{\downarrow}_x=\ket{6S_{1/2},f=4,m_x=3} $.
Similarly, to include the transfer of coherence in the squeezing process, we define the transfer state as $ \ket{T}_x=\ket{6S_{1/2},f=4,m_x=2} $ resulted from applying $ \hat{f}_z$ on the coupled state. 
For simplicity, we remove all the subscript $ x $ of quantum states and assume we always work in the $ x $-basis if no explicit notations for the Faraday interaction protocol. 

We consider the far-detuning regime, where we may be able to set the decay rates of the excited levels $ \Gamma_{f'}= \Gamma_0$ as constant for all $ f' $ in the same fine structure manifold, where we consider $ \Gamma_0 $ as the averaged modified decay rates from the excite fine structure level to the ground level when atoms are in a completely mixed state for simplicity.
We also ignore the tensor coupling strength related to $ C_{jj'ff'}^{(2)} $ terms in Eq.\eqref{eq:chippp} as the tensor interaction strength ($ \sim 1/\Delta^2 $) is relatively small compared to the vector interaction strength ($ \sim 1/\Delta $)~\cite{Deutsch2010a}. 
For a nanofiber geometry, the Faraday interaction coupling strength is independent of the azimuthal position of the atoms and can be simplified as
\begin{align}
\chi_{33} &= -\sum_{f'}n_g\sigma_0\frac{\Gamma_0}{\Delta_{ff'}+i\Gamma_0/2}C_{jj'ff'}^{(1)}u_{r\!_\perp}(r'\!_\perp)u_\phi(r'\!_\perp)\\
&=\frac{\sigma_0}{A_F}\frac{\Gamma_0}{\Delta_F},
\end{align}
where the effective Faraday interaction mode area $ A_F=1/2n_g|u_{r\!_\perp}(r'\!_\perp)u_\phi(r\!_\perp)| $, and the effective detuning $ \Delta_F=\sum_{f'}\frac{-C_{j'ff'}^{(1)}}{\Delta_{ff'}} $.
The measurement strength is now defined as
\begin{align}
\kappa\equiv|\chi_{33}|^2\dot{N}_L=\frac{\sigma_0A_{in}}{A_F^2}\gamma_s,
\end{align}
where the characteristic photon scattering rate $ \gamma_s\equiv \frac{\Gamma_0\Omega^2}{4\Delta_F}=\frac{\sigma_0}{A_{in}}\frac{\Gamma_0^2}{4\Delta_F^2}\dot{N}_L $ and the effective mode area $ A_{in}=1/n_g|u_{\mathrm{in}}(\br'\!_\perp)|^2 $.
Now we can define the OD per atom for the Faraday interaction using SCS by
\begin{align}
\frac{\mathrm{OD}}{N_A} \equiv \frac{\kappa}{\gamma_s}=\frac{\sigma_0A_{in}}{A_F^2}.
\end{align}

Since the Faraday measurement strength doesn't depend on the azimuthal direction of the atom position, ideally, to implement an optimal Faraday interaction geometry, it is preferable to place atoms along some azimuthal direction with the minimum impact from birefringence effects as well as the decoherence due to the presence of the nanofiber. 

Firstly, in the far-detuning regime, since the birefringence effect is dominated by the scalar coupling which is proportional to the intensity difference between the $H$ ($x$)- and $V$ ($ y $)-mode components at the atom positions, both diagonal and anti-diagonal directions in the transverse plane yields a minimum birefringence effect where the intensity of the $ H $- and $ V $-mode components are equal, based on the symmetry of the fiber and given a diagonally polarized $ D $-mode input.
That is the optimal position of atoms with vanished birefringence effect due to the intensity difference of the local $ H $- and $ V $-modes could be along the $ \phi'=n\pi/4 $ ($ n=1,3,5,7 $) radial directions. 
Secondly, to minimize the decoherence damages to spin squeezing, the optimal position of the atom should be chosen so that $ A_{\mathrm{in}} $ reaches the minimum value, which yields $ \phi'=3\pi/4 $ or $ 7\pi/4 $ where $ A_{\mathrm{in}}=1/2n_g|u_\phi(r'\!_\perp)|^2 $.
Combining these two factors, we find the optimal choice of atoms' azimuthal positions are along the anti-diagonal direction, that is $\phi'=3\pi/4 $ or $ 7\pi/4 $.

The number of equations needed to include different subspace...

\subsection{Decoherence and total angular momentum quantum number of atoms}

%=================== Simulations with waveguides =====================%
\section{Discussions} \label{Sec::Discussions}
\comment{Leave discussions on how does the internal structure of atoms and the total atomic quantum number affect the spin squeezing parameter for future works.}


%=================== A toy model with a spin-1/2 system =====================%
\subsection{Spin squeezing with a spin-$1/2$ system} \label{Sec::squeezingwithspinhalfsystems}


%============================ spin squeezing with nanofibers ==============================%
\subsection{QND measurement and spin squeezing with nanofibers} \label{Sec::Nanofiber}

The optimal choice of atom position.

Squeezing results.

Comparison with free space and birefringence.


%=================== Comparison and generalization =====================%
\subsection{Spin squeezing with rectangular waveguide and general criteria in optimizing spin squeezing effect} \label{Sec::Waveguide}

Generalized OD/$N_A$ using Green function method (or may be just with two orthogonal mode bases).

Theoretical upper bounds for the Birefringence and Faraday protocols using a waveguide:

1. The maximum achievable OD$ /N_A $ can be defined when the two orthogonal $ H $- and $ V $-modes are purely linearly polarized (transverse modes).

2. The relationship between the upper bound OD$ /N_A $ and spin-squeezing parameter due to the waveguide effect and effective mode area at the atom position. 
General rules to find the optimal choice of atom positions for the Birefringence and Faraday spin squeezing protocols.


As one example of near-linearly polarized modes, we can consider using a dielectric waveguide with a square intersection. 
Coupling strength, decoherence, and spin squeezing analysis based on the modes of the squared waveguides to show improvement compared with the nanofiber geometry...

Relation of OD$ /N_A $ to cooling efficiency and fictitious magnetic field applications.

%====== SECTION: Summary and outlook ======%
\section{Summary and Outlook} \label{Sec::Conclusion}


ACKNOWLEDGMENTS
We thank the UNM Center for Advanced Research Computing for computational resources used in this work.
This work was supported by the NSF, under grants PHY-1212445, xxxxx.

\bibliography{refs/Archive}


%=========== APPENDIX ===========%
\begin{appendix}

%===================APPENDIX: Hamiltonian =====================%
\section{Faraday interaction Hamiltonian} \label{Appendix::FaradayInteractionHamiltonian}
In the Faraday interaction spin squeezing protocol, we define the fiducial, coupled and transfer states by $ \ket{\uparrow}=\ket{f=4,f_x=4} $, $ \ket{\downarrow}=\ket{f=4,f_x=3} $ and $ \ket{T}=\ket{f=4,f_x=2} $ respectively in the $ x $-basis. 
A set of spin operators projected onto the truncated qutrit subspace spanned by these three basis states can be defined by
\begin{align}
\hat{f_x} &= -f \ket{\uparrow}\bra{\uparrow} -(f-1)\ket{\downarrow}\bra{\downarrow}-(f-2)\ket{T}\bra{T},\\
\hat{f_y} &=i\left[\sqrt{\frac{f}{2}}\left(\ket{\downarrow}\bra{\uparrow}-\ket{\uparrow}\bra{\downarrow}\right) +\sqrt{\frac{2f-1}{2}}\left(\ket{T}\bra{\downarrow}-\ket{\downarrow}\bra{T} \right) \right] ,\\
\hat{f_z} &= \sqrt{\frac{f}{2}}\left(\ket{\downarrow}\bra{\uparrow}+\ket{\uparrow}\bra{\downarrow}\right) +\sqrt{\frac{2f-1}{2}}\left(\ket{T}\bra{\downarrow}+\ket{\downarrow}\bra{T} \right).
\end{align}

We also define a set of Stokes operators as below,
\begin{align}
\hat{S}_0 &= \smallfrac{1}{2}\big[ \hat{a}^\dag_H(t) \hat{a}_H(t)+\hat{a}^\dag_V(t) \hat{a}_V(t) \big],\\
\hat{S}_1 &= \smallfrac{1}{2}\big[ \hat{a}^\dag_H(t) \hat{a}_H(t)-\hat{a}^\dag_V(t) \hat{a}_V(t) \big],\\
\hat{S}_2 &= \smallfrac{1}{2}\big[ \hat{a}^\dag_H(t) \hat{a}_V(t)+\hat{a}^\dag_V(t) \hat{a}_H(t) \big],\\
\hat{S}_3 &= \smallfrac{1}{2i}\big[ \hat{a}^\dag_H(t) \hat{a}_V(t) -\hat{a}^\dag_V(t) \hat{a}_H(t) \big],
\end{align}
where the photon annihilation operators of the horizontally and vertically polarized modes denoted with subscription $ H $ and $ V $ respectively are related to the left- and right-circularly polarized modes denoted with subscription $ L $ and $ R $ by $ \hat{a}_{H}=(\hat{a}_L+\hat{a}_R) /\sqrt{2}$ and $ \hat{a}_{V}=i(\hat{a}_R-\hat{a}_L)/\sqrt{2} $.
The local quasimonochromatic electric field operator at $\br= (r\!_\perp,\phi,z) $ with a set of degenerate guided modes at frequency $ \omega_0 $ and propagating in the group velocity of $ v_g $ can be given by
\begin{align}\label{eq:Ebp}
\hat{\mathbf{E}}^{(+)}(r\!_\perp,\phi,z;t) &= \sum_{b,p} \sqrt{ \frac{2 \pi \hbar \omega_0}{ v_g} } \mathbf{u}_{b,p}(r\!_\perp,\phi) \hat{a}_{b,p}(z,t)  e^{i (b\beta_0 z- \omega_0 t)},
\end{align}
where $ b $ is the sign of the propagation direction and $ p $ is the polarization label of the orthonormal modes.
In our case, we only have forward propagating modes for the QND measurement protocols, and hence we will only have $ b=+ $.

\comment{Todo: Rewrite the effective Hamiltonian using the collective operators defined above.}

%===================APPENDIX: Circular VS linear polarization mode bases =====================%
\section{Circular V.S. linear polarization bases}\label{Appendix:LRbases}
\comment{This content is mainly typed from my notes on Faraday Interaction in The Nanofiber Geometry. More can be done here.}

We define a set of polarization vector transformation relationships by 
\begin{subequations}
\begin{align}
\hat{a}_H &= \frac{1}{\sqrt{2}}(\hat{a}_R+\hat{a}_L )\\
\hat{a}_V &= \frac{i}{\sqrt{2}}(\hat{a}_R-\hat{a}_L ),
\end{align}
\end{subequations}
or the inverse
\begin{subequations}
\begin{align}
\hat{a}_R &= \frac{1}{\sqrt{2}}(\hat{a}_H-i\hat{a}_V )\\
\hat{a}_L &= \frac{1}{\sqrt{2}}(\hat{a}_H+i\hat{a}_V ),
\end{align}
\end{subequations}
where $ R $($ L $) indicates the right(left)-circularly polarized mode.
The Stokes operators can then be defined in both linear ($ H $ and $ V $) and circular ($ L $ and $ R $) polarization bases by
\begin{subequations}\label{eq:SaHVaRL}
\begin{align}
\hat{S}_0 &= \frac{1}{2} \left[\hat{a}_H^\dagger\hat{a}_H+\hat{a}_V^\dagger\hat{a}_V \right] = \frac{1}{2} \left[\hat{a}_R^\dagger\hat{a}_R+\hat{a}_L^\dagger\hat{a}_L \right]\\
\hat{S}_1 &= \frac{1}{2} \left[\hat{a}_H^\dagger\hat{a}_H-\hat{a}_V^\dagger\hat{a}_V \right] = \frac{1}{2} \left[\hat{a}_R^\dagger\hat{a}_L+\hat{a}_L^\dagger\hat{a}_R \right]\\
\hat{S}_2 &= \frac{1}{2} \left[\hat{a}_H^\dagger\hat{a}_V+\hat{a}_V^\dagger\hat{a}_H \right] = \frac{i}{2} \left[\hat{a}_L^\dagger\hat{a}_R-\hat{a}_R^\dagger\hat{a}_L \right]\\
\hat{S}_3 &= \frac{1}{2i} \left[\hat{a}_H^\dagger\hat{a}_V-\hat{a}_V^\dagger\hat{a}_H \right] = \frac{1}{2} \left[\hat{a}_R^\dagger\hat{a}_R-\hat{a}_L^\dagger\hat{a}_L \right].
\end{align}
\end{subequations}
The inversed transformations can be easily derived by inverting the transformation coefficient matrices. 

Based on Eq.\eqref{eq:Ebp}, the E-field operator when the input probe is linearly polarized along the diagonal ($ D $) direction can be written in the linear and circular polarization bases by 
\begin{align}
\hat{\mathbf{E}}^{(+)}(r\!_\perp,\phi,z;t) &= \sqrt{ \frac{2 \pi \hbar \omega_0}{ v_g} } \left[\mathbf{u}_H(r\!_\perp,\phi) \hat{a}_H(z,t) + \mathbf{u}_V(r\!_\perp,\phi) \hat{a}_V(z,t)\right]  e^{i (\beta_0 z- \omega_0 t)}\\
&= \sqrt{ \frac{2 \pi \hbar \omega_0}{ v_g} } \frac{1}{\sqrt{2}}\left[\mathbf{u}_R(r\!_\perp,\phi) \hat{a}_R(z,t) + \mathbf{u}_L(r\!_\perp,\phi) \hat{a}_L(z,t) \right.\nn\\
&\quad\quad\quad\quad\quad\quad\left. + i\mathbf{u}_R(r\!_\perp,\phi) \hat{a}_R(z,t) - i\mathbf{u}_L(r\!_\perp,\phi) \hat{a}_L(z,t)\right]  e^{i (\beta_0 z- \omega_0 t)}\\
&= \sqrt{ \frac{2 \pi \hbar \omega_0}{ v_g} } \left[e^{i\frac{\pi}{4}}\mathbf{u}_R(r\!_\perp,\phi) \hat{a}_R(z,t) + e^{-i\frac{\pi}{4}}\mathbf{u}_L(r\!_\perp,\phi) \hat{a}_L(z,t)\right]  e^{i (\beta_0 z- \omega_0 t)}.
\end{align}
Therefore, the effective atom-light interaction Hamiltonian can be given in those bases by 
\begin{align}
\hat{h}_\eff &= -\hat{\mathbf{E}}^{(-)}(\br')\cdot\hat{\tensor{\mathbf{\alpha}}}\cdot\hat{\mathbf{E}}^{(+)}(\br')\nn\\
&= \hbar\left[(\hat{\chi}_{HH}+\hat{\chi}_{VV})\hat{S}_0 + (\hat{\chi}_{HH}-\hat{\chi}_{VV})\hat{S}_1 + (\hat{\chi}_{HV}+\hat{\chi}_{VH})\hat{S}_2 + i(\hat{\chi}_{HV}-\hat{\chi}_{VH})\hat{S}_3 \right]\\
&= \hbar\left[(\hat{\chi}_{RR}+\hat{\chi}_{LL})\hat{S}_0 + i(\hat{\chi}_{LR}-\hat{\chi}_{RL})\hat{S}_1 + (\hat{\chi}_{RL}+\hat{\chi}_{LR})\hat{S}_2 + (\hat{\chi}_{RR}-\hat{\chi}_{LL})\hat{S}_3 \right]\\
&=\hbar\sum_{i=0}^3 \hat{\chi}_{i}\hat{S}_i\\
&=\hbar\sum_{i,j=0} \chi_{ij}\hat{f}_i\hat{S}_j,
\end{align}
with $\hat{\chi}_{pp'} $ defined in Eq.\eqref{eq:chippp}.

More generally, if we define an arbitrary linear polarization basis by
\begin{align}
\left(\!\begin{array}{c}
\mathbf{e}_n \\ \mathbf{e}_{\bar{n}}
\end{array}\!\right) &= 
\left(\!\!\begin{array}{cc}
\cos\theta & \sin\theta \\
- \sin\theta & \cos\theta
\end{array}\!\!\right)\bullet
\left(\!\begin{array}{c}
\mathbf{e}_H \\ \mathbf{e}_V
\end{array}\!\right)
=\mathbf{R}(\theta)\bullet \left(\!\begin{array}{c}
\mathbf{e}_H \\ \mathbf{e}_V
\end{array}\!\right),
\end{align}
or the inversed relationship
\begin{align}
\left(\!\begin{array}{c}\mathbf{e}_H \\ \mathbf{e}_V\end{array}\!\right)&= \mathbf{R}^{-1}(\theta)\bullet\left(\!\begin{array}{c}\mathbf{e}_n \\ \mathbf{e}_{\bar{n}}\end{array}\!\right),
\end{align}
where $ \theta $ is the angle of the $ \mathbf{e}_n $ basis rotated from the $ H $-direction around $ z $-axis, and $ \mathbf{e}_{\bar{n}} $ is the basis vector $ 90^\circ $ from the $ \mathbf{e}_n $ direction; $ \mathbf{R}(\theta)=\mathbf{R}_z(\theta) $ is the Euler rotation matrix about the $ z $-axis by $ \theta $ in the real-number $ \mathbf{SO}(3) $ rotation group, which has the property that $ \mathbf{R}^{-1}(\theta)=\mathbf{R}^T(\theta)=\mathbf{R}(-\theta) $. 
In general, the basis transformation matrix is an unitary matrix determined by two parameters (two degrees of freedom)--$ \theta $ and $ \phi $--corresponding to the rotating angles around one axis and an relative phase between the base components, which is in the $ \mathbf{SU}(2) $ group.
%We denote the general case with $ \mathbf{R}=\mathbf{R}(\theta,\phi) $, or in the form of two-step rotations around $ i $-axis and then around $ j $-axis by $ \mathbf{R}=\mathbf{R}_j(\theta)\mathbf{R}_i(\phi) $, always satisfying $ \mathbf{R}^{-1}=\mathbf{R}^\dagger $ for either rotations.

Not to be confused, we have also defined an operator space spanned by operator vectors, like $ \left(\!\begin{array}{cc}\mathbf{e}_n,&\mathbf{e}_{\bar{n}}\end{array}\! \right) $, which has vectors, tensors or operators as the elements.
We have also defined the bullet operator ($ \bullet $) in the operator vector space isomorphically the same as the dot ($ \cdot $) product or matrix product in the conventional vector space while the sign of $ \cdot $ can usually be ignored and we will denote complex conjugates explicitly if needed. 
$ \mathbf{R}(\theta) $ and its transformations is a tensor defined in the operator vector space as well.
When a conventional vector or tensor multiplies with an operator vector or tensor, we will use $ \cdot $ between them and the conventional vector or tensor will be formally treated as a scalar to be $ \cdot $ multiplied with the elements of the operator vector or tensor. 
Two operator vectors in a $ \bullet $ multiplication form a mutual covariant relationship in the operator space.
 
With the coordinate basis rotated passively, both the mode components and the field annihilation operators should be rotated actively by $ -\theta $ to be transformation-equivalent.
Written in the matrix form in the operator space, 
\begin{align}
\left(\!\begin{array}{c}\mathbf{u}_n \\ \mathbf{u}_{\bar{n}}\end{array}\!\right) &= \mathbf{R}^{-1}(\theta)\bullet\left(\!\begin{array}{c}\mathbf{u}_H \\ \mathbf{u}_V\end{array}\!\right)\\
\hat{\mathbf{a}}_{n,\bar{n}} &=\mathbf{R}^{-1}(\theta)\bullet \hat{\mathbf{a}}_{H,V},
\end{align}
where $ \hat{\mathbf{a}}_{n,\bar{n}}=[\hat{a}_n;\hat{a}_{\bar{n}}] $ and $ \hat{\mathbf{a}}_{H,V}=[\hat{a}_H;\hat{a}_V] $ are the annihilation operator vectors using the $ \{n,\bar{n} \} $ and $ \{H,V \} $ bases, respectively.
In our notation, we use $ [\cdot ;\cdots] $ notation to indicate $ 1\times n $ vectors as general matrices.

Using the definition in the $ \{ H,V\} $-basis of the Stokes operators (Eq.\eqref{eq:SaHVaRL}) and the annihilation operator basis transformation relationships above, one can rewrite the Stokes operators in the $ \{\mathbf{e}_n, \mathbf{e}_{\bar{n}}\} $ basis by
\begin{subequations}\label{eq:Snnbar}
\begin{align}
\hat{S}_0 &= \frac{1}{2}\left[\left(\!\begin{array}{cc}\hat{a}_n^\dagger,& \hat{a}_{\bar{n}}^\dagger\end{array} \!\right)\bullet\mathbf{R}_{[:,1]}(\theta)\bullet\mathbf{R}^\dagger_{[1,:]}(\theta)\bullet
\left(\!\begin{array}{c}\hat{a}_n\\ \hat{a}_{\bar{n}}\end{array} \!\right)
+ \left(\!\begin{array}{cc}\hat{a}_n^\dagger,& \hat{a}_{\bar{n}}^\dagger\end{array} \!\right)\bullet\mathbf{R}_{[:,2]}(\theta)\bullet\mathbf{R}^\dagger_{[2,:]}(\theta)\bullet
\left(\!\begin{array}{c}\hat{a}_n\\ \hat{a}_{\bar{n}}\end{array} \!\right) \right]\nn\\
&= \frac{1}{2}\left\{\left(\!\begin{array}{cc}\hat{a}_n^\dagger,& \hat{a}_{\bar{n}}^\dagger\end{array} \!\right)\bullet
\left[\left(\!\begin{array}{cc}\cos^2\theta,& -\frac{1}{2}\sin 2\theta \\ -\frac{1}{2}\sin 2\theta, & \sin^2\theta\end{array} \!\right)
+ \left(\!\begin{array}{cc}\sin^2\theta,& \frac{1}{2}\sin 2\theta \\ \frac{1}{2}\sin 2\theta, & \cos^2\theta\end{array} \!\right)\right]\bullet
\left(\!\begin{array}{c}\hat{a}_n\\ \hat{a}_{\bar{n}}\end{array} \!\right)\right\}\nn\\
&=\frac{1}{2} \left[\hat{a}_n^\dagger\hat{a}_n+\hat{a}_{\bar{n}}^\dagger\hat{a}_{\bar{n}} \right]\\
\hat{S}_1 &= \frac{1}{2}\left\{\left(\!\begin{array}{cc}\hat{a}_n^\dagger,& \hat{a}_{\bar{n}}^\dagger\end{array} \!\right)
\bullet\left[\mathbf{R}_{[:,1]}(\theta)\bullet\mathbf{R}^\dagger_{[1,:]}(\theta) - \mathbf{R}_{[:,2]}(\theta)\bullet\mathbf{R}^\dagger_{[2,:]}(\theta) \right]
\bullet\left(\!\begin{array}{c}\hat{a}_n\\ \hat{a}_{\bar{n}}\end{array} \!\right) \right\}\nn\\
&= \frac{1}{2} \left[\cos 2\theta \hat{a}_n^\dagger\hat{a}_n - \sin 2\theta \hat{a}_n^\dagger\hat{a}_{\bar{n}} - \sin 2\theta \hat{a}_{\bar{n}}^\dagger\hat{a}_n -\cos 2\theta \hat{a}_{\bar{n}}^\dagger\hat{a}_{\bar{n}} \right]\\
\hat{S}_2 &= \frac{1}{2}\left\{\left(\!\begin{array}{cc}\hat{a}_n^\dagger,& \hat{a}_{\bar{n}}^\dagger\end{array} \!\right)
\bullet\left[\mathbf{R}_{[:,1]}(\theta)\bullet\mathbf{R}^\dagger_{[2,:]}(\theta) + \mathbf{R}_{[:,2]}(\theta)\bullet\mathbf{R}^\dagger_{[1,:]}(\theta) \right]
\bullet\left(\!\begin{array}{c}\hat{a}_n\\ \hat{a}_{\bar{n}}\end{array} \!\right) \right\}\nn\\
&= \frac{1}{2} \left[\sin 2\theta \hat{a}_n^\dagger\hat{a}_n + \cos 2\theta \hat{a}_n^\dagger\hat{a}_{\bar{n}} + \cos 2\theta \hat{a}_{\bar{n}}^\dagger\hat{a}_n -\sin 2\theta \hat{a}_{\bar{n}}^\dagger\hat{a}_{\bar{n}} \right]\\
\hat{S}_3 &= \frac{1}{2i}\left\{\left(\!\begin{array}{cc}\hat{a}_n^\dagger,& \hat{a}_{\bar{n}}^\dagger\end{array} \!\right)
\bullet\left[\mathbf{R}_{[:,1]}(\theta)\bullet\mathbf{R}^\dagger_{[2,:]}(\theta) - \mathbf{R}_{[:,2]}(\theta)\bullet\mathbf{R}^\dagger_{[1,:]}(\theta) \right]
\bullet\left(\!\begin{array}{c}\hat{a}_n\\ \hat{a}_{\bar{n}}\end{array} \!\right) \right\}\nn\\
&= \frac{1}{2i} \left[\hat{a}_n^\dagger\hat{a}_{\bar{n}} - \hat{a}_{\bar{n}}^\dagger\hat{a}_n  \right].
\end{align}
\end{subequations}
In deriving the equations above, we have denoted $ \mathbf{R}_{[:,i]}(\theta) $ as the $ i $-th column of $ \mathbf{R}(\theta) $ and $ \mathbf{R}_{[i,:]}^\dagger(\theta) $ as the $ i $-th row of $ \mathbf{R}^\dagger(\theta) $.
As a shorthand, these relationships of Stokes operators can be expressed as an operator transformation, $ \hat{\mathbf{S}}=\mathbf{M}\bullet\hat{\mathbf{A}}_{n,\bar{n}} $, where $ \hat{\mathbf{S}}=[\hat{S}_0;\hat{S}_1;\hat{S}_2;\hat{S}_3] $ and $ \hat{\mathbf{A}}_{n,\bar{n}}=[\hat{a}_n^\dagger\hat{a}_n;\hat{a}_n^\dagger\hat{a}_{\bar{n}};\hat{a}_{\bar{n}}^\dagger\hat{a}_n;\hat{a}_{\bar{n}}^\dagger\hat{a}_{\bar{n}}] $ are the operator vectors and $ \mathbf{M} $ is the transformation matrix defined by the transformation coefficients in Eqs.\eqref{eq:Snnbar}.
One can prove that the inversed transformation matrix $ \mathbf{M}^{-1}=2\mathbf{M}^\dagger $, and $ \mathbf{M} $ can be derived in the following form, in general,
\begin{align}
\mathbf{M} &=\frac{1}{2}\left(\!\begin{array}{c}
\mathrm{vec}_r\left[\mathbf{R}_{[:,1]}(\theta)\mathbf{R}^\dagger_{[1,:]}(\theta) + \mathbf{R}_{[:,2]}(\theta)\mathbf{R}^\dagger_{[2,:]}(\theta) \right]\\
\mathrm{vec}_r\left[\mathbf{R}_{[:,1]}(\theta)\mathbf{R}^\dagger_{[1,:]}(\theta) - \mathbf{R}_{[:,2]}(\theta)\mathbf{R}^\dagger_{[2,:]}(\theta) \right]\\
\mathrm{vec}_r\left[\mathbf{R}_{[:,1]}(\theta)\mathbf{R}^\dagger_{[2,:]}(\theta) + \mathbf{R}_{[:,2]}(\theta)\mathbf{R}^\dagger_{[1,:]}(\theta) \right]\\
-i\mathrm{vec}_r\left[\mathbf{R}_{[:,1]}(\theta)\mathbf{R}^\dagger_{[2,:]}(\theta) - \mathbf{R}_{[:,2]}(\theta)\mathbf{R}^\dagger_{[1,:]}(\theta) \right]
 \end{array} \!\right)\\
&= \frac{1}{2}\left(\!\begin{array}{cccc} 1,&0,&0,& 1\\
\cos 2\theta,&-\sin 2\theta, & -\sin 2\theta, & -\cos 2\theta\\
\sin 2\theta, & \cos 2\theta, & \cos 2\theta, & -\sin 2\theta\\
0,&-i,& i,& 0\end{array}\!\right),
\end{align}
where $ \mathrm{vec}_r[\cdot] $ means the vectorization of a matrix by concatenating its rows. 


By using the vector space representation, the E-field operator can be written in the $ \{n,\bar{n}\} $ basis by 
\begin{align}
\hat{\mathbf{E}}^{(+)}(\br;t) &= \sqrt{ \frac{2 \pi \hbar \omega_0}{ v_g} } \left[\mathbf{u}_H(r\!_\perp,\phi) \hat{a}_H(z,t) + \mathbf{u}_V(r\!_\perp,\phi) \hat{a}_V(z,t)\right]  e^{i (\beta_0 z- \omega_0 t)}\\
&= \sqrt{ \frac{2 \pi \hbar \omega_0}{ v_g} } \left(\!\begin{array}{cc}\mathbf{u}_H, & \mathbf{u}_V\end{array}\!\right)
\bullet \left(\!\begin{array}{c}\hat{a}_H\\ \hat{a}_V\end{array} \!\right)  e^{i (\beta_0 z- \omega_0 t)}\\
&= \sqrt{ \frac{2 \pi \hbar \omega_0}{ v_g} } \left(\!\begin{array}{cc}\mathbf{u}_H, & \mathbf{u}_V\end{array}\!\right)\bullet\mathbf{R}(\theta)
\bullet \mathbf{R}^{-1}(\theta)\bullet\left(\!\begin{array}{c}\hat{a}_H\\ \hat{a}_V\end{array} \!\right)  e^{i (\beta_0 z- \omega_0 t)}\\
&= \sqrt{ \frac{2 \pi \hbar \omega_0}{ v_g} } \left(\!\begin{array}{cc}\mathbf{u}_n, & \mathbf{u}_{\bar{n}}\end{array}\!\right)
\bullet \left(\!\begin{array}{c}\hat{a}_n\\ \hat{a}_{\bar{n}}\end{array} \!\right)  e^{i (\beta_0 z- \omega_0 t)}\\
&= \sqrt{ \frac{2 \pi \hbar \omega_0}{ v_g} } \left[\mathbf{u}_n(r\!_\perp,\phi) \hat{a}_n(z,t) + \mathbf{u}_{\bar{n}}(r\!_\perp,\phi) \hat{a}_{\bar{n}}(z,t)\right]  e^{i (\beta_0 z- \omega_0 t)}.\label{eq:Efieldop_nnbar}
\end{align}
As expected, the field operator preserves the form in the new basis as defined in Eq.\eqref{eq:Ebp}.

Therefore, the effective atom-light interaction Hamiltonian can be written as
\begin{align}
\hat{h}_\eff &= -\hat{\mathbf{E}}^{(-)}(\br')\cdot\hat{\tensor{\mathbf{\alpha}}}\cdot\hat{\mathbf{E}}^{(+)}(\br')\nn\\
&=-\frac{2 \pi \hbar \omega_0}{ v_g}
\left(\!\begin{array}{cc}\hat{a}_n^\dagger, & \hat{a}_{\bar{n}}^\dagger\end{array} \!\right)\bullet \left(\!\begin{array}{c}\mathbf{u}_n^*\\ \mathbf{u}_{\bar{n}}^*\end{array}\!\right)
\cdot\hat{\tensor{\mathbf{\alpha}}}\cdot
\left(\!\begin{array}{cc}\mathbf{u}_n, & \mathbf{u}_{\bar{n}}\end{array}\!\right)
\bullet \left(\!\begin{array}{c}\hat{a}_n\\ \hat{a}_{\bar{n}}\end{array} \!\right) \nn\\
&= \hbar \left(\!\begin{array}{cc}\hat{a}_n^\dagger, & \hat{a}_{\bar{n}}^\dagger\end{array} \!\right)\bullet
\left(\!\begin{array}{cc} \hat{\chi}_{nn},&\hat{\chi}_{n\bar{n}}\\
\hat{\chi}_{\bar{n}n},&\hat{\chi}_{\bar{n}\bar{n}}\end{array} \!\right)
\bullet\left(\!\begin{array}{c}\hat{a}_n\\ \hat{a}_{\bar{n}}\end{array} \!\right) \nn\\
&= \hbar \hat{\boldsymbol{\chi}}_{n,\bar{n}}^\dagger\bullet \hat{\mathbf{A}}_{n,\bar{n}}= \hbar \hat{\boldsymbol{\chi}}_{n,\bar{n}}^\dagger\bullet 2\mathbf{M}^\dagger\bullet \hat{\mathbf{S}}\\
&= \hbar\left\{(\hat{\chi}_{nn}+ \hat{\chi}_{\bar{n}\bar{n}})\hat{S}_0 \right.\nn\\
&\quad+ [\cos 2\theta \hat{\chi}_{nn}- \sin 2\theta(\hat{\chi}_{n\bar{n}}+\hat{\chi}_{\bar{n}n}) - \cos 2\theta\hat{\chi}_{\bar{n}\bar{n}}]\hat{S}_1 \nn\\
&\quad+ [\sin 2\theta \hat{\chi}_{nn}+ \cos 2\theta(\hat{\chi}_{n\bar{n}}+\hat{\chi}_{\bar{n}n}) - \sin 2\theta \hat{\chi}_{\bar{n}\bar{n}}]\hat{S}_2 \nn\\
&\quad+\left. i \left(\hat{\chi}_{n\bar{n}}-\hat{\chi}_{\bar{n}n}\right)\hat{S}_3 \right\}.\label{eq:heff_nnbarChiS}
\end{align}
In deriving the expression above, we have defined $\hat{\boldsymbol{\chi}}_{n,\bar{n}}=[\hat{\chi}_{nn};\hat{\chi}_{n\bar{n}};\hat{\chi}_{\bar{n}n};\hat{\chi}_{\bar{n}\bar{n}}]$ as the coupling operator vector.
The coupling coefficient elements of $ 2\mathbf{M}^\dagger $, $ 2M_{ij} $, correspond to the coupling coefficients in Eq.\eqref{eq:heff_nnbarChiS} between the spin operators in $ \hat{\boldsymbol{\chi}}_{n,\bar{n}} $ and the polarization operators in $ \hat{\mathbf{S}} $, and each column of $ 2\mathbf{M}^\dagger $ corresponds to the coefficients in the corresponding line of the $ \hat{S}_i $ coupling term in Eq.\eqref{eq:heff_nnbarChiS}.


If we set $ \theta=\pi/4 $, we will be in the $ \{\mathbf{e}_D,\mathbf{e}_{\bar{D}} \} $ basis.
\begin{subequations}
\begin{align}
\hat{a}_D &= \frac{1}{\sqrt{2}}(\hat{a}_H-\hat{a}_V)\\
\hat{a}_{\bar{D}} &= \frac{1}{\sqrt{2}}(\hat{a}_H+\hat{a}_V),
\end{align}
\end{subequations}
or,
\begin{subequations}
\begin{align}
\hat{a}_H &= \frac{1}{\sqrt{2}}(\hat{a}_D+\hat{a}_{\bar{D}})\\
\hat{a}_V &= \frac{1}{\sqrt{2}}(\hat{a}_{\bar{D}}-\hat{a}_D).
\end{align}
\end{subequations}
The Stokes operators becomes
\begin{subequations}
\begin{align}
\hat{S}_0 &= \frac{1}{2} \left[\hat{a}_D^\dagger\hat{a}_D+\hat{a}_{\bar{D}}^\dagger\hat{a}_{\bar{D}} \right]\\
\hat{S}_1 &= -\frac{1}{2} \left[ \hat{a}_D^\dagger\hat{a}_{\bar{D}} +  \hat{a}_{\bar{D}}^\dagger\hat{a}_D \right]\\
\hat{S}_2 &= \frac{1}{2} \left[\hat{a}_D^\dagger\hat{a}_D - \hat{a}_{\bar{D}}^\dagger\hat{a}_{\bar{D}} \right]\\
\hat{S}_3 &= \frac{1}{2i} \left[\hat{a}_D^\dagger\hat{a}_{\bar{D}}-\hat{a}_{\bar{D}}^\dagger\hat{a}_D \right].
\end{align}
\end{subequations}
The field operator becomes 
\begin{align}
\hat{\mathbf{E}}^{(+)}(\br;t) &=\sqrt{ \frac{2 \pi \hbar \omega_0}{ v_g} } \left[\mathbf{u}_D(r\!_\perp,\phi) \hat{a}_D(z,t)+\mathbf{u}_{\bar{D}}(r\!_\perp,\phi) \hat{a}_{\bar{D}}(z,t)  \right]e^{i (\beta_0 z- \omega_0 t)}.
\end{align}
The effective Hamiltonian can be simplified from Eq.\eqref{eq:heff_nnbarChiS} to 
\begin{align}
\hat{h}_\eff &= \hbar[(\hat{\chi}_{DD}+\hat{\chi}_{\bar{D}\bar{D}})\hat{S}_0 -(\hat{\chi}_{D\bar{D}}+\hat{\chi}_{\bar{D}D} )\hat{S}_1\nn\\
&\quad + (\hat{\chi}_{DD}-\hat{\chi}_{\bar{D}\bar{D}})\hat{S}_2+i(\hat{\chi}_{D\bar{D}}-\hat{\chi}_{\bar{D}D} )\hat{S}_3].
\end{align}
The Hamiltonian expression above should satisfy the cyclical transformation from the $ H $- and $ V $-basis expression.

%===================APPENDIX: Rectangular waveguides and mode walkoff ==================== %
\section{Rectangular waveguides and mode walkoff}
Based on Eq.\eqref{eq:Ebp}, if there are two non-degenerate guided modes used for the QND measurement, we will have
\begin{align}\label{eq:EdifferentHV}
\hat{\mathbf{E}}^{(+)}(r\!_\perp,\phi,z;t) &= \sqrt{ \frac{2 \pi \hbar \omega_0}{ v_g^H} } \mathbf{u}_H(r\!_\perp,\phi) \hat{a}_H(z,t)  e^{i (\beta_0^H z- \omega_0 t)}\nn\\
&\quad +\sqrt{ \frac{2 \pi \hbar \omega_0}{ v_g^V} } \mathbf{u}_V(r\!_\perp,\phi) \hat{a}_V(z,t)  e^{i (\beta_0^V z- \omega_0 t)},
\end{align}
where $ v_g^{H/V} $ and $ \beta_0^{H/V} $ are the group velocities and propagation constants of the $ H $- and $ V $-modes, respectively. 
The offset of $ v_g^{H/V} $ and $ \beta_0^{H/V} $ for the two guided modes could lead to a walkoff effect of the modes and cause an intrinsic phase shift.

%===================APPENDIX: Photon scattering and optical pumping rates =====================%
\section{Photon scattering and optical pumping rates} \label{Appendix::Rates}

In this Appendix, we give the explicit expressions for the photon scattering rates following the formalism outlined in~\cite{Deutsch2010a} and ~\cite{Qi2016} while also considering the modification of decay rates of atoms due to the presence of the nanophotonic structures. 

We define the Green's function tensor in presence of a dielectric medium satisfying the equation~\cite{Qi2016},
	\begin{align} \label{Eq::GreensDiffEq}
		\left[ -\nabla\times\nabla\times + n^2(\mbf{r}) k_0^2 \right] \tensor{\mathbf{G}}(\br, \br';\omega_0) &= -4\pi 
k_0^2 \delta^{(3)}(\mathbf{r}-\mathbf{r}') \unittens,
	\end{align}
where $\unittens$ is the unit tensor.
Under the Born approximation, the field at $\br$ due to an input field $ \mathbf{E}_{\inp}(\br) $ and a dipole source placed at $\br'$ can then be written as
\begin{align}
		\mathbf{E}_{\out}(\br) 
		&\approx \mathbf{E}_{\inp}(\br)+ \tensor{\mathbf{G}}^{(+)}(\br , \br'; \omega_0) \cdot 
\tensor{\boldsymbol{\alpha}}\cdot \mathbf{E}_{\inp}(\br'), \label{Eq::ScatteredField}
\end{align}
where $ \tensor{\boldsymbol{\alpha}} $ is the polarizability of the dipole source.
We can specify the dipole source as an atom with arbitrary excited state level $ e $ and ground level $ g $ in the hyperfine structure manifold. 
The atomic decay rate of the excited level $ e $ will be modified due to the presence of the nanophotonic structure and can be given by
\begin{align}
\Gamma_e &= \sum_g \bra{e}\hat{\mathbf{d}}\ket{g} \cdot \mathrm{Im}\left[\tensor{\mathbf{G}}(\br',\br';\omega_{eg}) \right]\cdot \bra{g}\hat{\mathbf{d}}\ket{e}
\end{align}
with the total Green's function tensor as a combination of the contributions from the guided modes and radiative modes, $ \tensor{\mathbf{G}}(\br',\br';\omega_{eg})=\tensor{\mathbf{G}}_{gyd}(\br',\br';\omega_{eg})+\tensor{\mathbf{G}}_{rad}(\br',\br';\omega_{eg}) $.
For simplicity, we will hide the frequency-dependence part from the Green's function tensors below.
Correspondingly, $ \Gamma_e=\Gamma_{e,gyd}+\Gamma_{e,rad} $ can also be decomposed into the guided mode and radiative mode contribution parts. 
In the free space, the local Green's function tensor has a divergent real part and the imaginary part in the CGS units can be given by 
$\mathrm{Im}\left[\tensor{\mathbf{G}}_0\right]=G_0\tensor{\mathbbm{1}}$ with $G_0(\mathbf{r}',\mathbf{r}';\omega_{eg})=\frac{2}{3}k_0^3$.
In presence of a waveguide, the Green's function tensor due to the radiative modes can be decomposed into the free-space or homogeneous radiation contribution and the waveguide-induced radiation contribution, that is $ \mathrm{Im}\left[\tensor{\mathbf{G}}_{rad}(\br',\br')\right]=\tensor{\mathbf{G}}_0(\br',\br')+\tensor{\mathbf{G}}_{ind,rad}(\br',\br';\omega_{eg}) $.
The waveguide-induced Green's function tensor elements due to the radiative modes can be calculated by
$$G_{ind,rad}^{ij}(\mathbf{r}',\mathbf{r}')=\frac{2k_0^2}{3\pi^2}\int_{-k_0}^{k_0} d\beta E_j^i(\mathbf{r}')=\frac{4k_0^2}{3\pi^2}\int_{0}^{k_0} d\beta E_j^i(\mathbf{r}')$$
once $ E_j^i(\mathbf{r}') $--the $i$-th electric field component measured at the dipole position by putting a unit dipole orientated along $j$ direction--can be obtain using numerical methods like boundary element method (BEM) or analytically.

    
What we are interested in is the relative ratio between the modified decay rates for a given $e\rightarrow g$ decay transition and the natural linewidth, $ \Gamma_0 $, of the atoms, that is $ \Gamma_e^q/\Gamma_0 $. 
We can define an equivalent classical dipole, $ \mathbf{d}=\bra{g}\hat{\mathbf{d}}\ket{e} $, corresponding the $ e\rightarrow g $ decay transition which is along the $ \mathbf{e}_q $ ($q=\{\pm,0\}$) unit vector direction in the spherical irreducible harmonic basis, where
\begin{align}
    \mathbf{e}_\pm &=\mp \frac{\mathbf{e}_{\tilde{x}}\pm i\mathbf{e}_{\tilde{y}}}{\sqrt{2}}\\
    \mathbf{e}_0 &=\mathbf{e}_{\tilde{z}}
\end{align}
correspond to the $\sigma_\pm$ and $\pi$ transitions, respectively.
We have defined $ (\tilde{x},\tilde{y},\tilde{z}) $ as the quantization basis.
In our simulation of the square waveguide case, we have used $\varepsilon=4$ (without loss) to calculate the radiative mode induced Green's function tensor elements.
Therefore, the radiative mode caused decay rates can be calculated by
\begin{align}
\frac{\Gamma_{e,rad}^{q}}{\Gamma_0} &= 1+ \frac{\mathrm{Im}\left[\mathbf{e}_q^*\cdot \tensor{\mathbf{G}}_{ind,rad}(\mathbf{r}',\mathbf{r}')\cdot \mathbf{e}_q\right]}{ \mathrm{Im}\left[\mathbf{e}_q^*\cdot \tensor{\mathbf{G}}_0(\mathbf{r}',\mathbf{r}')\cdot \mathbf{e}_q\right]}
=1+ \frac{\mathbf{e}_q^*\cdot \mathrm{Im}\left[\tensor{\mathbf{G}}_{ind,rad}(\mathbf{r}',\mathbf{r}')\right]\cdot \mathbf{e}_q}{G_0(\mathbf{r}',\mathbf{r}')}.
\end{align}
Similarly, the guided mode contribution to the decay rates can be given by
\begin{align}
\frac{\Gamma_{e,gyd}^{q}}{\Gamma_0} &= \frac{\mathrm{Im}\left[\mathbf{e}_q^*\cdot \tensor{\mathbf{G}}_{gyd}(\mathbf{r}',\mathbf{r}')\cdot \mathbf{e}_q\right]}{ \mathrm{Im}\left[\mathbf{e}_q^*\cdot \tensor{\mathbf{G}}_0(\mathbf{r}',\mathbf{r}')\cdot \mathbf{e}_q\right]}
= \frac{\mathbf{e}_q^*\cdot \mathrm{Im}\left[ \tensor{\mathbf{G}}_{gyd}(\mathbf{r}',\mathbf{r}')\right]\cdot \mathbf{e}_q}{ G_0(\mathbf{r}',\mathbf{r}')}.
\end{align}
As has been derived in Ref~\cite{Qi2016},
\begin{align}
\mathrm{Im}[\tensor{\mathbf{G}}_{gyd}(\br',\br')] &= \pi \frac{\omega_{eg}}{v_g } \sum_{b, p} 
		\mathbf{u}_{b, p} (\br_{\!\perp}^\prime)\mathbf{u}^*_{b , p} (\br_{\!\perp}^\prime),
\end{align}
where $\mathbf{u}_{b, p} (\br_{\!\perp}^\prime)$ is the guided mode with propagation direction $ b=\pm 1 $ and polarization degeneracy index $ p $ at the dipole position in the transverse plane of the waveguide crossing section perpendicular to the waveguide axis.
$v_g$ is the group velocity of the degenerate guided modes.
For the nanofiber case, $ p=\pm 1 $ correspond to the right- and left-circularly polarized fundamental $\mathrm{HE}_{11}$ modes; 
for the SWG case, $ p=\mathrm{H}/\mathrm{V} $ corresponding to the H- and V-modes chosen for the spin squeeze protocol.

 
To find the dependence on the input field intensity, we define a characteristic photon scattering rate, $\gamma_s \equiv \frac{\Gamma_0\Omega^2}{4\Delta_{F}^2}= \frac{\sigma_0}{A_{\rm in}}\frac{\Gamma_0^2}{4 \Delta_{F}^2} \dot{N}_L $ with an effective detuning $ \Delta_F $ defined by $ \frac{1}{\Delta_F}=\sum_{f'}\frac{C_{f'ff'}^{(1)}}{\Delta_{ff'}} $ and $ \Delta_{ff'}=\omega-\omega_{ff'} $.
We have also defined Rabi frequency $ \Omega=2\bra{j}|d|\ket{j'}\mathcal{E}^{(+)}_{\rm in}/\hbar $ with reduced optical dipole matrix element $\bra{j}|d|\ket{j'}$ and field amplitude $ \mathcal{E}^{(+)}_{\rm in}=|\mathbf{E}_{\rm in}^{(+)}(\br')| $.
The total rate of photon scattering by an atom from the $\ket{a}\equiv \ket{f,f_x=a}$ to $ \ket{b}\equiv\ket{f,f_x=b} $ hyperfine ground state in the $x$-basis is
	\begin{equation}\label{Eq::gammaf}
		\gamma_{ba}=- \frac{2}{\hbar} {\rm Im} \big[ \bra{f,b} \hat{h}_{\rm eff}\ket{f,a} \big] ,
	\end{equation}
where the effective non-Hermitian light-shift Hamiltonian for one atom is given by
\begin{align}
\gamma_s\hat{h}_{\rm eff} &= - \hat{\mathbf{E}}^{(-)}_{\rm in}(\mathbf{r}' ; t ) \cdot \poltens \cdot \hat{\mathbf{E}}^{(+)}_{\rm in}(\mathbf{r}' ;t ),
\end{align}
where $\charpolq = -\frac{\sigma_0}{8\pi k_0\gamma_s}\frac{\Gamma_{f'}^q}{\Delta_{ff'}^q+i\Gamma_{f'}^q/2}$ is the complex polarizability and the polarizability operator $  \poltens=\sum_{f',q}\charpolq\hat{\tensor{\mathbf{A}}}(f,f')$ with elements of $ \hat{\tensor{\mathbf{A}}} $ given by
\begin{align} \label{Eq::PolarizabilityIrrep}
		\hat{A}_{ij}(f,f')&\equiv \hat{e}_i^*\cdot\hat{\mathbf{D}}_{ff'}\hat{\mathbf{D}}_{f'f}^\dagger \cdot \hat{e}_j \\
		&=  C_{ff'}^{(0)} \delta_{i,j}\hat{\mathbbm{1}}+ iC_{ff'}^{(1)}\epsilon_{ijk}\hat{f}_k+ C_{ff'}^{(2)} \Big[ \smallfrac{1}{2} ( \hat{f}_i\hat{f}_j +\hat{f}_j\hat{f}_i )-\smallfrac{1}{3} \hat{\mathbf{f}}\!\cdot\!\hat{\mathbf{f}} \delta_{i,j} \Big], 
\end{align}
where $\hat{\mathbf{f}}$ is the atomic spin operator in hyperfine multiplet $f$, and $ \epsilon_{ijk} $ is the Levi-Civita symbol. 
In the Faraday interaction protocols, the polarizability induced light response and hence the measurement backaction is mainly determined by the vector interaction term as the second term of the equation above.
Definitions of the $ C_{ff'}^{(n)} $ can be found in Ref~\cite{Qi2016}.

We define the Lindblad jump operators of optical pumping among ground states $\ket{a}\equiv \ket{f,a} \rightarrow \ket{b}\equiv \ket{f,b}$ and generating a $q$ photon emission as~\cite{Deutsch2010a}
	\begin{align}\label{Eq::Wq_Faraday}
		\hat{W}_q &= \frac{1}{\sqrt{\gamma_s}}\sum_{f'}\frac{\Omega/2}{\Delta_{f'f}^q+i\Gamma_{f'}^q/2}\mathbf{e}_q^*\cdot(\hat{\mathbf{D}}_{ff'}  \hat{\mathbf{D}}^\dagger_{f'f} )\cdot\mathbf{e}_{\rm in}\\
		&=\sum_{f',m',q',a,b} w_{baq }^{f'm'q'}\ket{b}\bra{a}.
	\end{align}
Each jump operator $\hat{W}_q$ is associated with absorption of the probe photon polarized along $ \mathbf{e}_{\rm in} $ followed by spontaneous emission of a photon with polarization $ \mathbf{e}_q $, where $q= \{0,\pm 1\}$ labels spherical basis elements for $\pi$ and $ \sigma_\pm$ transitions. 
Here the dimensionless raising operator $ \mathbf{e}_q\cdot\hat{\mathbf{D}}_{f'f}^\dagger= \sum_{m',m} o_{jf}^{j'f'} C_{f',m'}^{f,m;1, q}\ket{f',m'}\bra{f,m} $,
where $ C_{f',m'}^{f,m;1, q}=0 $ unless $ m'=m+q $ with $ C_{f',m+q}^{f,m;1, q}=\Braket{f',m+q}{f,m;1,q}$ being the Clebsch-Gordan coefficients, and
\begin{equation}
\big| o_{jf}^{j'f'} \big|^2=(2j'+1)(2f+2) \bigg\{
\begin{array}{ccc}
f' & 7/2 & j' \\
 j & 1 & f
 \end{array}
 \bigg\}
\end{equation}
are the relative oscillator strengths determined by the relevant Wigner 6-$J$ symbol.
In our protocols, we assume the probe light is so far-detuned from any of the atomic resonances that the tilting of the hyperfine structure levels due to an external magnetic field becomes irrelevant and we can set $ \Delta_{ff'}^q=\Delta_{ff'} $ and $ \Delta_{ff'}\gg \Gamma_f'^q $ for arbitrary $ f' $ and $ q $.


The optical pumping dynamics of the $ j $-th atom are governed by 
\begin{align}
\left.\dt{\hat{\rho}^{(j)}}\right|_{op} &= \gamma_s\mathcal{D}[\hat{\rho}^{(j)}]=-\frac{i\gamma_s}{\hbar}\left\{\hat{h}^{\rm loss}_{\rm eff},\hat{\rho}^{(j)} \right\}_+ + \gamma_s\sum_{q}\hat{W}_q(\br'_j)\hat{\rho}^{(j)}\hat{W}_q(\br'_j)\\
\\
&=-\gamma_s\frac{1}{2}\sum_{a,b}\gamma_{ba}\ket{b}\bra{a}+
\!\!\!\!\!\!\sum_{q,q',q'',a,b,c,d,f',f'',m',m''}\!\!\!\!\!\! \gamma_sw_{dcq}^{f''m''q''}\left(w_{abq}^{f'm'q'}\right)^*\ket{d}\bra{c}\hat{\rho}^{(i)}\ket{b}\bra{a}
\end{align}
Eqs.~\eqref{Eq::gammaf} and~\eqref{Eq::Wq_Faraday} yield,
\begin{subequations}
	\begin{align}
		\gamma_{ba} 
		&=\frac{n_g\dot{N}_L}{\gamma_s}  \sum_{f',q} \sigma (\Delta_{ff'} ) \mathbf{u}^*_\inp(\br'_\perp)\cdot \bra{b} \hat{\tensor{\mbf{A}}}(f,f') \ket{a}  \cdot \mathbf{u}_\inp(\br'_\perp)\\
		&\approx  \sum_{f',m'} \frac{\Delta_{F}^2}{\Delta_{ff'}^2}\sum_{q,q'} \big| o_{jf}^{j'f'} \big|^2C_{f',b+q'}^{f,b;1, q'}C_{f',a+q}^{f,a;1, q} \mathbf{e}_{q'}^* \cdot (\mathbf{e}_{\rm in}\mathbf{e}_{\rm in}^* )\cdot \mathbf{e}_q,
	\end{align}
\end{subequations}
	\begin{align}
		w_{baq}^{f'm'q'}
		&\approx  \frac{\Delta_{F}}{\Delta_{ff'}+i\Gamma_{f'}^q/2} \big| o_{jf}^{j'f'}  \big|^2 C_{f'm'}^{f,b;1 q}C_{f',m'}^{f,a;1,q'} (\mathbf{e}_{q'}^* \cdot \mathbf{e}_{\rm in}),
	\end{align}
where $ \sigma (\Delta_{ff'} )  = \sigma_0 \Gamma_0^2/4\Delta^2_{f' f}$ is the the scattering cross section at the probe detuning in free space. 

\comment{Todo: Need to unify and correct the factor of $\gamma_s$ in the optical pumping master equations and related quantities. The following content is mainly from Optical pumping in the rotating frame in the handwritten notes: QND measurement and spin squeezing using Faraday interactions (part VIII).}

Now, we consider a static magnetic field is applied to the atoms to fix the quantization axis to direction $ \mathbf{e}_{\tilde{z}} $ along the waveguide axis.
We assume the magnetic field is so strong that the Larmor processing is much faster than the atomic decay and atom-photon interaction processes and the transverse components of the atomic angular momentum operators will be averaged out in the process of spin squeezing dynamics.
In theory, this leads us to transfer the master equations of the collective spin dynamics to the rotating frame determined by the fast-rotating transform operator
\begin{align}
\hat{U}_B(t) &= e^{-i\Omega_Bt\hat{f}_z},
\end{align}
where $ \Omega_B $ is the Larmor processing frequency of the external magnetic field.
In the rotating frame, a quantum operator $ \hat{A} $ is transfered into $ \hat{A}' $ through $ \hat{A}\rightarrow \hat{A}'=\expect{\hat{U}_B^\dagger\hat{A}\hat{U}_B }_T $, where the notation $ \expect{\cdot}_T=\frac{1}{T}\int\cdot dt $ is the time average of observables in a period $ T $.
Similarly, the density operator is transfered to the rotating frame through $ \hat{\rho}\rightarrow \hat{\rho}'= \expect{\hat{U}_B^\dagger\hat{\rho}\hat{U}_B }_T $.
We can solve the transformed master equations by employing the Baker-Campbell-Hausdorff formula that $ e^{\lambda\hat{A}}\hat{B}e^{-\lambda\hat{A}}=\sum_{n=0}^\infty\frac{\lambda^n}{n!}\hat{C}_n $, where $ \hat{C}_0=\hat{B} $ and $ \hat{C}_n=\left[\hat{A},\hat{C}_{n-1} \right] $ for $ n>1 $, and the commutators of atomic angular momentum operators, $ \left[\hat{f}_m, \hat{f}_n\right]=i\sum_p\epsilon_{mnp}\hat{f}_p $.
The following static-rotating frame transformation relationships can be proved easily:
\begin{subequations}\label{eq:rotationtransf}
	\begin{align}
	\hat{f}_{x}&=\hat{f}_{y} \rightarrow 0, \quad \hat{f}_{z}\rightarrow\hat{f}_z, \quad \hat{f}_{z}^ 2\rightarrow\hat{f}_z^2,\\
	\hat{f}^2_{x} &= \hat{f}^ 2_{y} \rightarrow \frac{1}{2}(\hat{\mathbf{f}}^2-\hat{f}_z^2),\\
	\hat{f}_{x}\hat{f}_{y} &\rightarrow\frac{1}{2}\hat{f}_z,\quad \hat{f}_{y}\hat{f}_{x}\rightarrow -\frac{1}{2}\hat{f}_z,\quad \hat{f}_{i=x,y}\hat{f}_{z}\rightarrow 0.
	\end{align}
\end{subequations}
We can see that, in the rotating frame, the transverse components of the atomic angular momentum operators got averaged out due to symmetry.

Given the geometry of the Faraday spin squeezing protocols with optical nanofiber and square waveguides, we only consider a linearly polarized local field at the atom position. 
By denoting the local field is polarized along $ x'$-direction, the effective loss Hamiltonian in a static reference frame can be written as 
\begin{align}
\hat{h}_{\rm loss} &= -\frac{i\hbar}{2}\sum_{f'} \gamma'_s \left[C_{j'ff'}^{(0)}\hat{\mathbbm{1}}+C_{j'ff'}^{(2)}(\hat{f}_{x'}^2-\frac{\hat{\mathbf{f}}^2}{3} ) \right],
\end{align}
where the characteristic photon scattering rate in the far-detuning regime $ \gamma'_s=\frac{\Gamma_{f'}\Omega^2}{4(\Delta_{ff'}^2+\Gamma_{f'}^2/4 )} \approx \frac{\Gamma_0\Omega^2}{4\Delta_{\rm eff}}=\frac{\sigma_0}{\Ain}\left(\frac{\Gamma_0}{2\Delta_{\rm eff}} \right)^2\dot{N}_L $.
In the last two steps, we have used the averaged decay rate $ \Gamma_0 $ to replace the hyperfine structure dependent decay rate $ \Gamma_{f'} $ for all excited levels in a fine structure manifold $ j' $; the effective detuning is also an averaged detuning from the fine structure excited level $ j' $ to the ground fine structure manifold $ j $ with resonant frequency $ \omega_D $--that is $ \Delta_{\rm eff}=\omega -\omega_D $ with probe frequency at $ \omega $ in vacuum. 
Compared to the normal characteristic photon scattering rate $ \gamma_s $, we can see they are defined in different scales.
In general, they are related given a transition between ground hyperfine structure level $ f $ and excited hyperfine structure level $ f' $ by 
\begin{align}
\gamma'_s(f')=\gamma_s \frac{\Delta_F^2}{\Delta_{ff'}},
\end{align}
and hence $ \frac{\sqrt{\Gamma_{f'}}\Omega/2}{\Delta_{ff'}\pm i\Gamma_{f'}/2}\approx \frac{\sqrt{\Gamma_0}\Omega/2}{\Delta_{ff'}}=\sqrt{\gamma_s}\frac{\Delta_F}{\Delta_{ff'}} $ in the far-detuning regime.

Using the transformation relationships defined in Eqs.\eqref{eq:rotationtransf}, the loss Hamiltonian in the rotating frame becomes
\begin{subequations}\label{eq:rotationtransf_hloss}
\begin{align}
\hat{h}_{\rm loss} =-\frac{i\hbar}{2} \sum_{f'}\gamma'_s \left[C_{j'ff'}^{(0)}\hat{\mathbbm{1}} + \frac{C_{j'ff'}^{(2)}}{6}(\hat{\mathbf{f}}^2-3\hat{f}_z^2 ) \right],
\end{align}
\end{subequations}
where $ z $-direction is the waveguide axis direction, and $ \hat{\mathbf{f}}^2=\hat{\mathbf{f}}\cdot\hat{\mathbf{f}} $.

Similarly, in the static frame, the jump operators can be given by
\begin{align}
\hat{W}_q &= \sum_{f'} \frac{\sqrt{\Gamma_{f'}}\Omega/2}{\Delta_{ff'}+i\Gamma_{f'}/2}\mathbf{e}_q^*\cdot \hat{\mathbf{D}}_{ff'} \hat{\mathbf{D}}_{f'f}^\dagger \cdot \mathbf{e}_L \nn \\
&= \sum_{f'k} \frac{\sqrt{\Gamma_{f'}}\Omega/2}{\Delta_{ff'}+i\Gamma_{f'}/2} \left[\delta_{qx'}C_{j'ff'}^{(0)}\hat{\mathbbm{1}} + iC_{j'ff'}^{(1)}\epsilon_{qx'k}\hat{f}_k + C_{j'ff'}^{(2)}\left(\frac{\hat{f}_q\hat{f}_{x'}+\hat{f}_{x'}\hat{f}_q }{2} - \frac{\delta_{qx'}}{3}\hat{\mathbf{f}}^2 \right) \right].
\end{align}
Therefore, in the static $ \left\{x',y',z \right\} $ basis, we have
\begin{align}
\hat{W}_{x'} &= \sum_{f'} \frac{\sqrt{\Gamma_{f'}}\Omega/2}{\Delta_{ff'}+i\Gamma_{f'}/2} \left[C_{j'ff'}^{(0)}\hat{\mathbbm{1}} + C_{j'ff'}^{(2)}\left(\hat{f}_{x'}^2-\frac{1}{3}\hat{\mathbf{f}}^2 \right) \right]\\
\hat{W}_{y'} &= \sum_{f'} \frac{\sqrt{\Gamma_{f'}}\Omega/2}{\Delta_{ff'}+i\Gamma_{f'}/2} \left(-iC_{j'ff'}^{(1)}\hat{f}_z + C_{j'ff'}^{(2)}\frac{\hat{f}_{x'}\hat{f}_{y'}+\hat{f}_{y'}\hat{f}_{x'}}{2} \right)\\
\hat{W}_{z} &= \sum_{f'} \frac{\sqrt{\Gamma_{f'}}\Omega/2}{\Delta_{ff'}+i\Gamma_{f'}/2} \left(iC_{j'ff'}^{(1)}\hat{f}_{y'} + C_{j'ff'}^{(2)}\frac{\hat{f}_z\hat{f}_{x'}+\hat{f}_{x'}\hat{f}_z}{2}  \right).
\end{align}
By using the transformation relationships of Eqs.\eqref{eq:rotationtransf}, the jump operators become 
\begin{subequations}\label{eq:rotationtransf_Wxyz}
\begin{align}
\hat{W}_{x'} &= \sum_{f'} \frac{\sqrt{\Gamma_{f'}}\Omega/2}{\Delta_{ff'}+i\Gamma_{f'}/2} \left[C_{j'ff'}^{(0)}\hat{\mathbbm{1}} + \frac{C_{j'ff'}^{(2)}}{6}\left(\hat{\mathbf{f}}^2-3\hat{f}_{z}^2 \right) \right]\\
\hat{W}_{y'} &= \sum_{f'} -\frac{i\sqrt{\Gamma_{f'}}\Omega/2}{\Delta_{ff'}+i\Gamma_{f'}/2} C_{j'ff'}^{(1)}\hat{f}_z  \\
\hat{W}_{z} &= 0.
\end{align}
\end{subequations}

By using the fact that 
\begin{subequations}
\begin{align}
\hat{\mathbf{f}}^2 &=f(f+1)\hat{\mathbbm{1}}\\
\hat{f}_z &= \left(
    \begin{array}{ccccccc}
    f                                      		\\
      & f-1           &      & &      &\text{\huge0}\\
      &               &\ddots& 					\\
      &               &      &0                	\\
      &               &      & &\ddots          \\
      & \text{\huge0} &      & &      &-(f-1)   \\
      &               &      & &      &      &-f
    \end{array}
    \right)=\sum_{m=1}^{2f+1}(f-m+1)\hat{\sigma}_{mm}\\
\hat{f}_z^2 &= \sum_{m=1}^{2f+1}(f-m+1)^2\hat{\sigma}_{mm}
\end{align}
\end{subequations}
in the rotating frame, both $ \hat{h}_{\rm loss} $ and $ \hat{W}_{q} $ become diagonal, and Eqs.\eqref{eq:rotationtransf_hloss} and~\eqref{eq:rotationtransf_Wxyz} can be simplified as
\begin{subequations}
\begin{align}
\hat{h}_{\rm loss} &= -\frac{i\hbar}{2} \sum_{f'} \gamma'_s(f') \sum_{m=-f}^f\left[C_{j'ff'}^{(0)} + \frac{C_{j'ff'}^{(2) }}{6}(f(f+1)-3m^2) \right]\hat{\sigma}_{mm}\\
\hat{W}_{x'} &= \sum_{f'} \frac{\sqrt{\Gamma_{f'} }\Omega/2}{\Delta_{ff'}+i\Gamma_{f'}/2 } \sum_{m=-f}^f\left[C_{j'ff'}^{(0)} + \frac{C_{j'ff'}^{(2) }}{6}(f(f+1)-3m^2) \right]\hat{\sigma}_{mm}\\
\hat{W}_{y'} &= -i \sum_{f'} \frac{\sqrt{\Gamma_{f'} }\Omega/2}{\Delta_{ff'}+i\Gamma_{f'}/2 } \sum_{m=-f}^f C_{f'ff'}^{(1)}m\hat{\sigma}_{mm}\\
\hat{W}_z &=0.
\end{align}
\end{subequations}
By further replacing $ \hat{\rho}\rightarrow \expect{\hat{U}_B^\dagger \hat{\rho}\hat{U}_B }_T$ and $ \left. \dt{\hat{\rho}}\right|_{\rm op}\rightarrow \expect{\left.\dt{\hat{U}_B^\dagger}\right|_{\rm op}\hat{\rho}\hat{U}_B}_T + \expect{\hat{U}_B^\dagger\hat{\rho}\left. \dt{\hat{U}_B}\right|_{\rm op}}_T +\expect{\hat{U}_B^\dagger\left.\dt{\hat{\rho}}\right|_{\rm op}\hat{U}_B}_T= \expect{\hat{U}_B^\dagger\left.\dt{\hat{\rho}}\right|_{\rm op}\hat{U}_B}_T$, one can obtain the optical pumping master equation in the rotating frame as
\begin{align}
\left. \dt{\hat{\rho}}\right|_{\rm op} &= \expect{\hat{U}_B^\dagger\left.\dt{\hat{\rho}}\right|_{\rm op}\hat{U}_B}_T\nn\\
&= \expect{\hat{U}_B^\dagger\gamma_s\mathcal{D}\left[\hat{\rho} \right]\hat{U}_B}_T\nn\\
&= - \sum_{f'} \gamma'_s(f') \left[\left(C_{j'ff'}^{(0)}+\frac{f(f+1)}{12}C_{j'ff'}^{(2)} \right)\hat{\rho}-\frac{C_{j'ff'}^{(2)}}{4}(\hat{f}_z^2\hat{\rho}+\hat{\rho}\hat{f}_z^2) \right]\nn\\
&\quad+\sum_{f',f''} \frac{\sqrt{\Gamma_{f'}\Gamma_{f''} }\Omega^2/4 }{\Delta_{ff'}\Delta_{ff''}+\Gamma_{f'}\Gamma_{f''}/4+i(\Delta_{ff''}\Gamma_{f'}-\Delta_{ff'}\Gamma_{f''} ) }\nn\\
&\quad\cdot\left\{C_{f'ff'}^{(0)}C_{f'ff''}^{(0)}\hat{\rho}+ C_{j'ff'}^{(0)}C_{j'ff''}^{(2)}\frac{\hat{\rho}}{6}(\fo^2-3\hat{f}_z^2) + C_{j'ff''}^{(0)}C_{j'ff'}^{(2)}(\fo^2-3\hat{f}_z^2)\frac{\hat{\rho}}{6} \right.\nn\\
&\quad\quad + \frac{1}{2}C_{j'ff'}^{(1)}C_{j'ff''}^{(1)}(\hat{f}_x\hat{\rho}\hat{f}_x+\hat{f}_y\hat{\rho}\hat{f}_y+2\hat{f}_z\hat{\rho}\hat{f}_z )\nn\\
&\quad\quad -\frac{1}{4}C_{j'ff'}^{(1)}C_{f'ff''}^{(2)}(\fx\rhoo\fx-\fy\rhoo\fy-2i\fx\rhoo\fz\fy+2i\fy\rhoo\fx\fz )\nn\\
&\quad\quad +\frac{1}{4}C_{j'ff''}^{(1)}C_{f'ff'}^{(2)}(\fx\rhoo\fx-\fy\rhoo\fy-2i\fz\fy\rhoo\fx+2i\fx\fz\rhoo\fy ) \nn\\
&\quad +C_{j'ff'}^{(2)}C_{j'ff''}^{(2)}\left[\frac{1}{4}f^2(f+1)^2\rhoo-\frac{f(f+1)}{6}(\fo^2-\fz^2)\rhoo-\frac{f(f+1)}{6}\rhoo(\fo^2-\fz^2) \right.\nn\\
&\quad\quad\quad+\frac{1}{2}\fx^2\rhoo\fx^2+\frac{1}{2}\fy^2\rhoo\fy^2+\frac{1}{4}(i\fz+2\fy\fx)\rhoo(i\fz+2\fy\fx)\nn\\
&\left.\left.\quad\quad\quad +\frac{1}{8}(i\fy+2\fx\fz)\rhoo(i\fy+2\fx\fz)+\frac{1}{8}(i\fx+2\fz\fy)\rhoo(i\fx+2\fz\fy) \right]\right\}.
\end{align}
In the far-detuning regime, the detuning on hyperfine sublevels will be degenerate and $ f'=f'' $ so that we can ignore all tensor polarizability terms, and the optical pumping part of the master equation above becomes
\begin{align}
\left.\dt{\rhoo}\right|_{\rm op} = \gamma'_s \sum_{f'} \left[C_{j'ff'}^{(0)}(C_{j'ff'}^{(0)}-1)\rhoo+\frac{1}{2}(C_{j'ff'}^{(1)})^2(\fx\rhoo\fx+\fy\rhoo\fy+2\fz\rhoo\fz ) \right].
\end{align}


%===================APPENDIX: Equations of motion =====================%
\section{Generalized master equation for spin squeezing dynamics with arbitrary spin number $f$} \label{Appendix::OpticalPumpingForGeneralF}
\comment{Todo: see notes on QND measurement and spin squeezing using Faraday interactions (part VI).}


%===================APPENDIX: Relationships between collective spin operators and microscopic operators =====================%
\section{Relationships between some collective operators and the first- and second-moments in the symmetric subspace} \label{Appendix::collectivespinoperators}

\comment{Todo: Relations between collective angular momentum operators and microscopic $\sigma_{ba}$ operators in the qubit and qutrit symmetric subspace. See handwritten notes: QND measurement and spin squeezing using Faraday interactions (part IV).}
\begin{subequations}
	\begin{align}
	\expect{\hat{F}_x} &= \sum_i^{N_A}\expect{\hat{f}_x}\nonumber\\
	&= -N_A \left[f\expect{\sigmauu}+(f-1)\expect{\sigmadd}+(f-2)\expect{\sigmatt } \right]\\
	\Delta F_z^2 &= N_A\expect{\Delta f_z^2} + N_A(N_A-1)\expect{\Delta f_z^{(1)}\Delta f_z^{(2)} }_s\nn\\
	&=N_A\left\{ \frac{f}{2}\left(\expect{\sigmauu}+\expect{\sigmadd}-2\expect{\sigmaud}\expect{\sigmadu}-\expect{\sigmaud}^2-\expect{\sigmadu}^2 \right)\right. \nn\\
	&\quad\quad+ \frac{2f-1}{2}\left(\expect{\sigmadd}+\expect{\sigmatt}-2\expect{\sigmadt}\expect{\sigmatd}-\expect{\sigmadt}^2-\expect{\sigmatd}^2 \right)\nn\\
	&\quad\quad +\left. \frac{\sqrt{f(2f-1)}}{2}\left[\expect{\sigmaut }+\expect{\sigmatu} -2\expect{\sigmaud}(\expect{\sigmadt}+\expect{\sigmatd}) -2\expect{\sigmadu}(\expect{\sigmadt}+\expect{\sigmatd} ) \right] \right\}\nn\\
	&\quad +N_A(N_A-1)\left[\frac{f}{2}(\expect{\Dsigmaud^{(1)}\Dsigmaud^{(2)} }_s +2\expect{\Dsigmaud^{(1)}\Dsigmadu^{(2)} }_s+\expect{\Dsigmadu^{(1)}\Dsigmadu^{(2)} }_s )\right.\\
	&\quad\quad + \frac{2f-1}{2}(\expect{\Dsigmadt^{(1)}\Dsigmadt^{(2)} }_s +2\expect{\Dsigmadt^{(1)}\Dsigmatd^{(2)} }_s +\expect{\Dsigmatd^{(1)}\Dsigmatd^{(2)} }_s) \nn\\
	&\quad\quad + \left. \sqrt{f(2f-1)}(\expect{\Dsigmaud^{(1)}\Dsigmadt^{(2)} }_s +\expect{\Dsigmaud^{(1)}\Dsigmatd^{(2)} }_s +\expect{\Dsigmadu^{(1)}\Dsigmadt^{(2)} }_s+\expect{\Dsigmadu^{(1)}\Dsigmatd^{(2)} }_s )\right]\nn
	\end{align}
\end{subequations}



\end{appendix}

\end{document}
