\documentclass[preprint,aps,pra,onecolumn,superscriptaddress]{revtex4-1} %reprint
%\tightenlines

%\draft
\usepackage{etex}
\usepackage{amsmath}
\usepackage{bm}
\usepackage{bbm}
\usepackage{listings}
% % \textwidth 16cm \textheight 23.5cm
% \renewcommand{\baselinestretch}{1.2}
\usepackage{graphicx}
\usepackage{graphics}
\usepackage{epsfig}
\usepackage{color}
\usepackage[dvipsnames]{xcolor}
\usepackage{multirow}
\usepackage[colorlinks]{hyperref}
\usepackage{fancyhdr}
\usepackage{calc}
\usepackage{natbib} %[numbers]
\usepackage{bibentry}

% todo list and commands
%\usepackage{todonotes}
%% to avoid the conflict with amths package % not working
%\makeatletter
%\providecommand\@dotsep{5}
%\makeatother
%\listoftodos\relax
%\usepackage{makeidx}
%\allowdisplaybreaks
%% for eps transfering to pdf.
%\usepackage[update,prepend]{epstopdf}
%\usepackage{ifpdf}
%
%\ifpdf
%   \usepackage{graphicx}
%   \usepackage{epstopdf}
%   \epstopdfsetup{suffix=}
%   \DeclareGraphicsRule{.eps}{pdf}{.pdf}{`epstopdf #1}
%   \pdfcompresslevel=9
%\else
%   \usepackage{graphicx}
%\fi
% subfig
%\usepackage{mwe}
%\usepackage{subfig}
% to fix a figure's position using [H] option of thec figure.
%\usepackage{float}
% to use \lesssim and other math symbols
%\usepackage{amssymb}


% self-defined short-cuts and commands
%\input{Mydef.tex}
\DeclareMathOperator{\tr}{tr}
\newcommand{\dt}[1]{\frac{{\mathrm d} {#1}}{{\mathrm d}t}}
\def\br{\mathbf{r}}
\def\bra#1{\langle{#1}\rvert}%{\mathinner{\langle{#1}\rvert}}
\def\ket#1{\lvert{#1}\rangle}%{\mathinner{\lvert{#1}\rangle}}
\def\Braket#1#2{\mathinner{\langle{#1}\! \mid\! {#2} \rangle}}
%========================================================================================
\newcommand{\erf}[1]{Eq.~(\ref{#1})}
\newcommand{\frf}[1]{Fig.~\ref{#1}}
\newcommand{\srf}[1]{Sec.~\ref{#1}}
\newcommand{\nn}{\nonumber}
\newcommand{\mbf}[1]{\mathbf{#1}}
%========================================================================================
% General quantum mechanics macros
%========================================================================================
\newcommand{\op}[2]{\ket{#1}\bra{#2}}
\newcommand{\expt}[1]{\langle{#1}\rangle}
\newcommand{\dg}{^\dagger}
\newcommand{\smallfrac}[2]{\mbox{$\frac{#1}{#2}$}}
\newcommand{\Tr}{\mbox{Tr}}
%========================================================================================
\newcommand{\expect}[1]{\big\langle #1 \big\rangle}
\newcommand{\eff}{\text{eff}}



% Redefine the tensor command.
%\renewcommand{\tensor}[1]{\boldsymbol{#1}}


%==== Ben's new macros ======
%\newcommand{\srf}[1]{Sec. \ref{#1}}
\newcommand{\half}{\smallfrac{1}{2}}

%==== subscripts ======
\newcommand{\oneD}{{\rm 1D}}
\newcommand{\vac}{{\rm vac}}
\newcommand{\cav}{{\rm cav}}
\newcommand{\inp}{{\rm in}}
\newcommand{\out}{{\rm out}}
\newcommand{\inter}{{\rm int}}
\newcommand{\scs}{{\rm SCS}}
\newcommand{\fwd}{+}
\newcommand{\bwd}{-}
\newcommand{\trans}{+}
\newcommand{\refl}{-}

 %==== operators/moments ======
\newcommand{\der}[1]{\frac{d {#1}}{dt}}
\newcommand{\unittens}{\tensor{\mathbf{I}}}
\newcommand{\poltens}{\hat{\tensor{\boldsymbol{\alpha}}}}
\newcommand{\varz}{\Delta J_3^2}
\newcommand{\jx}{\hat{J}_1}
\newcommand{\jz}{\hat{J}_3}
\newcommand{\shotnoise}{\Delta \mathcal{M}^2 |_{\rm SN}}
\newcommand{\projnoise}{\Delta \mathcal{M}^2_{\rm PN}}
\newcommand{\polcomp}{\hat{K}} % p,p' component of the tensor polarizability

%==== physical parameters ======
\newcommand{\Eamp}{\mathcal{F}_0^{(+)}}
\newcommand{\charpol}{\alpha_0(\Delta_{f\!f'})}
\newcommand{\qaxis}{\mathbf{e}_{\tilde{z}}}
\newcommand{\qangle}{\varphi}
\newcommand{\magic}[1]{\tilde{\omega}_{#1}}
\newcommand{\chiN}{\chi_{N}}
\newcommand{\NA}{N_C}
\newcommand{\chieff}{\chi_{\raisebox{-.1pt}{\tiny $J_3$}}}

%==== scattering and optical pumping rates ====%
\newcommand{\gammauu}{\gamma_{\uparrow \rightarrow \uparrow}}
\newcommand{\gammadd}{\gamma_{\downarrow \rightarrow \downarrow}}
\newcommand{\gammaud}{\gamma_{\uparrow \rightarrow \downarrow}}
\newcommand{\gammadu}{\gamma_{\downarrow \rightarrow \uparrow}}
\newcommand{\gammau}{\gamma_{\uparrow}}
\newcommand{\gammad}{\gamma_{\downarrow}}

%==== effective areas ======
\newcommand{\Ain}{A_{\rm in}}
\newcommand{\Abir}{A_N}

%==== eigenfunctions ======
\newcommand{\eigenf}{\mbf{f}_\eta}
\newcommand{\eigenfp}{\mbf{f}_{\eta'}}
\newcommand{\eigeng}{\mbf{g}_\eta}
\newcommand{\eigengp}{\mbf{g}_{\eta'}}

%==== field operators ======
\newcommand{\awg}{\hat{a}_{b,p}(\omega)}
\newcommand{\awr}{\hat{a}_{m,p}(\omega,\beta)}

%==== colors for editing ======
\newcommand{\change}[1]{{\color{RoyalBlue} #1}}
\newcommand{\comment}[1]{{\color{Maroon} #1}}
\newcommand{\error}[1]{{\color{red} #1}}

% =============================================================================


\begin{document}
\title{Effective optical depth per atom of spin squeezing dynamics using nanophotonic waveguides}
\author{Xiaodong Qi}
\affiliation{Center for Quantum Information and Control, University of New Mexico, Albuquerque, New Mexico 87131, USA}
\author{Ezad Shojaee}
\affiliation{Center for Quantum Information and Control, University of New Mexico, Albuquerque, New Mexico 87131, USA}
\author{Poul S. Jessen}
\affiliation{Center for Quantum Information and Control, University of Arizona, Tucson, Arizona 87521, USA}
\author{Ivan H. Deutsch}
\affiliation{Center for Quantum Information and Control, University of New Mexico, Albuquerque, New Mexico 87131, USA}
\author{Yuan-Yu Jau}
\affiliation{Sandia National Laboratories, Albuquerque, New Mexico 87185, USA}
\date{\today}
\pacs{42.50.Lc, 03.67.Bg, 42.50.Dv, 42.81.Gs}

%================================================================%
\begin{abstract}
We study the strong coupling between photons and atoms that can be achieved in nanophotonic geometries in the dispersive regime to implement efficient quantum interface with neutral atoms.
First, we extend our previous work on spin squeezing and quantum nondemolition (QND) measurement with the birefringence protocol using a nanofiber to a more commonly considered protocol using the Faraday interaction.
We established a general theory to calculate the spin squeezing dynamics and show that $7dB$ of spin squeezing may be achievable with $2500$ atoms trapped $1.8$ times of the fiber radius from the fiber axis, in comparison with $~5dB$ of spin squeezing using the birefringence protocol.
The Faraday interaction protocol does not require a sophisticated search for the magic frequencies and can avoid the difficulty of preparing the atoms on the harder-to-prepare clock states as we did for the birefringence protocol,
and hence is more robust and easier to implement.
Meanwhile, compared to the spin squeezing protocol using Faraday interaction in free space, the nanofiber platform enables us to increase
the photon-atom coupling strength dramatically while to reduce the decoherence due to photon scattering process simultaneously,
and hence can achieve a high peak spin squeezing efficiently.
To achieve an even stronger spin squeezing effect towards a non-Gaussian collective spin state, for example, we propose a generalized optical depth per atom concept applicable to general nanophotonic waveguides and explicitly separate the photon-atom coupling strength from the decoherence characteristic parameter, which can be used to guide the design of novel nanophotonic quantum interfaces and protocols towards efficient quantum control and measurement of atomic states as well as preparing non-Gaussian atomic ensemble states.
We also study the decoherence mechanism for atoms with arbitrary total angular momentum quantum number $f$ with the nanofiber interface, and confirm $f=1$ is the optimal case for the Faraday protocol we study.
Finally, we give an example of analysing a nanophotonic waveguide interface with square cross-section to implement efficiently strong photon-atom coupling and discuss the challenges of using waveguides without cylindrical symmetry beyond optical nanofibers towards efficient photon-atom coupling.
\end{abstract}

\maketitle

%===================INTRODUCTION=====================%
\section{Introduction}

Strong coupling between atoms and photons via nanophotonic structures and such.

The difficulty and challenges.

Our protocol and the structure of this paper.


%========================== Theory ===================================%
\section{Theory} \label{Sec::Theory}

\subsection{Method to calculate spin squeezing dynamics in a QND measurement process}
To study the spin dynamics of the QND measurement process, it is usually valid to calculate the density if the dimension is small. 
For a large ensemble of atoms, however, it requires a big dimension to store the density matrix and it becomes extremely difficult to calculate the evolution of the density operator.
Below, we outline our general method to study the spin dynamics and the spin squeezing evolution without calculating the density operator.

We recall the spin squeezing parameter defined by Wineland {\emph{et al.}}~\cite{Wineland1992},
\begin{align}
\zeta^2 &\equiv \frac{\expect{\hat{F}_\parallel(t=0)}^2}{\Delta F_\perp^2(t=0)} \frac{\Delta F_\perp^2}{\expect{\hat{F}_\parallel}^2},
\end{align}
which only requires to calculate the expectation value of the collective atomic angular momentum operator in parallel with the total atomic angular momentum vector in the generalized Bloch sphere, $ \expect{\hat{F}_\parallel} $, as well as the variance of the collective atomic angular momentum operator perpendicular to the total atomic angular momentum operator, $ \Delta F_\perp^2 $. 
Assume the atom number is $ N_A $, these two quantities can be given by 
\begin{align}
\expect{\Delta F_\perp^2} &= N_A \expect{\Delta f_\perp^2}+\frac{N_A(N_A-1)}{2}\left. \expect{\Delta f_\perp^{(i)}\Delta f_\perp^{(j)}}_s\right|_{i\neq j}\label{eq:DeltaFz2}\\
\expect{\hat{F}_\parallel } &= \sum_i^{N_A} \hat{f}_\parallel ^{(i)}=N_A \hat{f}_\parallel,\label{eq:expectFx}
\end{align}
where the first term of Eq.~\eqref{eq:DeltaFz2} and Eq.~\eqref{eq:expectFx} are solely determined by a symmetric sum over $N_A$ identical spin-$f$ single-body operators, $ \hat{f}_\perp=\hat{f}_\perp^{(i)} $ and $ \hat{f}_\parallel=\hat{f}_\parallel^{(i)} $ with atom labels $ i=1,\cdots,N_A $; the second term of Eq.~\eqref{eq:DeltaFz2} is determined by symmetric two-body covariance terms, $ \left.\expect{\Delta f_\perp^{(i)}\Delta f_\perp^{(j)}}_s\right|_{i\neq j}=\expect{\Delta f_\perp^{(1)}\Delta f_\perp^{(2)}}_s\equiv \expect{\hat{f}_\perp^{(1)}\hat{f}_\perp^{(2)}}_s-\left( \expect{\hat{f}_\perp^{(1)}} \expect{\hat{f}_\perp^{(1)}}\right)_s $, which generate the pairwise entanglement among atoms and eventually yield spin squeezing~\cite{Wang2003Spin}.
Above, we have assumed there is a pairwise exchange symmetry among atoms so that we only care about symmetrized quantities like $ \expect{\Delta f_\perp^{(1)}\Delta f_\perp^{(2)}}_s=\left(\expect{\Delta f_\perp^{(1)}\Delta f_\perp^{(2)}} + \expect{\Delta f_\perp^{(2)}\Delta f_\perp^{(1)}} \right)/2 $. 
We can bring in the spin coherence operator $\hat{\sigma}_{ba}=\ket{b}\bra{a}$ to represent the matrix element of any atomic angular momentum operator $ \hat{f}_m $ ($ m=x,y,z $) of a single atom, and hence the spin squeezing dynamics can be characterized by the expectation value of single-body spin coherence operators $\expect{\hat{\sigma}_{ba}}$ and the symmetric two-body covariances $\expect{\Delta \sigma_{ba}^{(1)}\Delta\sigma_{dc}^{(2)} }_s$. 
%the symmetric three-body correlations $\expect{\Delta \sigma^{(1)}_{b_1a_1}\Delta \sigma^{(2)}_{b_2a_2}\Delta \sigma^{(3)}_{b_3a_3} }_s$ and so on. 
%As higher-order correlations becomes negligible, one can cut off the correlation terms at a certain order.


The stochastic master equation of the QND measurement process can be given by
\begin{align}\label{eq:totaldrhodt}
\dt{\hat{\rho}}=\left.\dt{\hat{\rho}}\right|_{op} + \left.\dt{\hat{\rho}}\right|_{QND}.
\end{align}
It includes two collective spin dynamic processes. 
The first process is the optical pumping dynamics on each individual atom $i$ positioned at $\br'$ which yields the $ \left.\dt{\hat{\rho}^{(i)}}\right|_{op} $ term given by
\begin{align}
\left.\dt{\hat{\rho}^{(i)}}\right|_{op} &=\gamma_s\mathcal{D}^{(i)}\\
&= -\frac{i\gamma_s}{\hbar} \left\{\hat{h}_{\rm loss},\hat{\rho}^{i} \right\} + \gamma_s\sum_q \hat{W}_q(\br')\hat{\rho}^{i}\hat{W}_q(\br'),
\end{align}
where the characteristic photon scattering rate $ \gamma_s\equiv \frac{\Gamma_0\Omega^2}{4\Delta_F}=\frac{\sigma_0}{A_{in}}\frac{\Gamma_0^2}{4\Delta_F^2}\dot{N}_L $ with the effective input mode area $ A_{in}=1/n_g|u_{\mathrm{in}}(\br'\!_\perp)|^2 $.
$\gamma_s$ characterizes the rate of decoherent processes.
The second term on the right-hand side of Eq.\eqref{eq:totaldrhodt} gives raise the collective spin dynamics due to QND measurement,
\begin{align}
\left.\dt{\hat{\rho}}\right|_{QND} &= \sqrt{\frac{\kappa}{4}}\mathcal{H}\left[\hat{\rho} \right]dW + \frac{\kappa}{4}\mathcal{L}\left[ \hat{\rho}\right]dt.
\end{align}
Above, we have defined the measurement strength $\kappa \equiv |\chi|^2\dot{N}_L\equiv \frac{\sigma_0A_{in}}{A_{int}^2}\gamma_s $ determining the rate of the spin squeezing in absence of decoherent processes, where $\dot{N}_L$ is the photon number flux, $\chi$ is the light-atom coupling strength and $A_{int}$ is the effective interaction mode area which can be specified for a particular QND measurement protocol. We have also assumed the measurement backation is a stochastic Weiner process where $dW$ is the increment satisfying $dW^2 = dt$. The conditional dynamics that result from the measurement are generated by the superoperator
\begin{align}
\mathcal{H}\left[ \hat{\rho}\right] &= \hat{F}_\perp\hat{\rho} + \hat{\rho}\hat{F}_\perp -2\expect{\hat{F}_\perp}\hat{\rho}
\end{align}
and the collective Lindblad map due to photon scattering is
\begin{align}
\mathcal{L}\left[ \hat{\rho} \right] &= \hat{F}_\perp\hat{\rho}\hat{F}_\perp-\frac{1}{2}\left(\hat{\rho}\hat{F}_\perp^2+\hat{F}_\perp^2\hat{\rho} \right)=\frac{1}{2}\left[\hat{F}_\perp,\left[\hat{\rho},\hat{F}_\perp \right] \right].
\end{align}

As shown in the equations above, the spin squeezing dynamics is a competition between the coherent squeezing process and all decoherent processes where are characterized by $\kappa$ and $\gamma_s$. 
If we define an effective OD per atom quantity for the spin squeezing dynamics by
\begin{align}
\frac{\mathrm{OD}_{\rm eff}}{N_A} \equiv \frac{\kappa}{\gamma_s}=\frac{\sigma_0A_{in}}{A_F^2},
\end{align}
the peaking spin squeezing dynamics can then be characterized by $\frac{\mathrm{OD}_{\rm eff}}{N_A}$, and the geometry of the spin squeezing protocol can then be roughly designed with the goal to maximize $\frac{\mathrm{OD}_{\rm eff}}{N_A}$ by minimizing $A_{in}$ and maximizing $A_{int}$.  


If one can truncate the spin dynamics up to the two-body correlations, we only need the following two sets of stochastic differential equations:
\begin{subequations}
\begin{align}
d\expect{\hat{\sigma}_{ba}} &=\left. d{\expect{\hat{\sigma}_{ba}}}\right|_{op} + \left. d{\expect{\hat{\sigma}_{ba}}}\right|_{\mathcal{H}}+\left. d{\expect{\hat{\sigma}_{ba}}}\right|_{\mathcal{L}} \\
d\expect{\Delta \sigma_{ba}^{(1)}\Delta \sigma_{dc}^{(2)}}_s &= \left. d{\expect{\Delta \sigma_{ba}^{(1)}\Delta \sigma_{dc}^{(2)}}_s}\right|_{op} + \left. d{\expect{\Delta \sigma_{ba}^{(1)}\Delta \sigma_{dc}^{(2)}}_s}\right|_{\mathcal{H}} + \left. d{\expect{\Delta \sigma_{ba}^{(1)}\Delta \sigma_{dc}^{(2)}}_s}\right|_{\mathcal{L}}.
\end{align}
\end{subequations}

In details, the optical dynamics part can be given by
\begin{align}
\left. \dt{\expect{\hat{\sigma}_{ba}}}\right|_{op} &= \gamma_s\expect{\mathcal{D}^\dagger \left[ \hat{\sigma}_{ba}\right]}\\
&= \gamma_s\sum_{d,c}\tr\left(\mathcal{D}^\dagger \left[ \hat{\sigma}_{ba}\right]\hat{\sigma}_{dc} \right)\expect{\hat{\sigma}_{dc} }\\
\left. \dt{\expect{\Delta \sigma_{ba}^{(1)}\Delta \sigma_{dc}^{(2)}}_s}\right|_{op} &=\gamma_s\expect{\Delta\mathcal{D}^\dagger[\hat{\sigma}_{ba}^{(1)}]\Delta\sigma_{dc}^{(2)} }_s + \gamma_s\expect{\Delta\sigma_{ba}^{(1)}\Delta\mathcal{D}^\dagger[\hat{\sigma}_{dc}^{(2)}] }_s\\
&= \gamma_s\sum_{m,n}\tr\left(\mathcal{D}^\dagger[\hat{\sigma}_{ba}]\hat{\sigma}_{mn} \right)\expect{\Delta \sigma_{mn}^{(1)}\Delta \sigma_{dc}^{(2)} }_s + \gamma_s\sum_{m,n}\tr\left(\mathcal{D}^\dagger[\hat{\sigma}_{dc}]\hat{\sigma}_{mn} \right) \expect{\Delta \sigma_{ba}^{(1)}\Delta \sigma_{mn}^{(2)} }_s.
\end{align} 
Similarly, we will need the one- and two-body correlations due to the $ \mathcal{H} $ and $ \mathcal{L} $ superoperators given by the following.
\begin{subequations}
\begin{align}
\left.d\expect{\hat{\sigma}_{ba}}\right|_\mathcal{H} &=\sqrt{\frac{\kappa}{4}}\expect{\mathcal{H}^\dagger\left[\hat{\sigma}_{ba} \right]}dW \\
\left.d\expect{\hat{\sigma}_{ba}}\right|_\mathcal{L} &= \frac{\kappa}{4}\expect{\mathcal{L}^\dagger\left[\hat{\sigma}_{ba} \right]}dt
\end{align}
\end{subequations}
In principle, the two-body covariance terms can be coupled to high-order many-body terms. 
In our case, we assume the state of the ensemble can be well captured in the symmetric Gaussian state limit, and hence the two-body covariance equations due to the collective measurement can be given by
\begin{subequations}
\begin{align}
\left.d\expect{\Delta \sigma_{ba}^{(1)} \Delta \sigma_{dc}^{(2)}} \right|_\mathcal{H} &= -\kappa\expect{\Delta\sigma_{ba}^{(1)}\Delta F_\perp }_s \expect{\Delta F_\perp \Delta \sigma_{dc}^{(2)} }dt \\
\left.d\expect{\Delta \sigma_{ba}^{(1)} \Delta \sigma_{dc}^{(2)}}\right|_\mathcal{L} &= 0.
\end{align}
\end{subequations}

\subsection{A spin squeezing protocol using Faraday interaction with spin coherent states}
\comment{To be done: Double check all equations in this section.}

Beside the case we have discussed in our previous work, we propose here a spin squeezing protocol based on Faraday interaction.
We define the fiber axis or $ z $-axis as the quantization, and set $ \hat{F}_\perp=\hat{F}_z $ as the squeezing operator. 
Let us consider an ensemble of $ ^{133} $Cs atoms initially prepared as a spin coherent state (SCS) with every atom in the stretch state of the $ 6S_{1/2}$ $f=4 $ ground manifold in the $ x $-basis, where the quantization $ x $-direction is along the diagonal or $ \phi=\pi/4 $ direction in the $ H $-$ V $ Cartesian coordinate system sitting on the fiber axis.
We call the initial state of one atom as the fiducial state or $ \ket{\phi_0}=\ket{\uparrow}_x = \ket{6S_{1/2},f=4,m_x=4} $ and the collective SCS is $ \ket{\Psi_0}=\ket{\uparrow}_x^{\otimes N_A} $.
We define the coupled state by applying the individual squeezing operator $ \hat{f}_z$ on the fiducial state $\ket{\uparrow}_x $.
In our case, the coupled state can be defined as $ \ket{\downarrow}_x=\ket{6S_{1/2},f=4,m_x=3} $.
Similarly, to include the transfer of coherence in the squeezing process, we define the transfer state as $ \ket{T}_x=\ket{6S_{1/2},f=4,m_x=2} $ as a result of applying $ \hat{f}_z$ on the coupled state. 
For simplicity, we remove all the subscript $ x $ of quantum states and always work in the $ x $-basis if no explicit notations for the Faraday interaction protocol. 

Following the process demonstrated in our previous work~\cite{Qi2016}, the light-atom interaction Hamiltonian can be written as
\begin{align}
\hat{H} &= \hbar \left[\left(\chi_{RR\uparrow} + \chi_{RR\downarrow} +\chi_{LL\uparrow}+\chi_{LL\downarrow} \right)\hat{F}_0\hat{S}_0 \right.\nonumber\\
&\quad+\left(\chi_{RR\uparrow} + \chi_{RR\downarrow} -\chi_{LL\uparrow}-\chi_{LL\downarrow} \right)\hat{F}_0\hat{S}_3\nonumber\\
&\quad+\left(\chi_{RR\uparrow} + \chi_{LL\uparrow} -\chi_{RR\downarrow}-\chi_{LL\downarrow} \right)\hat{F}_3\hat{S}_0\nonumber\\
&\quad+\left(\chi_{RR\uparrow} - \chi_{RR\downarrow} +\chi_{LL\downarrow}-\chi_{LL\uparrow} \right)\hat{F}_3\hat{S}_3\nonumber\\
&\quad+i\left(\chi_{LR\uparrow} - \chi_{RL\uparrow} +\chi_{RL\downarrow}-\chi_{RL\downarrow} \right)\hat{F}_0\hat{S}_1\nonumber\\
&\quad+\left(\chi_{RL\uparrow} + \chi_{LR\uparrow} +\chi_{RL\downarrow}+\chi_{LR\downarrow} \right)\hat{F}_0\hat{S}_2\nonumber\\
&\quad+i\left(\chi_{LR\uparrow} - \chi_{RL\uparrow} +\chi_{RL\downarrow}-\chi_{LR\downarrow} \right)\hat{F}_3\hat{S}_1\nonumber\\
&\quad+\left.\left(\chi_{LR\uparrow} + \chi_{RL\uparrow} -\chi_{LR\downarrow}-\chi_{RL\downarrow} \right)\hat{F}_3\hat{S}_2 \right]\\
&=\hbar\sum_{i,j=0}^3 \chi_{ij}\hat{F}_i\hat{S}_j,
\end{align}
where $ \hat{S}_i $ are the Stokes vector operators of the light indicating its polarization, and the mode-atom coupling strengths
\begin{align}
\chi_{pp'm} 
%&=-\frac{2\pi \omega}{v_g}\mathbf{u}_{p'}^*(r'\!_perp,\phi')\cdot \bra{f=4,m}\hat{\tensor{\alpha}}\ket{f=4,m}\cdot \mathbf{u}_p(r'\!_perp,\phi')
&= \sum_{f'} \frac{n_g\sigma_0}{4}\frac{\Gamma_{f'}}{\Delta_{ff'}+i\Gamma_{f'}/2}\cdot \left\{ C_{jj'ff'}^{(0)}\mathbf{u}_p^*(r'\!_\perp)\cdot \mathbf{u}_{p'}(r'\!_\perp)+imC_{jj'ff'}^{(1)}\left(\mathbf{u}_p^*(r'\!_\perp)\times\mathbf{u}_{p'}(r'\!_\perp) \right)\cdot \hat{e}_{\tilde{z}} \right. \nonumber\\
&\left.+C_{jj'ff'}^{(2)}\left[\mathbf{u}_p^*(r'\!_\perp)\cdot \mathbf{u}_{p'}(r'\!_\perp)\left(\frac{f(f+1)}{6}-\frac{m^2}{2} \right)+\mathbf{u}_p^*(r'\!_\perp)\cdot (\hat{e}^*_{\tilde{z}}\hat{e}_{\tilde{z}})\cdot \mathbf{u}_{p'}(r'\!_\perp)\left(\frac{3m^2}{2}-\frac{f(f+1)}{2} \right) \right] \right\}
\end{align}
with the right(R)- and left(L)-circularly polarized fundamental fiber modes, $ \mathbf{u}_p(r'\!_\perp) $, at the atom position $ \br'=(r'\!_\perp,\phi',z') $ defined in the appendix of our previous paper~\cite{Qi2016}. 
\comment{Introduce other parameters or make a glossary in the appendix?}

To implement a pure Faraday interaction with the minimum impact from Birefringence effect due to the presence of the nanofiber, it is necessary to position atoms at places where the Birefringence effect is vanished. 
We consider working in the far-detuning regime where the Birefringence effect due to the atomic tensor polarizability can be ignored~\cite{Deutsch2010a}. 
Based on the symmetry of the fiber modes, with a linearly polarized $H$ ($x$)- and $V$ ($ y $)-modes inputs to form a diagonally polarized $ D $-mode, the optimal position of atoms with vanished Birefringence effect due to the intensity difference of the local $ H $- and $ V $-modes could be along the $ \phi'=n\pi/4 $ ($ n=1,3,5,7 $) radial directions. 

In the far-detuning regime, we can also set the decay rates of the excited levels $ \Gamma_{f'}= \Gamma_0$ as constant for all $ f' $ in the same fine structure level.
By ignoring the tensor coupling strength related to $ C_{jj'ff'}^{(2)} $~\cite{Deutsch2010a} at the potential optimal trapping positions discussed above, the Faraday interaction coupling strength can be simplified as
\begin{align}
\chi_{33} &= -\sum_{f'}n_g\sigma_0\frac{\Gamma_0}{\Delta_{ff'}+i\Gamma_0/2}C_{jj'ff'}^{(1)}u_{r\!_\perp}(r'\!_\perp)u_\phi(r'\!_\perp)\\
&=\frac{\sigma_0}{A_F}\frac{\Gamma_0}{\Delta_F},
\end{align}
where the effective Faraday interaction mode area $ A_F=1/2n_g|u_{r\!_\perp}(r'\!_\perp)u_\phi(r\!_\perp)| $, and the effective detuning $ \Delta_F=\sum_{f'}\frac{-C_{jj'ff'}^{(1)}}{\Delta_{ff'}} $.
The measurement strength is now defined as
\begin{align}
\kappa\equiv|\chi_{33}|^2\dot{N}_L=\frac{\sigma_0A_{in}}{A_F^2}\gamma_s,
\end{align}
where the characteristic photon scattering rate $ \gamma_s\equiv \frac{\Gamma_0\Omega^2}{4\Delta_F}=\frac{\sigma_0}{A_{in}}\frac{\Gamma_0^2}{4\Delta_F^2}\dot{N}_L $ and the effective mode area $ A_{in}=1/n_g|u_{\mathrm{in}}(\br'\!_\perp)|^2 $.
Now we can define the OD per atom for the Faraday interaction using SCS by
\begin{align}
\frac{\mathrm{OD}}{N_A} \equiv \frac{\kappa}{\gamma_s}=\frac{\sigma_0A_{in}}{A_F^2}.
\end{align}

The optimal position of the atom should be chosen so that $ A_{\mathrm{in}} $ reaches the minimum value, which yields $ \phi'=3\pi/4 $ or $ 7\pi/4 $ where $ A_{\mathrm{in}}=1/2n_g|u_\phi(r'\!_\perp)|^2 $.

The number of equations needed to include different subspace...

\subsection{Decoherence and total angular momentum quantum number of atoms}

%=================== Simulations with waveguides =====================%
\section{Discussions} \label{Sec::Discussions}
\comment{Leave discussions on how does the internal structure of atoms and the total atomic quantum number affect the spin squeezing parameter for future works.}


%=================== A toy model with a spin-1/2 system =====================%
\subsection{Spin squeezing with a spin-$1/2$ system} \label{Sec::squeezingwithspinhalfsystems}


%============================ spin squeezing with nanofibers ==============================%
\subsection{QND measurement and spin squeezing with nanofibers} \label{Sec::Nanofiber}

Optimal choice of atom position.

Squeezing results.

Comparison with free space and birefringence.


%=================== Comparison and generalization =====================%
\subsection{Spin squeezing with rectangular waveguide and general criteria in optimizing spin squeezing effect} \label{Sec::Waveguide}

Generalized OD/$N_A$ using Green function method (or may be just with two orthogonal mode bases.)

Theoretical upper bounds for the Birefringence and Faraday protocols using a waveguide:

1. The maximum achievable OD$ /N_A $ can be defined when the two orthogonal $ H $- and $ V $-modes are purely linearly polarized (transverse modes).

2. The relationship between the upper bound OD$ /N_A $ and spin-squeezing parameter due to the waveguide effect and effective mode area at the atom position. 
General rules to find the optimal choice of atom positions for the Birefringence and Faraday spin squeezing protocols.


As one example of near-linearly polarized modes, we can consider using a dielectric waveguide with square intersection. 
Coupling strength, decoherence and spin squeezing analysis based on the modes of the squared waveguides to show improvement compared with the nanofiber geometry...

Relation of OD$ /N_A $ to cooling efficiency and fictitious magnetic field applications.

%====== SECTION: Summary and outlook ======%
\section{Summary and Outlook} \label{Sec::Conclusion}


ACKNOWLEDGMENTS
We thank the UNM Center for Advanced Research Computing for computational resources used in this work.
This work was support by the NSF, under grants PHY-1212445, xxxxx.
\bibliography{refs/Archive}


%=========== APPENDIX ===========%
\begin{appendix}

%===================APPENDIX: Hamiltonian =====================%
\section{Faraday interaction Hamiltonian} \label{Appendix::FaradayInteractionHamiltonian}
In the Faraday interaction spin squeezing protocol, we define the fiducial, coupled and transfer states by $ \ket{\uparrow}=\ket{f=4,m=4} $, $ \ket{\downarrow}=\ket{f=4,m=3} $ and $ \ket{T}=\ket{f=4,m=2} $ respectively. 
In the $ x $-basis, a set of spin operators projected onto the truncated qutrit subspace spanned by these three basis states can be defined by
\begin{align}
\hat{f_x} &= -f \ket{\uparrow}\bra{\uparrow} -(f-1)\ket{\downarrow}\bra{\downarrow}-(f-2)\ket{T}\bra{T},\\
\hat{f_y} &=i\left[\sqrt{\frac{f}{2}}\left(\ket{\downarrow}\bra{\uparrow}-\ket{\uparrow}\bra{\downarrow}\right) +\sqrt{\frac{2f-1}{2}}\left(\ket{T}\bra{\downarrow}-\ket{\downarrow}\bra{T} \right) \right] ,\\
\hat{f_z} &= \sqrt{\frac{f}{2}}\left(\ket{\downarrow}\bra{\uparrow}+\ket{\uparrow}\bra{\downarrow}\right) +\sqrt{\frac{2f-1}{2}}\left(\ket{T}\bra{\downarrow}+\ket{\downarrow}\bra{T} \right).
\end{align}
We also define a set of Stokes operators as below,
\begin{align}
\hat{S}_0 &= \smallfrac{1}{2}\big[ \hat{a}^\dag_H(t) \hat{a}_H(t)+\hat{a}^\dag_V(t) \hat{a}_V(t) \big],\\
\hat{S}_1 &= \smallfrac{1}{2}\big[ \hat{a}^\dag_H(t) \hat{a}_H(t)-\hat{a}^\dag_V(t) \hat{a}_V(t) \big],\\
\hat{S}_2 &= \smallfrac{1}{2}\big[ \hat{a}^\dag_H(t) \hat{a}_V(t)+\hat{a}^\dag_V(t) \hat{a}_H(t) \big],\\
\hat{S}_3 &= \smallfrac{1}{2i}\big[ \hat{a}^\dag_H(t) \hat{a}_V(t) -\hat{a}^\dag_V(t) \hat{a}_H(t) \big],
\end{align}
where the photon annihilation operators of the horizontally and vertically polarized modes denoted with subscription $ H $ and $ V $ respectively are related to the left- and right-circularly polarized modes denoted with subscription $ L $ and $ R $ by $ \hat{a}_{H}=(\hat{a}_L+\hat{a}_R) /\sqrt{2}$ and $ \hat{a}_{V}=i(\hat{a}_R-\hat{a}_L)/\sqrt{2} $.
The local electric field at $ (r\!_\perp,\phi,z) $ with a diagonally polarized incident mode can be given by
\begin{align}
\hat{\mathbf{E}}^{(+)}(r\!_\perp,\phi,z;t) &= \sum_{b,p} \sqrt{ \frac{2 \pi \hbar \omega_0}{ v_g} } \mathbf{u}_{b,p}(r\!_\perp,\phi) \hat{a}_{b,p}(z,t)  e^{i b \beta_0 z}.
\end{align}

%===================APPENDIX: Photon scattering and optical pumping rates =====================%
\section{Photon scattering and optical pumping rates} \label{Appendix::Rates}

In this Appendix we give the explicit expressions for the photon scattering rates used in Sec.~\ref{Sec::QNDMeasurement} following the formalism given in~\cite{deutsch_quantum_2010}.  The total rate of photon scattering by an atom in the clock state $\ket{f,0}$ is
	\begin{equation}\label{Eq::gammaf}
		\gamma_{f}=
		%- \frac{2}{\hbar} {\rm Im} \big[ \bra{f,0} \hat{h}_{\rm eff}\ket{f,0} \big] ,
	\end{equation}
where the effective non-Hermitian light-shift Hamiltonian for one atom is
\begin{align}
\hat{h}_{\rm eff} = - \hat{\mathbf{E}}^{(-)}_{\rm in}(\mathbf{r}' ; t ) \cdot \poltens \cdot \hat{\mathbf{E}}^{(+)}_{\rm in}(\mathbf{r}' ;t )
\end{align}
as follows from \erf{Eq::LightShiftHam}, where $\charpol = -\frac{\sigma_0}{8\pi k_0}\frac{\Gamma}{\Delta_{ff'}+i\Gamma/2}$ is the complex polarizability and the irreducible tensor operator $ \hat{\tensor{\mbf{A}}}(f,f') $ is given in \erf{Eq::PolarizabilityIrrep}.


The rate of optical pumping between clock states $\ket{f,0} \rightarrow \ket{\tilde{f},0}$ is
	\begin{equation}\label{Eq::gammaff}
		\gamma_{f \rightarrow \tilde{f} }
		%=\sum_{q}\big| \bra{\tilde{f},0} \hat{W}_q^{\tilde{f}f} \ket{f,0} \big|^2,
	\end{equation}
where $ \hat{W}_q^{\tilde{f}f} = \frac{1}{\gamma_s}\sum_{f'}\frac{\Omega/2}{\Delta_{f'\tilde{f}}+i\Gamma/2}(\mathbf{e}_q^*\cdot\hat{\mathbf{D}}_{\tilde{f} f'} )(\mathbf{e}_{\rm in}\cdot \hat{\mathbf{D}}^\dagger_{f'f} ) $ are the Lindblad jump operators for optical pumping between ground levels $ f\rightarrow \tilde{f} $~\cite{deutsch_quantum_2010}.
Each jump operator $\hat{W}_q^{\tilde{f}f}$ is associated with absorption of the probe photon polarized along $ \mathbf{e}_{\rm in} $ followed by spontaneous emission of a photon with polarization $ \mathbf{e}_q $, where $q= \{0,\pm 1\}$ labels spherical basis elements for $\pi$ and $ \sigma_\pm$ transitions.

To find the dependence on the input field intensity, we define a characteristic photon scattering rate, $\gamma_s \equiv \frac{\Gamma\Omega^2}{4\Delta_{J_3}^2}= \frac{\sigma_0}{A_{\rm in}}\frac{\Gamma^2}{4 \Delta_{J_3}^2} \dot{N}_L $, with Rabi frequency $ \Omega=2\bra{j}|d|\ket{j'}\mathcal{E}^{(+)}_{\rm in}/\hbar $, reduced optical dipole matrix element $\bra{j}|d|\ket{j'}$, and field amplitude $ \mathcal{E}^{(+)}_{\rm in}=|\mathbf{E}_{\rm in}^{(+)}(\br')| $.
Eqs.~\eqref{Eq::gammaf} and~\eqref{Eq::gammaff} yield,
\begin{subequations}
	\begin{align}
		\gamma_f 
		%&=n_g\dot{N}_L  \sum_{f'} \sigma (\Delta_{ff'} ) \mathbf{u}^*_\inp(\br'_\perp)\cdot \bra{f,0} \hat{\tensor{\mbf{A}}}(f,f') \ket{f,0}  \cdot \mathbf{u}_\inp(\br'_\perp)\\
		%&\approx  \gamma_s \sum_{f'} \frac{\Delta_{J_3}^2}{\Delta_{ff'}^2}\sum_q \big| o_{jf}^{j'f'}C_{f'q}^{f0;1 q} \big|^2 \mathbf{e}_q^* \cdot (\mathbf{e}_{\rm in}\mathbf{e}_{\rm in}^* )\cdot \mathbf{e}_q,
	\end{align}
\end{subequations}
	\begin{align}
		\gamma_{f \rightarrow \tilde{f}}
		%&\approx \gamma_s \sum_{f'} \frac{\Delta_{J_3}^2}{\Delta_{ff'}^2}\sum_q \big| o_{j\tilde{f}}^{j'f'} o_{jf}^{j'f'}C_{f'q}^{\tilde{f}0;1 q}C_{f'q}^{f0;1q} \big|^2 \mathbf{e}_q^* \cdot (\mathbf{e}_{\rm in}\mathbf{e}_{\rm in}^* )\cdot \mathbf{e}_q,
	\end{align}
where $ \sigma (\Delta_{ff'} )  = \sigma_0 \Gamma^2/4\Delta^2_{f' f}$ is the the scattering cross section at the probe detuning, $ C_{f'q}^{f0;1 q}=\Braket{f'q}{f0;1q}$ are the Clebsch-Gordan coefficients, and
\begin{equation}
\big| o_{jf}^{j'f'} \big|^2=(2j'+1)(2f+2) \bigg\{
\begin{array}{ccc}
f' & 7/2 & j' \\
 j & 1 & f
 \end{array}
 \bigg\}
\end{equation}
are the relative oscillator strengths determined by the relevant Wigner 6-$J$ symbol.

%===================APPENDIX: Equations of motion =====================%
\section{Modified atomic decay rates in presence of a nanophotonic waveguide} \label{Appendix::decayratesimulation}
The key is to calculate the Green's function tensor (GFT) of the nanophotonic waveguide structure. 
One can prove that the imaginary part of the total GFT responded at the atom position $\mathbf{r}$ is semidefinite and can be decomposed into its eigen basis by
\begin{align}
\mathrm{Im}\left[\tensor{G}(\br,\br)\right] &=\sum_{i=1,2,3} g_i\hat{e}_i\hat{e}_i^*,
\end{align}
where $\hat{e}_i$ $(i=1,2,3)$ are the three eigen vector of the tensor and $g_i\ge 0$ are the corresponding eigenvalues.

We also know that $\mathrm{Im}[\tensor{G}(\br,\br)]=\mathrm{Im}[\tensor{G}_{rad}(\br,\br)]+\mathrm{Im}[\tensor{G}_{gyd}(\br,\br)]$ and the guided mode contribution part $\mathrm{Im}[\tensor{G}_{gyd}(\br,\br)]$ can be calculated by
\begin{align}
\mathrm{Im}[\tensor{G}_{gyd}(\br,\br)] &= \frac{\omega}{n_g} \cdots
\end{align}
The total modified decay rates can be calculated using the guided mode contribution part plus the LDOS of the radiation mode contribution generated by dipoles orientated along the three eigen vector directions numerically via boundary element method (BEM).


%===================APPENDIX: Equations of motion =====================%
\section{Derivation of the equations of motion for the moments} \label{Appendix::OpticalPumping}

In this Appendix we provide, for reference, the detailed equations to describe the QND measurement and spin squeezing dynamics.


\end{appendix}

\end{document}
